\section{Diskussion}
\label{sec:Diskussion}
\subsection{Diskussion des Materials}
\label{sec:dis1}
Der Begriff "Federstahl" ist nicht eindeutig.
Die Messungen ergeben Werte, die mit den Literaturwerten des genormten Federstahls \textit{X12CrNi18-8}, Werkstoff \textit{1.4310} \cite{federstahl} verglichen werden.
Die geringe Abweichung der Messung von der Theorie erhärten die Vermutung über das Material.
\subsection{Diskussion der Fehler}
Es sind die exakten Werte für das Schubmodul $G$ und für die Horizontalkomponente des Erdmagnetfeldes $B_\text{Erde}$ bekannt.
Der Abweichung des Schubmodules, wie in Abschnitt \ref{sec:dis1} beschrieben wird, gibt Aufschluss über das Material und weist daher a priori eine geringe Abweichung auf.
Der in Abschnitt \ref{sec:auswertung3} bestimmte Wert des Erdmagnetfeldes weicht mit etwa $68,65\%$ stark von der Literaturangabe ab.
Die Literaturwerte stammen von einer Sekundärquelle, sodass auf exakte Information nicht zugegriffen werden kann.
Weiter hängt der Wert von dem Inklinationswinkel ab, der in der Quelle für Deutschland gemittelt angenommen wird.

Die Methode zur Bestimmung der Schwingungsdauer in Abschnitt \ref{sec:durchfuehrung2} zeigt dadurch ihre Eignung, dass die in Tabellen \ref{tab:T_G}, \ref{tab:T_m} und \ref{tab:T_E} aufgeführten Messwerte nur geringfügig untereinander abweichen.
Es wird daher angenommen, dass Schwankungen durch geringe äußere Einflüsse entstehen.
