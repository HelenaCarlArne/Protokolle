\newpage
\section{Auswertung}
\label{sec:Auswertung}


\begin{table}
	\centering
	\begin{tabular}{c cc}
	\toprule
	{Drahtdurchmesser} & \multicolumn{2}{c}{Drahtlängen} \\
	{$d/\:\si{\milli\meter}$} & {$l_1/\:\si{m}$} & {$l_2/\:\si{m}$}\\
	\midrule
0,190 & 0,551 & 0,500 \\
0,189 & 0,551 & 0,510 \\
0,191 & 0,550 & 0,500 \\
0,193 & 0,551 & 0,520 \\
0,195 & 0,551 & 0,520 \\
	\bottomrule
	\end{tabular}
	\caption{Abmessungen des Drahtes.}
	\label{tab:draht}
\end{table}

\begin{table}
	\centering
	\begin{tabular}{c}
	\toprule
	{$T_G/\:\si{\second}$}\\
	\midrule
18,585\\
18,598\\
18,593\\
18,588\\
18,585\\
18,588\\
18,589\\
18,592\\
18,594\\
18,594\\
18,598\\
	\bottomrule
	\end{tabular}
	\caption{Schwingungsdauern zur Berechnung des Schubmoduls $G$.}
	\label{tab:T_G}
\end{table}

\begin{table}
	\centering
	\begin{tabular}{cc}
	\toprule
	{$I/\:\si{\ampere}$} & {$T_m/\:\si{\second}$}\\
	\midrule
0,1 & 11,424 \\
0,1 & 11,441 \\
0,1 & 11,386 \\
0,1 & 11,411 \\
0,1 & 11,384 \\
0,2 &  9,526 \\
0,2 &  9,513 \\
0,2 &  9,490 \\
0,2 &  9,480 \\
0,2 &  9,473 \\
0,4 &  6,889 \\
0,4 &  6,949 \\
0,4 &  6,953 \\
0,4 &  6,955 \\
0,4 &  6,954 \\
0,6 &  5,837 \\
0,6 &  5,785 \\
0,6 &  5,841 \\
0,6 &  5,831 \\
0,6 &  5,844 \\
0,8 &  5,171 \\
0,8 &  5,169 \\
0,8 &  5,172 \\
0,8 &  5,162 \\
0,8 &  5,162 \\
	\bottomrule
	\end{tabular}
	\caption{Schwingungsdauern zur Berechnung des magnetischen Moments $G$.}
	\label{tab:T_m}
\end{table}

\begin{table}
	\centering
	\begin{tabular}{c}
	\toprule
	{$T_G/\:\si{\second}$}\\
	\midrule
18,413\\
18,411\\
18,390\\
18,409\\
18,379\\
18,404\\
18,395\\
18,399\\
18,396\\
18,391\\
	\bottomrule
	\end{tabular}
	\caption{Schwingungsdauern zur Berechnung des Schubmoduls $G$.}
	\label{tab:T_G}
\end{table}
Das unmenschlich, widerliche AA-Gesicht von Lars Hoffmann hat diesen PC verflucht, weil die Besitzerin sehr fies war. Nun wird zu einem Zeitpunkt t>1h sein Zorn auf dich herniederfahren.


Arne war's...
