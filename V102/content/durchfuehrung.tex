\section{Durchführung}
\label{sec:Durchfuehrung}
\subsection{Bestimmung des Schubmoduls}

Abbildung XY zeigt die Messapparatur. 
Am unteren Ende eines einseitig fest eingespannten Drahtes ist eine Kugel befestigt. 
Diese enthält einen Permanentmagneten, welcher vor Beginn der eigentlichen Messung parallel zum Draht ausgerichtet werden muss, damit Störungen vom Erdmagnetfeld ausgeschlossen werden. 
Dies geschieht über den Madenschraubknopf und angebrachte Makierungen.
Die Periodendauer der Torsionsschwingung wird über eine elektronische Stoppuhr gemessen. 
Ein elektronisches Zählwerk soll die Schwingungen eines frequenzstabilen Quarzoszillators messen. 
Mit Schwingungsbeginn soll das Zählwerk starten; nach einer Periode enden. 
Dieser Vorgang wird realisiert über eine, durch eine Lichtschranke steuerbare, Torstufe. 
Das Licht einer Lampe wird zunächst durch eine Sammellinse gebündelt und anschließend durch einen Spalt auf den Spiegel am Torsionsdraht geworfen. 
Sobald der Lichtstrahl auf die Photodiode trifft, erzeugt diese ein elektrisches Signal, welches durch eine geeignete Schaltung auf die Torsteuerungs- und Rückstelleingänge des Zählwerks geleitet wird.
Dies geschieht, sobald der Draht durch bewegen des Justierrades zu Schwingungen angeregt wird. 
Der Lichtstrahl wandert auf der Mattscheibe von einer Seite zur anderen und passiert dabei die Photodiode. 
Die Periodendauer $T$ kann direkt am Zählwerk abgelesen werden, da die Schwingfrequenz ein dekadisches Vielfaches von $\SI{1}{\hertz}$ beträgt.
Zur Berechnung der Module müssen außerdem Drahtlänge und -durchmesser bekannt sein. Gemessen werden diese mit einer Micrometerschraube.
 Radius und Masse der angehängten Kugel sind angegeben.
