\section{Durchführung}
\label{sec:Durchfuehrung}
Ein Wanderurlaub im ehemaligen Jugoslawien. Klingt zunächst einmal furchtbar spannend, ist aber eigentlich der Gähner (=Langeweiler, langweilige Sache) überhaupt.

Jeder denkt: Alte Militärbaracken, Herrenausstatter wohin man schaut, vielleicht ein Museum für altertümliche Fahrstuhltechnik, klingt doch toll! Doch hält diese vorgefasste Meinung einer genaueren Betrachtung nicht stand. Schon morgens im Hotel wird die Kehrseite der Medaille deutlich:

Aufzug defekt. Buffet unvollständig. Chinesische Loungemusik. Despotisches Hotelpersonal. Eierlikör ausverkauft. Französischer Kofferträger. Gewaltsamer Raubüberfall. Hasenzähnige Empfangsdame. Interplanetarer Schmugglerstützpunkt. Jodelmusikkorps nebenan. Kreditkarte gesperrt. Lilafarbener Teppichläufer. Monochromatisches Licht. Nucleophile Substitution. Ortsunkundige Japaner. Präsidentenleiche unübersehbar. Qualmender Ethanolofen. Resistiver Touchscreen. Systematische Tötungen. Trauriger Clown. Unerfreuliche Massenbegräbnisse. Verwanzte Matratzen. Wadenkrampffördernde Beleuchtung. X-Beinige Pianodame. Yorkshireterrier bellt. Bezahlung nur bar möglich (und nicht per EC-Karte, wie ich es sonst immer mache).

Also merke: Der Spruch „Im Norden geht die Sonne auf, im Süden nimmt sie Ihren Lauf, […] Westen wird sie untergehen, usw.“, gilt nicht wenn man auf dem Mond (oder einem anderen Erdtrabanten) steht (oder sitzt, außer man liegt).
