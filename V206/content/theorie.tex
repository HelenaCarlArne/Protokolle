\section{Ziel}
\label{sec:Ziel}
Ziel des Versuches ist es, eine Wärmepumpe zu charakterisieren, mit deren Hilfe Wärme einem kalten Reservoir entzogen und einem warmen Reservoir hingeführt werden kann.
Hierzu wird die Güteziffer $\nu$, der Massedurchsatz $\frac{\Delta{m}}{\Delta{t}}$ und die mechanische Leistung des verwandten Kompressors $W_\text{mech.}$ bestimmt.
\section{Theorie}
\label{sec:Theorie}
Ohne von außen Arbeit aufzubringen, gleicht sich ein Temperaturunterschied zwischen zwei Reservoiren so aus, 
dass Wärme von dem warmen in das kalte Reservoir strömt.
Weiter existiert nach dem zweiten Hauptsatz der Thermodynamik keine Maschine, deren einzige Wirkung darin besteht, Wärme von einem kalten in ein warmes Reservoir zu transportieren\cite{hauptsaetzederthermodynamik}.
Die zusätzliche Arbeit $A$, die zur Unterhaltung dieses umgekehrten Wärmestromes erforderlich ist, wird nach dem ersten Hauptsatz der Thermodynamik dem warmen Reservoir neben der transportierten Wärme $Q_2$ %%aus kalten Reservoir $R_1$ %%
zugeführt.
Es sei $Q_1$ die von dem warmen Reservoir aufgenommene Wärme mit
\begin{equation}
	Q_1 = Q_2+A.
	\label{eq:waermetransport}
\end{equation}
Die Güteziffer $\nu$ \footnote{Auch als \textit{Coefficient of Performance} (COP) gemäß EN 14511 bekannt} ist das Verhältnis der transportierten Energie $Q_1$ und der hierzu aufgewandten Kompressorarbeit $A$,
\begin{equation}
	\nu = \frac{Q_1}{A}.
	\label{eq:gueteziffer}
\end{equation}

\noindent Die Änderung der Wärmemengen $\mathup{d}Q$ ist für das wärmere Reservoir positiv, für das kältere Reservoir negativ.
Mit der Annahme, dass sich die Temperaturen der Reservoire nicht ändern, kann die Wärmemenge mit der reduzierten Wärmemenge $\int \frac{\mathup{d}Q}{T}$ beschrieben werden. 
Kann weiter die durch den Prozess aufgenommene Wärme $Q_1$ durch einen umgekehrten Prozess wieder vollständig zu $Q_2 + A$ zurückgewonnen werden, das heißt, dass die Wärme nicht aus dem idealen, isolierten System tritt, gilt
\begin{equation}
	\frac{Q_1}{T_1}-\frac{Q_2}{T_2} = 0.
	\label{eq:redwaemre}
\end{equation}
Mit \eqref{eq:redwaemre} gilt für die Güteziffer
\begin{equation}
	\nu_{\text{ideal}} = \frac{T_1}{T_1-T_2}.
	\label{eq:gueteziffer_ideal}
\end{equation}
In einem realen System ist die Änderung $\mathup{d}Q_1$ größer als die Änderung $\mathup{d}Q_2$, das heißt, dass die dem kühlem Reservoir entnommene Wärme nicht vollständig in das wärmere Reservoir übertragen wird.
Daher gilt für reale Systeme
\begin{align}
	 0 &< \frac{Q_1}{T_1}-\frac{Q_2}{T_2}.\\
	\intertext{und analog}
	\nu_{\text{real}} &< \frac{T_1}{T_1-T_2}
\end{align}
