\section{Auswertung}
\label{sec:Auswertung}
\subsection{Bestimmung der Schallgeschwindigkeit}
In den folgenden Tabellen sind die Schalllaufzeiten in den Acrylzylindern verschiedener Höhe aufgetragen.
Die Maße der Zylinder ist in Tabelle () aufgetragen.

Da die Zeiten bei der Impuls-Echo-Methode die Laufzeiten für den Hin- und Rückweg beschreiben, wird für die folgenden Betrachtungen die Hälfte der Zeiten benützt. 
Hierfür wird in Gleichung \eqref{eq:geschwindigkeit} der Faktor $k$ eingeführt, der für die Impuls-Echo-Methode konsequent $k=2$ und für die Durchschallmethode konsequent $k=1$ beträgt.
Es gilt die Formel
\begin{equation}
	c=k*\frac{s}{t}
	\label{eq:geschwindigkeit}
\end{equation}

In den folgenden Diagrammen ist die Höhe der Zylinder gegen die gemessene Laufzeit aufgetragen.
Bei der Impuls-Echo-Methode wird konsequent die Hälfte der gemessenen Zeit verwendet, da diese Werte die Laufzeiten für den Hin- und Rückweg beschreiben.

Die lineare Regression der drei Datenpaare ergeben die Steigung $m$ und den y-Achsenabschnitt $b$, aus denen im Weiteren die Nullstelle $x_0=-\sfrac{b}{m}$ berechnet werden kann.
\begin{table}
	\centering
	\begin{tabular}{c S S S}
	{Plot}&{Steigung}&{y-Abschnitt}&{Nullstelle}\\
	1&2797(11)&0.0002(4)&-0.7(13)e-07\\
	2&2745(6)&0.0002(2)&-7(7)e-08\\
	3&2776(181)&-0.004(6)&1.3(21)e-06
	\end{tabular}
	\caption{Regressionsparameter}
	\label{tab:params}
\end{table}
Das Diagramm lässt die Identifikation der Steigung als Schallgeschwindigkeit zu.

Im Idealfall durchqueren die Regressionsgeraden den Koordinatenursprung.
Es liegt zwischen der Sonde und der Oberfläche eine Anpassungsschicht, die systematische Fehler in der Laufzeitmessung hervorruft. \cite{skript}
Diese zusätzlich auftretende Zeit wird als positive Verschiebung der Kurve nach rechts erwartet und ist in Abbildung \ref{fig:geschwindigkeit} durch die Nullstellen erkennbar.
Die Nullstellen der Regressionen (vgl. Tabelle \ref{tab:params}) weisen für die Impuls-Echo-Methode starke Unsicherheit und einen negativen Nominalwert auf.
Die Nullstelle für die Durchschallungsmethode ist durch die hohe Unsicherheit ungeeignet.
%Die Nullstellen der Regression sind daher für die Betrachtung und Bezifferung der systematischen Fehler unbrauchbar.
Der durch die Anpassungsschicht hervorgerufene Fehler wird somit nicht berücksichtigt.

Der Mittelwert beträgt
\begin{equation}
	c_\text{Schall}=\SI{2770(60)}{\meter\per\second}
	\label{qu:geschwindigkeit}
\end{equation}


1. Bestimmen Sie die Schallgeschwindigkeit in Acryl mittels Impuls-Echo-Verfahren und mittels Durchschallungs-Verfahren
2. Ein Acrylblock mit definierten Fehlstellen unterschiedlicher Größe und Tiefe soll mit dem Impuls-Echo Verfahren untersucht werden (A-Scan und B-Scan)
3. Es soll an einem Augenmodell mit einem A-Scan die Abmessungen des Auges bestimmt werden
