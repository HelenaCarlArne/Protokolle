\section{Durchführung}
\label{sec:Durchfuehrung}
\subsection{Resonanzfrequenz}
Vor Messbeginn muss die fest eingestellte Resonanzfrequenz eines Schwingkreises gemessen und anschließend der Andere darauf abgestimmt werden. Dazu wird die in Abbildung ABC gezeigte Schaltung aufgebaut. Für die grobe Messung der Frequenz wird der X-Eingang zunächst nicht benötigt. Am Frequenzgenerator wird die Frequenz so lange variiert, bis auf dem Oszilloskop ein Strom mit maximaler Amplitude angezeigt wird. Die Feineinstellung wird mit X-Eingang vorgenommen, indem mit Lissajousfiguren mit im XY-Betrieb laufenden Oszilloskop, die Phasendifferenz zwischen Generator- und Schwingkreisstrom betrachtet wird.  Der Resonanzfall tritt ein, wenn die Phasenverschiebung null beträgt und eine Gerade auf dem Bildschirm erscheint.
Um den zweiten Schwingkreis auf dieselbe Frequenz zu eichen wird die Schaltung mit diesem Kreis erneut aufgebaut. Über eine variable Kapazität wird nach gleichem Verfahren der Resonanzfall eingestellt.

\subsection{Zeitabhängigkeit der Schwingungsenergie}

Für den ersten Versuchteil wird Schaltung 6 wie in Abbildung XY benötigt. Mit einem Rechteckimpuls wird ein Kreis zum Schwingen angeregt. Am ohmschen Widerstand fällt eine Spannung ab die durch das Oszilloskop dargestellt wird. Die Maxima jeder Schwebungsperiode werden bei verschiedenen Koppelkapazitäten abgezählt.

\subsection{Frequenzen der Fundamentalschwingungen in Abhängigkeit der Koppelkapazität}
Der zweite Versuchsteil basiert auf derselben Schaltung, die Schwingung wird nun jedoch mit einem Sinusgenerator angeregt.
Die Kapazität $C_\mathup{k}$ wird variiert. Für jede Kapazität wird die Lissajou-Figur betrachtet und die Frequenz so eingestellt, dass der Phasenunterschied $\varphi=0$ bzw. $\varphi=\sfrac{\pi}{2}$ beträgt.

\subsection{Frequenzabhängigkeit der Ströme}


