\newpage
\section{Auswertung}
\label{sec:Auswertung}

\subsection{Abstimmen der Systemparameter}
Der Kopplungskondensator wird überbrückt und somit effektiv nur der linke Schwingkreis angeregt.
Die Frequenz des Generators, bei welchem die Widerstandsspannung $U_\mathup{R,1}(t)$ maximal wird, ist die Resonanzfrequenz des Systems. 
Diese wurde mit 
\begin{equation}
	f_\text{Resonanz} = \SI{30.74}{\kilo\hertz}
\end{equation}
grob abgeschätzt.
Präzise ist der Phasenwinkel zwischen Erregerspannung $U(t)$ und der Widerstandsspannung $U_\mathup{R,2}(t)$ gleich Null, wenn die Erregerspannung die Resonanzfrequenz aufweist.
Beim Auftragen der Widerstandsspannung $U_\mathup{R,2}(t)$ gegen die Erregerspannung $U(t)$  wird eine Lissajou-Figur sichtbar.
Für den Phasenwinkel $\phi=0$ ist diese Figur nicht-elliptisch, gerade und verläuft im ersten und dritten Quadranten des Oszilloskopes.
Die  Resonanzfrequenz $f_\text{Resonanz}$ wird auf den Wert
\begin{equation}
	f_\text{Resonanz} = \SI{30.71}{\kilo\hertz}
\end{equation}präzisiert.
Der nach Gleichung \ref{eq:resonanz}!! gegebene Erwartungswert für die Resonanzfrequenz ist
\begin{equation}
	f_\text{Resonanz, Theorie} = \sqrt{\frac{1}{\mathup{LC}}-\frac{\mathup{R^2}}{4\mathup{L^2}}} = \SI{198.20\pm0.29}{\kilo\hertz}.
\end{equation}

Der zweite Schwingkreis wird durch den Kondensator an den ersten angeglichen.
\begin{table}
	\centering
	\begin{tabular}{ccc}
	\toprule
	{Kapazität $C_\mathup{1,2}$}&{Kapazität der Spule $C_\mathup{sp}$}&{Induktivität der Spule $L$}\\
	{$\si{\pico\farad}$}&{$\si{\pico\farad}$}&{$\si{\milli\henry}$}\\
	\midrule
		798\pm2 &37\pm1 &31.90\pm0.05\\
	\bottomrule
	\end{tabular}
	\caption{Die festen Parameter des gekoppelten Systems nach Abbildung \ref{fig:schaltung}!!.}
\end{table}
\subsection{Untersuchung der Schwebung}
\label{sec:Auswertung1}

Die Widerstandsspannung $U_\mathup{R,2}(t)$ ist ein Maß für den Strom $I_\mathup{2}(t)$, es besteht nach Ohmschen Gesetz der Zusammenhang $U(t)=\mathup{R}\cdot I(t)$ bei konstantem Widerstand R.
Bei dem Auftragen der Widerstandsspannung $U_\mathup{R,2}(t)$ gegen die Zeit wird gemäß der Theorie \ref{sec:theorie} eine Schwebung sichtbar.
(Helena) Um das Verhältnis der Schwebungs- und Schwingungsfrequenz zu bestimmen, wird innerhalb einer Schwebungsperiode die Anzahl der Schwingungsmaxima abgezählt. 
Das Verhältnis der Frequenzen $\alpha$ ist gegeben durch
\begin{equation}
	\alpha=\frac{n_\text{Osz.}}{n_\text{Schweb.}}.
\end{equation}
Daraus ergibt sich die Tabelle \ref{tab:verhaeltnis}
\begin{table}
	\centering
	\begin{tabular}{cc}
	\toprule
	{Kopplungskapazität}&{Frequenzverhältnis }\\
	{$C_\mathup{Kopplung}$}&{$\alpha$}\\
	{$\si{\nano\farad}$}&{1}\\
	\midrule
		9.99& 14\\
		8.00& 11\\
		6.47&  9\\
		5.02&  7\\
		4.00&  6\\
		3.00&  5\\
		2.03&  4\\
		1.01&  2\\
	\bottomrule
	\end{tabular}
	\caption{Die Frequenzverhältnisse $\alpha$ in Abhängigkeit von der Kopplungskapazität $C_\mathup{Kopplung}$}
\end{table}
\subsection{Untersuchung der Fundamentallösungen}
\label{sec:Auswertung2}
\begin{table}
	\centering
	\begin{tabular}{ccc}
	\toprule
	\multicolumn{2}{c}{Frequenz}&{Kopplungskapazität}\\
	{$f_\mathup{+}$}&{$f_\mathup{-}$}&{$C_\mathup{Kopplung}$}\\
	{$\si{\kilo\hertz}$}&{$\si{\kilo\hertz}$}&{$\si{\nano\farad}$}\\
	\midrule
		30.56	&32.85	&9.99\\
		30.57	&33.38	&8.00\\
		30.57	&33.98	&6.47\\
		30.57	&34.88	&5.02\\
		30.57	&35.86	&4.00\\
		30.57	&37.40	&3.00\\
		30.58	&40.12	&2.03\\
		30.58	&47.23	&1.01\\
	\bottomrule
	\end{tabular}
	\caption{Die Frequenzen der Fundamentallösungen in Abhängigkeit von der Kopplungskapazität $C_\mathup{Kopplung}$}
\end{table}

\subsection{Untersuchung der Stromkurve $I_2(t)$}
\label{sec:Auswertung2}
Der Generator in Schaltung \ref{fig:schaltung} durchläuft innerhalb von $\SI{20}{\milli\second}$ 
das Frequenzspektrum von \SI{10}{\kilo\hertz} bis \SI{80}{\kilo\hertz}. 
Wird die Widerstandsspannung $U_\mathup{R,2}(t)$ auf einem Oszilloskopen dargestellt, so zeigt sich der Auftrag von der Widerstandsspannung $U_\mathup{R,2}(t)$
gegen die Erregerfrequenz.
Die Zeitkoordinaten der Maxima sind in Tabelle \ref{tab:zeitkoord} dargestellt.

Mit den Anfangs- und Endkoordinaten
$\text{P}_\text{Anfang}=(\SI{-7}{\milli\second})|(\SI{10}{\kilo\hertz})$ und $\text{P}_text{Ende}=(\SI{14,3}{\milli\second})|(\SI{80}{\kilo\hertz})$ können die Zeitkoordinaten in Tabelle \ref{tab:zeitkoord} in Frequenzen umgerechnet werden.
Die lineare Regression aus diesen beiden Koordinaten gibt die Umrechnungsformel
\begin{equation}
	f = 3.7559t+36.2911
\end{equation}

\begin{table}[ht]
	\begin{tabular}{S[table-format=1.1]S[table-format=1.1]S[table-format=1.2]S[table-format=2.4]S[table-format=2.4]}
	\toprule
	\multicolumn{2}{c}{Zeitkoordinate}&{Kopplungskapazität}&\multicolumn{2}{c}{Frequenz}\\
	{$t_\mathup{+}$}&{$t_\mathup{-}$}&{$C_\mathup{Kopplung}$}&{$f_\mathup{+}$}&{$f_\mathup{-}$}\\
	{$\si{\milli\second}$}&{$\si{\milli\second}$}&{$\si{\nano\farad}$}&{$\si{\kilo\hertz}$}&{$\si{\kilo\hertz}$}\\
	\midrule
		0.3& 	0.9&	9.99&	37.4178& 	39.6714\\
		0.3&	1.1&	8.00&	37.4178&	40.4225\\
		0.4&	1.2&	6.47&	37.4934&	40.7981\\
		0.3&	1.5&	5.02&	37.4178&	41.9249\\
		0.2&	1.8&	4.00&	37.0423&	43.0516\\
		0.2&	2.2&	3.00&	37.0423&	44.554\\
		0.2&	3.0&	2.03&	37.0423&	47.5587\\
		0.2&	5.1&	1.01&	37.0423&	55.446\\
	\bottomrule
	\end{tabular}
	\caption{Die Zeitkoordinaten und die Frequenzen der Strommaxima in Abhängigkeit von der Kopplungskapazität $C_\mathup{Kopplung}$.}
	\label{tab:zeitkoord}
\end{table}