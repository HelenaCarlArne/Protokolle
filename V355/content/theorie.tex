\section{Zielsetzung}
In diesem Versuch werden gekoppelte elektrische Schwingkreise betrachtet. 
Ziel ist es, das Verhalten der Energie bzw. des Energieübergangs zwischen den einzelnen Systemen insbesondere unter dem Aspekt der Zeitabhängigkeit zu betrachten. 
Desweiteren soll das Verhalten des Gesamtsystems bei erzwungenen Schwingungen, also äußerer periodischer Anregung, untersucht werden.
\section{Theorie}
\label{sec:Theorie}
Sind zwei schwingfähige Systeme miteinander gekoppelt, so beeinflussen sie sich gegenseitig.
Wird eines der Systeme zum Schwingen angeregt pendelt die zugeführte Energie zwischen beiden Systemen; die Energieerhaltung gilt.
\subsection{Kapazitiv gekoppelte Schwingkreise}
Die zwei dargestellten eigenständigen Schwingkreise sind über den Kondensator der Kapazität $C_\mathup{k}$ miteinander verknüpft.
 Mit Knoten- und Maschenregel lassen sich folgende Gesetzmäßigkeiten herleiten:
\begin{equation}
	I_\mathup{k}=I_1-I_2
	\label{eq:I_k}
\end{equation}
\begin{equation}
	U_{1\mathup{C}}+U_{1\mathup{L}+U_{\mathup{k}}}=0
	\label{eq:U_1}
\end{equation}
\begin{equation}
	U_{2\mathup{C}}+U_{2\mathup{L}+U_{\mathup{k}}}=0
	\label{eq:U_2}
\end{equation}
Desweiteren gelten die Beziehungen
\begin{equation}
	U_\mathup{C}=\frac{1}{C}\int{I\mathup{d}t}
	\label{eq:U_C}
\end{equation}
und
\begin{equation}
	U_\mathup{L}=L\dot{I}.
	\label{eq:U_L}
\end{equation}
Einsetzen von \eqref{eq:U_C} und \eqref{eq:U_L}, sowie ableiten nach der Zeit ergibt ein Differentialgleichungssystem. 
Werden die Gleichungen entkoppelt, lassen sie sich unabhängig voneinander lösen.
Lösung der ersten Gleichung, entstanden durch Addition der DGL-System-Gleichungen,
\begin{equation}
	L(\ddot{I_1}+\ddot{I_2})+\frac{1}{C}({I_1}+{I_2})=0
	\label{eq:I_1+I_2}
\end{equation}
ist eine harmonische Schwingung der Form
\begin{equation}
	({I_1}+{I_2})(t)=({I_{10}}+{I_{20}})\cos(\frac{t}{\sqrt{LC}})
\end{equation}
mit der Schwingungsfrequenz $\nu⁺=\frac{1}{2\pi\sqrt{LC}}$.
Diese Frequenz entspricht derer eines einzelnen Oszillators mit den Bauteilen $L$ und $C$.
 Die Amplitude $({I_10}+{I_20})$ bleibt konstant.
Die zweite Gleichung, enstanden durch Subtraktion,
\begin{equation}
	L(\ddot{I_1}-\ddot{I_2})+\bigl(\frac{1}{C}+\frac{1}{C_\mathup{k}}\bigr)({I_1}-{I_2})=0
	\label{eq:I_1-I_2}
\end{equation}
wird gelöst durch
\begin{equation}
	({I_1}-{I_2})(t)=({I_{10}}-{I_{20}})\cos\biggl(t{\biggl[L\biggl({\frac{1}{C}+\frac{1}{C_\mathup{k}}}\biggr)⁻¹\biggr]^{-\frac{1}{2}}}\biggr).
\end{equation}
Die Schwingungsfrequenz $\nu⁻=\frac{1}{2\pi\sqrt{L\bigl({\frac{1}{C}+\frac{1}{C_\mathup{k}}}\bigr)⁻¹}}$ ist größer als $\nu+$.
Erneute Addition und Subtraktion der voneinander unabhägingen Gleichungen ergibt
\begin{equation}
	I_1(t)=\frac{1}{2}({I_{10}}+{I_{20}})\cos(2\pi\nu⁺t)+\frac{1}{2}({I_{10}}-{I_{20}})\cos(2\pi\nu⁻t)
	\label{eq:I_1_ur}
\end{equation}
und
\begin{equation}
	I_2(t)=\frac{1}{2}({I_{10}}+{I_{20}})\cos(2\pi\nu⁺t)-\frac{1}{2}({I_{10}}-{I_{20}})\cos(2\pi\nu⁻t).
	\label{eq:I_2_ur}
\end{equation}
Im folgenden werden zwei Spezialfälle betrachtet. 
Schwingen die Systeme mit gleicher Amplitude $I_{10}=I_{20}$ und gleicher Phase so verschwindet die Differenzschwingung \eqref{eq:lsg_I_1-I_2}, sodass die Oszillatoren jeweils mit $\nu⁺$ eines einzelnen Oszillators und gleicher Phase schwingen.
 Am Kondensator $C_\mathup{k}$ liegt zu keinem Zeitpunkt eine Spannung an, da die Ströme sich gegenseitig kompensieren.
Schwingen die Systeme mit entgegengesetzter Amplitude  $I_{10}=-I_{20}$ und Phase verschwindet die Summenschwingung \eqref{eq:lsg_I_1+I_2}. Die Oszillatoren schwingen gegenphasig mit der Frequenz $\nu⁻$.
Diese beiden Schwingungsmoden werden als Fundamentalschwingungen des Systems bezeichnet.

Wird zur Zeit $t=0$ nur ein Oszillator ausgelenkt, d.h. $I_{10}\neq0$,$I_{20}=0$, vereinfachen sich \eqref{{eq:I_1_ur}} und \eqref{{eq:I_2_ur}}. 
Mit einigen Umformungen und nutzen von Additionstheoremen ergeben sich
\begin{equation}
	I_1(t)=I_{10}\cos(\frac{1}{2}(\omega⁺+\omega⁻)t)\cos(\frac{1}{2}(\omega⁺-\omega⁻)t)
\end{equation}
und
\begin{equation}
	I_2(t)=I_{10}\sin(\frac{1}{2}(\omega⁺+\omega⁻)t)\sin(\frac{1}{2}(\omega⁺-\omega⁻)t).
\end{equation}
Gemäß der Annahme $\nu⁺\approx\nu⁻$ und $C_\mathup{k}>>C$ ist $\frac{1}{2}(\omega⁺-\omega⁻)\approx\omega⁺$ und $\omega⁻-\omega⁺<<\omega⁺$.
Die Oszillatoren schwingen mit der Frequenz $\frac{1}{2}(\nu⁺+\nu)$, welche ungefähr der Frequenz eines Einzeloszillators entspricht. 
Die Amplituden verändern sich Periodisch mit der Schwebungsfrequenz $\nu⁻-\nu⁺$. 
Die Verhältnisse haben sich nach der Zeit $T*\frac{1}{2}({\omega⁻-\omega⁺})=\frac{\pi}{2}$ umgekehrt, die Energie pendelt periodisch mit der Schwebungsfrequenz zwischen beiden Systemen.

\subsection{Erzwungene Schwingungen}

Erzwungene Schwingungen werden durch eine von außen angelegte Sinusspannung erzeugt. 
Die Resonanzkurve zweier gekoppelter Schwingkreise unterscheidet sich von der Kurve eines einzelnen Oszillators mit periodischer Anregung durch die Anzahl der auftretenden Maxima. 
Für sehr geringe bzw. sehr große Frequenzen nähert sich der Wert des Stromes der Null an.
 An den Stellen der in KAPITEL 1.2 ODER SO genannten Fundamentalfrequenzen treten Resonanzphänomene auf.
 Die Werte des Stromes an diesen Frequenzen entscheiden sich nur unwesentlich, sodass die Leitwerte $|L(\omega⁺)|=|L(\omega⁻)|\approx\frac{1}{2R}$  angenommen werden darf. 
Betrachtet man die Frequenz $\nu⁺$, welche unabhängig von der Koppelkapazität $C_\mathup{k}$ ist, können beide Schwingkreise als ein Kreis mit $L_\mathup{ges}=2L$ und $C_\mathup{ges}=\frac{1}{2}C$ angesehen werden. 
Durch die Koppelleitung fließt kein Strom.
 Bei der Frequenz $\nu⁻$ ist $C_\mathup{ges}=\frac{C C_\mathup{k}}{C_\mathup{k}+2C}$ - durch die Koppelleitung fließt nun der maximale Strom.
