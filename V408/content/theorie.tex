\section{Ziel}
\label{sec:Ziel}

Versuchsziel ist es Brennweiten verschiedener Linsen zunächst mit Hilfe der Linsengleichung und anschließend mit der Methode nach \textsc{Bessel} zu berechnen. 
Außerdem werden Brennweite und Lage der Hauptebenen eines Linsensystems nach \textsc{Abbe} bestimmt.

\section{Theorie}
\label{sec:Theorie}

Nach der geometrischen Optik, die immer gilt wenn alle Abmessungen einer Apparatur groß gegenüber der Wellenlänge sind, breitet sich Licht in Form von Strahlen aus. 
Ändert sich das Medium und damit die Dichte, in dem sich der Strahl ausbreitet, wird er nach dem \textsc{Snellius}'schen Brechungsgesetz gebrochen. 
Meist genutztes optisches Element ist die Linse, deren Material die Dichte von Luft überschreitet. 
Je nach Form der Linsen weisen diese verschiedene Eigenschaften auf. \emph{Sammellinsen} bündeln parallel eintreffende Lichtstrahlen im Brennpunkt und lassen ein reelles, sich hinter der Linse befindliches, Bild entstehen. Dabei sind der Abstand zwischen Gegenstand und Linse -- Gegenstandsweite $g$ -- und Abstand zwischen Linse und Schirm -- Bildweite $b$ -- positive Größen. 
\emph{Zerstreuungslinsen} mit negativer Gegenstands-, Bild- und Brennweite erschaffen ein virtuelles Bild, welches vor der Linse liegt. Zeichnerisch wird der Strahlengang durch Parallel-, Brennpunkts- und Mittelpunktsstrahl dargestellt.
Durchquert ein Lichtstrahl eine dünne Linse wird er an der Mittelebene gebrochen.
Die Brechkraft $D$, der Kehrwert der Brennweite mit der Einheit $dpt=\frac{1}{m}$, kann für dünne Linsen berechnet werden mit der \emph{Linsengleichung}

\begin{equation}
	D=\frac{1}{f}=\frac{1}{b}+\frac{1}{g}.
\end{equation}

Bei einer dicken Linse passiert die Brechung an zwei Hauptebenen $H$ und $H'$, da der Strahl hier einen weiteren Weg im Medium zurücklegen muss, als es bei dünnen Linsen der Fall ist.
 Relativ zu den Hauptebenen werden hier $b$,$b'$ und $g$, $g'$ bestimmt. 
Über das \emph{Abbildungsgesetz} 

\begin{equation}
	V=\frac{B}{G}=\frac{b}{g}
\end{equation}

mit Bildgröße $B$ und Gegenstandsgröße $G$ kann die Vergrößerung $V$ des Bildes bestimmt werden.
Natürlich treten bei der Brechung \emph{Abbildungsfehler} auf.
Zum Beispiel gilt die Nährung der geometrischen Optik nur für achsennahe Strahlen -- achsenferne Strahlen befinden sich von der optischen Achse eines Systems weiter entfernt. 
Deswegen werden sie stärker gebrochen und besizten einen näher an der Linse liegenden Brennpunkt. Dieses Phänomen wird \emph{sphärische Aberration} genannt. 
Neben dieser tritt auch \emph{chromatische Aberration} häufig auf. 
Wird nicht-monochromatisches Licht durch eine Linse geschickt werden geringe Wellenlägen stärker gebrochen als größere Wellenlängen. 
Daher liegt bespielsweise der Brennpunkt roten Lichtes weiter von der Linse entfernt als der Brennpunkt blauen Lichtes.

\subsection{Brennweitenbestimmung mit der Methode nach \textsc{Bessel}}

Ist der Abstand $e=g+b$ zwischen Gegenstand und Schirm konstant lassen sich zwei Linsenpositionen finden, die ein scharfes Bild erzeugen. Dabei können Gegenstands- und Bildweite symmetrisch vertauscht werden. 
Es gilt

\begin{equation}
	b_1=g2 \text{und} b_2=g_1.
\end{equation}

Ist außerdem der Abstand $d=g-b$ zwischen beiden Linsenpostitionen bekannt kann die Brennweite nach

\begin{equation}
	f=\frac{e²-d²}{4e}
\end{equation}

berechnet werden.

\subsection{Brennweitenbestimmung mit der Methode nach \textsc{Abbe}}

Über die Methode von \textsc{Abbe} können dicke Linsen oder auch Linsensysteme auf die Lage der Hauptachsen $H$, $H'$ und Brennweite $f$ untersucht werden. 
Für die Abstände gelten die Beziehungen

\begin{equation}
	g'=g+h=f\cdot\left(1+\frac{1}{V}\right)+h,\qquad
	b'=b+h'=f\cdot\left(1+V\right)+h'.
\end{equation}
