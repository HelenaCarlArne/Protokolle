\section{Durchführung}
\label{sec:Durchfuehrung}
\subsection{Brennweiten verschiedener Linsen}

In diesem Versuchsteil solldie Linsengleichung verifiziert werden. Eine Linse bekannter Brennweite wird ausgemessen und die berechnete Brennweite mit der Herstellerangabe verglichen.
Dazu wird ein Gegenstand "\emph{Pearl L}" auf einer optischen Bank vor einer Halogenlampe platziert.
 Zwischen Gegenstand und verschiebbarem Schirm wird eine Linse mit $f_1=100\,dpt$ gestellt.
 Anschließend wird die Bildweite $b_1$ für zehn verschiendene Gegenstandsweiten $g_1$ gemessen, in dem für jedes $g_1$ der Schirmabstand so variiert wird, dass der Gegenstand scharf abgebildet wird.
Als nächstes wird eine Linse mit $f_2=50\,dpt$ genutzt und der vorhergehend beschriebene Versuchsablauf wiederholt.

\subsection{Brennweitenbestimmung nach \textsc{Bessel}}

Nun wird bei der Linse mit $f=100\,dpt$ der Abstand $e=b+g$ zwischen Gegenstand und Schirm fest gehalten. 
Für zehn verschiedene $g_2$ werden zwei Linsenpositionen gesucht, die ein scharfes Bild erzeugen. 

\subsection{Brennweitenbestimmung nach \textsc{Abbe}}
