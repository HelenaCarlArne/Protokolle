\section{Auswertung} % (fold)
\label{sec:swrtng}
\subsection{Bestimmung der Winkelrichtgröße $D$}
\begin{table}
	\centering
	\def\arraystretch{1.3} 
	\begin{tabular}{ccc}
	\toprule
	\multicolumn{2}{c}{Winkel} & {Kraft} \\
	{$\phi$} & {$\phi/\:\text{rad}$} & {$F/\:\si{\newton}$} \\
	\midrule
 45$\text{°}$ & $\frac{\pi}{4}$   & 0.22 \\ 
 90$\text{°}$ & $\frac{\pi}{2}$   & 0.40 \\
120$\text{°}$ & $\frac{2\pi}{3}$  & 0.52 \\
135$\text{°}$ & $\frac{3\pi}{4}$  & 0.60 \\
180$\text{°}$ & $\pi$             & 0.78 \\
225$\text{°}$ & $\frac{5\pi}{4}$  & 0.92 \\
240$\text{°}$ & $\frac{4\pi}{3}$  & 0.90 \\
270$\text{°}$ & $\frac{3\pi}{2}$  & 1.08 \\
315$\text{°}$ & $\frac{7\pi}{4}$  & 1.26 \\
360$\text{°}$ & $2\pi$            & 1.42 \\
	\bottomrule
	\end{tabular}
	\label{tab:M1 I_D}
	\caption{Messung zur Bestimmung des Eigenträgheitsmomentes der Drillachse.}
\end{table}

Mit den Messwerten aus Tabelle \ref{tab:M1 I_D} und $r=0.09965\si{\meter}$ lässt sich das Mittel der Winkelrichtgröße $D$ mit $D_i = \frac{|\vec{F}|\cdot|\vec{r}|}{\phi}$ bestimmen. Die Unsicherheit ist die Standardabweichung des Mittelwertes $\sigma_D$ mit
\begin{align}
	\sigma_D &= \sqrt{\sigma_D^2} := \sqrt{\frac{1}{n-1} \sum_{i=1}^n{(D_i-\bar{D})^2}}
	\intertext{mit}
	\bar{D} &= \frac{1}{n} \sum_{i=1}^n{D_i}.
\end{align}
Es ergibt sich $D=(0.0757\pm0.0019)\si{\newton\meter}$
\subsection{Bestimmung des Eigenträgheitsmomentes $I_D$}
\begin{table}[hp]
	\centering
	\sisetup{table-format=2.3}
	\begin{tabular}{S[table-format=2.4] S[table-format=2.2] S[table-format=1.3] S[table-format=2.2] S[table-format=1.3] S[table-format=3.2] S[table-format=3.2]}
	\toprule
	{Abstand}&\multicolumn{4}{c}{Schwingungsdauer} & \multicolumn{2}{c}{Masse} \\
	{$a/\:\si{\centi\meter}$} & {$2T_{1}/\:\si{\second}$} & {$T_{1}/\:\si{\second}$} & {$2T_{2}/\:\si{\second}$} & {$T_{2}/\:\si{\second}$} & {$m_{1}/\:\si{\gram}$} & {$m_{2}/\:\si{\gram}$}\\
	\midrule
 6.4925 &  5.90 & 2.950 &  5.90 & 2.950 & 221.74 & 221.73 \\
 8.4925 &  6.55 & 3.275 &  6.46 & 3.230 & 221.74 & 221.73 \\
10.9925 &  7.29 & 3.645 &  7.21 & 3.605 & 221.75 & 221.73 \\
13.7925 &  8.66 & 4.330 &  8.63 & 4.315 & 221.76 & 221.75 \\
16.6925 & 10.07 & 5.035 & 10.10 & 5.050 & 221.75 & 221.74 \\
19.0925 & 11.13 & 5.565 & 11.15 & 5.575 & 221.75 & 221.74 \\
20.4925 & 11.92 & 5.960 & 11.76 & 5.880 & 221.75 & 221.73 \\
22.4925 & 13.03 & 6.515 & 13.00 & 6.500 & 221.76 & 221.74 \\
24.5925 & 14.10 & 7.050 & 13.96 & 6.980 & 221.75 & 221.75 \\
29.5925 & 16.67 & 8.335 & 16.73 & 8.365 & 221.76 & 221.74 \\
	\bottomrule
	\end{tabular}
	\caption{Messung zur Bestimmung des Eigenträgheitsmomentes der Drillachse}\label{tab:M2 I_D}
\end{table}
Zur Bestimmung des Eigenträgheitsmomentes $I_D$ wird die verwendete Stange als masselos angenommen, wodurch ihr Anteil am Trägheitsmoment vernachlässigbar ist. 
Das gemessene Trägheitsmoment setzt sich aufgrund der Linearität des Trägheitsmomentes aus den Trägheitsmomenten der Massestücke $m_1$ und $m_2$ als Punktmassen, sowie dem Eigenträgheitsmoment $I_D$ zusammen,
\begin{equation}
	I= I_D+I_{m_1}+I_{m_2}.
\end{equation}
Nach Einsetzen in Gleichung \eqref{eq:Winkelricht} wird der lineare Zusammenhang von $T^2$ und $a^2$ ersichtlich.
Es gilt
\begin{align*}
	 D &= 4\mathup{\pi^{2}}\cdot\frac{I}{T^2}\\
	   &= 4\mathup{\pi^{2}}\cdot\frac{I_D+I_{m_1}+I_{m_2}}{T^2}\\
	   &= 4\mathup{\pi^{2}}\cdot\frac{I_D+a^{2}(m_1+m_2)}{T^2}\\
	T^2&= 4\mathup{\pi^{2}}\frac{I_D}{D}+4\mathup{\pi^{2}}\frac{a^{2}(m_1+m_2)}{D}\\
\end{align*}
\begin{equation}
	T^2= \underbrace{4\mathup{\pi^{2}}\frac{(m_1+m_2)}{D}}_{m_{\text{Reg}}}\cdot a^{2}+\underbrace{4\mathup{\pi^{2}}\frac{I_D}{D}}_{b_{\text{Reg}}}
\end{equation}
Zur Bestimmung des Eigenträgheitsmoments $I_D$ werden die gemittelten Schwingperioden-Quadrate $\bar{T}^2$ gegen das Abstandsquadrat $a^2$ aufgetragen. Aus der Regression mittels der Formeln
\begin{align}
	\Delta &= N \sum({a^2}^2) - {\sum{a^2}}^2\\
    m_{\text{Reg}} &= \frac{N\sum{a^2\cdot T^2} - \sum{a^2} * \sum{T^2}}{\Delta}\\
    %B = (np.sum(x**2) * np.sum(y) - np.sum(x) * np.sum(x * y)) / Delta
    %sigma_y = np.sqrt(np.sum((y - A * x - B)**2) / (N - 2))
    %A_error = sigma_y * np.sqrt(N / Delta)
    %B_error = sigma_y * np.sqrt(np.sum(x**2) / Delta)
\end{align}
% section auswertung (end)