\section{Theorie} 
\label{sec:ziel}

In diesem Versuch sollen die Trägheitsmomente verschiedener Körper gemessen und anschließend mit den theoretisch errechneten Werten verglichen werden. Dafür werden die Winkelrichtgröße $D$ und das Trägheitsmoment der Drillachse $I_{\mathup{D}}$ benötigt.

\label{sec:theorie}

Bei Drehbewegungen sind das Drehmoment $M$, das Trägheitsmoment $I$ und die Winkelbeschleunigung {\Omega} maßgebliche Größen. Das Drehmoment 
\begin{equation}
\label{1}
M=r\times F=D\phi
\end{equation}
ist abhängig von der Kraft $F$, welche im Abstand $r$ von der Drehachse angreift. Es kann aber auch über die Winkelrichtgröße $D$ in Analogie zur Federkonstanten $k$ und der Auslenkung um den Winkel $\phi$ beschrieben werden. [Der Drehimpuls als Erhaltungsgröße kann nur durch ein äußeres Drehmoment M geändert werden.# eventuell weglassen; eigentlich unwichtig]

Die Masse ist wichtiges Element der Translation. 
Ihr Analogon bei Rotationsbewegungen ist das Trägheitsmoment. Dies ist definiert als
\begin{equation}
	\label{2}

I=\int_{m} {r_{i}^{2}}\mathup{d}m 
\end{equation}

für einen festen Körper mit homogener Massenverteilung, der sich um eine feste Achse, verlaufend durch den Körperschwerpunkt, mit einer konstanten Winkelgeschwindigkeit $\omega$ dreht. Falls die Drehachse nicht durch den Schwerpunkt verläuft kann das Trägheitsmoment mit dem Satz von Steiner 
\begin{equation}
 \label{3}
I= I_{\mathup{S}}+m\mathup{a^2} 
\end{equation}

berechnet werden. Das Gesamtträgheitsmoment $I$ setzt sich zusammen aus dem Trägheitsmoment des Körperschwerpunkts I_{\mathup{S} und dem Produkt aus der Gesamtmasse $m$ des Körpers und dem Quadrat des senkrechten Abstands der eigentlichen Drehachse die um $a$ parallel zur Schwerpunktsachse verschoben ist.

Mechanische Drehschwingungen - der Gegensatz ist die Translation - führen harmonische Schwingungen mit der Schwingungsdauer

\begin{equation}
\label{4}
T=2*\mathup{\pi} \sqrt{\frac{I}{D}}
\end{equation}

für kleine Auslenkungswinkel $\phi$ aus. $D$ berechnet sich mit Formel (1) und (4) zu

\begin{equation}
\label{5}
D= 4\mathup{\pi^{2}}*I*\frac{1}{T} =\frac{M}{\phi} =\frac{F*r}{\phi}
\end{equation}


,wobei die Auslenkung senkrecht ... [ #fehlt: keine Formulieridee].

% section theorie (end)
