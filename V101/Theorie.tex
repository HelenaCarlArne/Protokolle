
\section{Ziel}
\label{sec:ziel}
Es werden die Trägheitsmomente verschiedener Körper gemessen und anschließend mit den theoretisch errechneten Werten verglichen. Hierzu werden die Winkelrichtgröße $D$ und das Trägheitsmoment der Drillachse $I_{\mathup{D}}$ bestimmt.

\section{Theorie} 
\label{sec:theorie}
Translation und Rotation verbinden Analogien.
Bei Rotationen sind das Drehmoment $\vec{M}$, das Trägheitsmoment $I$ und die Winkelbeschleunigung $\dot{\vec{\omega}}$ maßgebliche Größen. Das Drehmoment $\vec{M}$ mit $\vec{M}=\vec{r}\times \vec{F}$ ist abhängig von der Kraft $\vec{F}$, 
welche  im Abstand $|\vec{r}|$ von der Drehachse angreift. 
Ausgedrückt über die Winkelrichtgröße $D$ und die Auslenkung des Winkels $\phi$ ist der Betrag des Drehmomentes ebenfalls
\begin{equation}
	\label{eq:Moment_Winkelricht}
	|\vec{M}|=D\phi.
\end{equation}

Das Trägheitsmoment $I$ ist, analog zur Masse $m$ in Translationen, der Widerstand eines Drehmoments $\vec{M}$.
Es gilt für Drehachsen durch den Masseschwerpunkt $S$
\begin{align}
	\label{eq:Tragheit}
	I_{\mathup{S}} &= \sum_{i=1}^n m_{i}\cdot r_{i}^{2}
\intertext{für diskrete Massestücke $m_{i}$ im Abstand $r_i$ von der Rotationsachse und}
	I_{\mathup{S}} &=\int_{m_{\text{K}}} {r_{i}^{2}}\mathup{d}m\\ 
	  			   &=\int_{V_{\text{K}}} \rho(\vec{r})\cdot{r_{\perp}^{2}}\mathup{d}V
\end{align}
für kontinuierliche Masseverteilungen mit Massenverteilung $\rho(\vec{r})$.
Ist die Drehachse um $a$ parallel zur Achse durch den Schwerpunkt verschoben, so kann das Trägheitsmoment $I_{a}$ unter Zuhilfenahme des Satzes von Steiner berechnet werden, 
\begin{equation}
	\label{eq:steiner}
	I_{a}= I_{\mathup{S}}+m_{\text{K}}\cdot a^2. 
\end{equation}
%Das Trägheitsmoment $I_{\mathup{a}}$ setzt sich zusammen aus dem Trägheitsmoment des Körperschwerpunkts $I_{\mathup{S}}$ und dem Produkt aus der Gesamtmasse $m$ des Körpers und dem Quadrat des senkrechten Abstands der eigentlichen Drehachse die um $a$ parallel zur Schwerpunktsachse verschoben ist.

Mechanische Drehschwingungen führen harmonische Schwingungen mit der Schwingungsdauer
\begin{equation}
	\label{eq:Schwingperiode}
	T=2\mathup{\pi} \sqrt{\frac{I}{D}}
\end{equation}
für kleine Auslenkungswinkel $\phi$ aus. 
Die Winkelrichtgröße $D$ berechnet sich bei $\vec{F}\perp \vec{r}$ mit Formel \eqref{eq:Moment_Winkelricht} und \eqref{eq:Schwingperiode} zu
\begin{align}
	\label{eq:Winkelricht}
	D&=\frac{F\cdot r}{\phi}\\
	 &= 4\mathup{\pi^{2}}\cdot\frac{I}{T^2}. 
\end{align}

% section theorie (end)
