\section{Diskussion} % (fold)
\label{sec:diskussion}

\subsection{Winkelrichtgröße $D$ und Eigenträgheitsmoment $I_\text{D}$ der Drillachse}
\label{Systembestimmend}
Mit einer relativen Ungenauigkeit von $2.51\%$ kann die Winkelrichtgröße über die angewandte Methode für weitere Berechnungen ausreichend genau bestimmt werden. 
Die größte Fehlerquelle liegt im Ablesen der Auslenkungswinkel.

Das Trägheitsmoment $I$ in Messung 2 (vgl. \ref{subsec:eigentragheit}) setzt sich aus dem Eigenträgheitsmoment $I_\text{D}$
der Drillachse und den Trägheitsmomenten $I_\mathup{m_i}$ der als punktförmig angenommenen Massestücke zusammen. 
%$I_\text{m_i}$
Der verwendete Stab wird als masselos angesehen und daraus resultierend das Trägheitsmoment $I{_\text{S}}$ vernachlässigt. 
Wird die Annahme eines masselosen Stabes nicht getroffen, so ergibt sich nach Abzug von hier unbekanntem $I_{\text{Stab}}$ ein deutlich kleinerer Wert für $I_{\text{D}}$.
Dies zeigt, dass, anders als der errechnete Wert \eqref{wert:eigentragheit}, der tatsächliche Wert $I_{\text{D}}$ wahrscheinlich in einer Größenordnung liegt, sodass er bei Messungen von Trägheitsmomenten berücksichtigt werden kann. Berücksichtigt man den errechneten Wert für $I_{\text{D}}$ in den Rechnungen ab Kapitel \ref{sub:traegheitsmoment_eines_zylinders}, so werden die Ergebnisse negativ und damit unbrauchbar. 

Dies widerlegt die Korrektheit vom errechneten Eigenträgheitsmoment $I_{\text{D}}$; der Wert weicht wahrscheinlich stark von der Realität ab.

\subsection{Bestimmung der Trägheitsmomente zweier Körper}
Für die Berechnung des Trägheitsmomentes $I_{\text{Z}}$ wird angenommen, dass es sich um einen Hohlzylinder aus Styropor handelt. 
Es ist davon zusätzlich von der zweiten Annahme in \ref{sub:traegheitsmoment_eines_zylinders} auszugehen, dass der theoretische Wert nicht korrekt ist, da der Zylinder nicht ausschließlich aus Styropor bestehen könnte. 
Befindet sich zur Stabilisierung ein Lack auf dem Zylinder oder ein Gerüst im Zylindermantel, ist bei gleicher Gesamtmasse ein größerer Teil der Masse weiter von der Drehachse entfernt, als hier angenommen wird. Dies liefert ein größeres Trägheitsmoment und nähert sich dadurch dem gemessenen Wert an. Dies erklärt die Abweichung von $81.84\%$.
Das gemessene Trägheitsmoment $I{_\text{Z}}$ des Zylinders hat eine relative Unsicherheit von $2.7\%$, kann also hinreichend genau gemessen werden.

Das gemessene Trägheitsmoment $I{_\text{K}}$ der Kugel mit einer relativen Unsicherheit von $2.45\%$ liegt in der Größenordnung des theoretischen Wertes, welcher eine Unsicherheit von $0.65\%$ aufweist. 
Der theoretische Wert weicht um $27.05\%$ von der Messung ab.

\subsection{Bestimmung der Trägheitsmomente der Modellpuppe}
\subsubsection{Gemessene Trägheitsmomente}
Das über die gemessene Periodendauer berechnete Gesamtträgheitsmoment der ersten Position ist $36.52\%$ größer als das der Zweiten, da in Position 1 sich mehr Masse weiter von der Drehachse entfernt befindet.
Die relativen Unsicherheiten von $2.21\%$ (Position 1) und $2.44\%$ (Position 2) liegen in einem vertretbaren Rahmen.

Größte Fehlerquelle ist die manuelle und einmalige Zeitmessung. 
Bei der Fehlerrechnung wird die Reaktionszeit des Laboranten nicht berücksichtigt. 
Der Fehler könnte minimiert werden, wenn anstelle von einer Schwingungsperiode grundsätzlich die Zeit für mehrere Schwingungen gemessen und anschließend durch deren Anzahl dividiert würde.
Zusätzlich ist, insbesondere bei schnellen Schwingungen, ein genaues Bestimmen der Periodendauer schwierig. 
\subsubsection{Theoretische Trägheitsmomente}
Die berechneten Trägheitsmomente von Kopf und Rumpf der Puppe weisen eine relative Unsicherheit von $5.66\%$ bzw. $11.67\%$ auf. 
Die Trägheitsmomente der Arme unterscheiden sich untereinander in Position 1 um $21.05\%$ und in Position 2 um $30.23\%$. 
Die Trägheitsmomente der Beine mit relativer Unsicherheit von $10.33\%$ (links) und $12.50\%$ unterscheiden sich voneinander um $6,67$\%. 
Dies ist erklärbar mit den nicht genau bestimmbaren Abmessungen der Puppe aufgrund der unregelmäßigen Form.

Wegen der zu groben Näherung des Puppenkörpers durch einzelne Zylinder und Kugeln lassen sich nur die Verhältnisse der Momente, nicht aber diese selbst, vergleichen.
Aus den Quotienten \eqref{eq:Quo1} der theoretischen Werte $\alpha=1.73\pm0.08$ und dem Quotienten \eqref{eq:Quo2} der gemessenen Werte $\beta=1.58$ folgt eine  Übereinstimmung von $109.96\pm5.13\%$. 
Die Abweichung resultiert aus der Tatsache, dass die Position von Armen und Beinen sich während der Schwingung verändert, sodass sie sich am Ende der Messung nicht senkrecht zur Drehachse befinden, sondern in einem unbestimmten Winkel geneigt sind.

\subsection{Zusammenfassung}
Die Annahmen zur Bestimmung System-bestimmender Größen (vgl. \ref{Systembestimmend}) und die Näherungen der Körper, im Einzelnen Geometrie und Masseverteilung, sind sehr grob. Daher eignet sich das Ermitteln des Trägheitsmomentes durch Messung der Schwingperioden nur bei großen Trägheitsmomente mit leicht bestimmbaren Schwingungsperioden. 

Unter Beachtung der Fehlerquellen stimmen theoretische und gemessene Werte in vertretbarem Maß überein, sodass die angewandten Formeln sowie der Satz von Steiner als verifiziert betrachtet werden können.

\end{document}
\cite{skript}
\cite{skript}
\cite{sample}
% section diskussion (end)
