
\begin{table}
	\centering
	\sisetup{table-format=2.3}
	\begin{tabular}{S[table-format=1.2] S[table-format=1.3] S[table-format=1.3] S[table-format=3.1]}
	\toprule
	%Hier ist iiirgendwo ein Fehler & 
	\multicolumn{2}{c}{Schwingungsdauer} & {Abmessungen} & {Masse} \\
	{$5T/\:\si{\second}$} & {$T/\:\si{\second}$} & {$\text{Durchmesser}\:D/\:\si{\centi\meter}$} & {$M/\:\si{\gram}$}\\
	\midrule
8.60 & 1.720 & 13.745 &	812.7 \\
8.56 & 1.721 & 13.730 &	812.7 \\
8.61 & 1.722 & 13.720 &	812.7 \\
8.58 & 1.716 & 13.745 &	812.7 \\
8.56 & 1.721 & 13.750 &	812.7 \\
8.61 & 1.722 & 13.740 &	812.7 \\
8.55 & 1.710 & 13.750 &	812.7 \\
8.61 & 1.722 & 13.710 &	812.7 \\
8.60 & 1.720 & 13.745 &	812.7 \\
8.61 & 1.722 & 13.730 &	812.7 \\
	\bottomrule
	\end{tabular}
	\caption{Messung zur Bestimmung des Eigenträgheitsmomentes einer Kugel}
	\label{tab:M4 I_K}
\end{table}

