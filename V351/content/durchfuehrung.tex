\section{Durchführung}
\label{sec:Durchfuehrung}
\subsection{\textsc{Fourier}-Analyse}
Es werden periodische elektrische Schwingungen in ihre \textsc{Fourier}-Komponenten zerlegt und die gemessenen \textsc{Fourier}-Koeffizienten mit den nach \textsc{Fourier}schen Theorem erwarteten Werten nach Gleichung \ref{eq:koeff} verglichen.

Es wird das Signal eines Funktionsgenerators mit den Kenndaten in Tabelle \ref{tab:kenndaten}
\begin{table}
	\centering
	\begin{tabular}{ccc}
		\toprule
		{Größe}&{Symbol}&{Wert}\\
		\midrule
		{Spannung}&$U_\text{pp}$& $\SI{2}{\volt}$\\
		{Frequenz}&$f$& $\SI{10000}{\hertz}$\\
		\bottomrule
	\end{tabular}
	\caption{Kenndaten und Einstellungen des Funktionsgenerators.}
	\label{tab:kenndaten}
\end{table}
auf ein Oszilloskop gegeben.
Die Signalform wird gewählt aus Rechteck-, Sägezahn- und Dreickspannung.
Dieses Oszilloskop wird im \textit{Math}-Modus betrieben, welcher über eine endliche Zeit eine \textit{Fast-\textsc{Fourier}-Transformation} \cite{FFT} durchführt. 
Nach Abschnitt \ref{sec:theorie3} werden für die periodischen Signale anstelle von diskreten Linien Peaks mit endlicher Breite erwartet. 
Die Höhe der Peaks nehmen im Frequenzspektrum mit wachsender Frequenz ab.\\
Die Amplitude der Elementarschwingung und damit die \textsc{Fourier}-Koeffizienten \ref{eq:koeff} entsprechen der Höhe der Peaks. 
Anschließend wird das Verfahren mit einer anderen Signalform widerholt, sodass alle drei Signalformen analysiert werden.
%

\subsection{\textsc{Fourier}-Synthese}
Aus den Gleichungen \ref{eq:koeff} sind für periodische Signalformen, etwa Rechteck-, Sägezahn- oder Dreickspannungen, die \textsc{Fourier}-Koeffizienten bekannt. 
Mit Hilfe einer Vorrichtung wird die Signalform aus den Elementarschwingungen mit berechneter Amplitude additiv zusammengesetzt.

Die Vorrichtung besteht aus einem Funktionsgenerator, der eine feste Grundschwingung $f=\mathup{const.}$ für $n=1$ und deren Oberschwingungen für $n\in\{2,...,9\}$ ausgibt.
Diese Elementarschwingungen sind in ihrer Amplitude und in ihrer Phase einstellbar, weiter kann das Summensignal ausgegeben werden.
Die einzelnen Oberschwingungen und die Grundschwingung können wahlweise dem Summensignal zugeschaltet werden.

Die aus den Gleichungen \ref{eq:koeff} bekannten Amplituden werden an der Vorrichtung eingestellt und die einzelnen Phase so versetzt, dass die Schwingungen in Phase sind.
Bei $a_n, b_n<0$ kann ein Phasenversatz von $\phi=180°$ eingestellt werden, es gilt
\begin{subequations}
	\begin{equation}
		-\mathup{sin}(x)=\mathup{sin}(x+180°).
	\end{equation}
	\begin{equation}
		-\mathup{cos}(x)=\mathup{cos}(x+180°)
	\end{equation}
\end{subequations}
