\section{Diskussion}
\label{sec:Diskussion}

Bei der \textsc{Fourier}-Analyse treten geringe Abweichungen zwischen berechneten und experimentell bestimmten Spannungswerten auf, deren Wert für größere $n$ ansteigt. 
Da alle Theoriewerte aus der ersten eingestellten Spannung berechnet werden, könnten eventuelle Fehler durch ungenaues Ablesen der dieser Amplitude auftreten. 
Durch weitere Messreihen könnte darüber ein stärkere Aussage getroffen werden.\\
Die Voraussetzung der Integration über eine unendlich lange Zeit kann vom verwendeten Oszilloskop nicht realisiert werden. 
Daher fallen die geraden Koeffizienten der Rechteck- und Sägezahnspannung nicht weg. 
Im Widerspruch zur Theorie nehmen nach Gleichungen \eqref{koeff1} und \eqref{koeff2} verschwindende Koeffizienten einen Wert größer als Null an.
Zusätzlich treten Abweichungen durch eine endliche Genauigkeit der verwendeten Geräte auf.

Prinzipiell lassen sich Spannungen durch die einzelnen \textsc{Fourier}-Koeffizienten so zusammensetzen, dass ihre Form gut zu erkennen ist. 
Rechteck- und Sägezahnspannung jedoch sind unstetige Funtkionen. 
Deswegen treten an den Sprungstellen nach dem \textsc{Gibbs}schen Phänomen Überschwingungen auf, die nicht vermieden werden können. 
Selbst für $n\to \infty$ werden die Abweichungen nicht geringer.
