\section{Auswertung}
\label{sec:Auswertung}
\begin{table}
	\centering
	\sisetup{table-format=2.3}
	\begin{tabular}{S[table-format=2.0] S[table-format=3.0] S[table-format=3.0] }
	\toprule
	%\multicolumn{1}{c}{Zeit} & {Temperaturen} \\
	{$n$} & {$U_n/\:\si{\milli\volt}$} & {$\frac{U_1}{n}/\:\si{\milli\volt}$} & {$Abweichung in \%$} \\
	\midrule

 1 & 920 & 912
 3 & 300 & 288
 5 & 170 & 168
 7 & 116 & 112
 9 &  86 &  80
11 &  64 &  56
	\bottomrule
	\end{tabular}
	\caption{Fourieranalyse der Rechteckspannung.}
	\label{tab:FA_RE}
\end{table}
 


\begin{table}
	\centering
	\sisetup{table-format=2.3}
	\begin{tabular}{S[table-format=2.0] S[table-format=3.1] S[table-format=3.0] }
	\toprule
	%\multicolumn{1}{c}{Zeit} & {Temperaturen} \\
	{$n$} & {$U_n/\:\si{\milli\volt}$} & {$\frac{U_1}{n}/\:\si{\milli\volt}$} & {$Abweichung in \%$} \\
	\midrule

 1 & 580 & 576.0
 3 &  63 &  61.6
 5 &  21 &  20.8
 7 &  10 &  10.4
 9 &   6 &   5.6
11 &   4 &   4.0
	\bottomrule
	\end{tabular}
	\caption{Fourieranalyse der Dreiecksspannung.}
	\label{tab:FA_RE}
\end{table}


