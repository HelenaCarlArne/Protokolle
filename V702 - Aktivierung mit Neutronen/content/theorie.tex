\section{Ziel}
\label{sec:Ziel}
Die Neutronen- und Protonenanzahl liegt für stabile Kerne innerhalb bestimmter Grenzen. Werden diese überschritten, so liegen instabile Kerne vor, die in einem Zeitraum mit bestimmter Wahrscheinlichkeit zerfallen.
Versuchsziel ist es, die Halbwertszeiten $T$ bestimmter Nuklide zu bestimmen.
\section{Theorie}
\label{sec:Theorie}
\subsection{Reaktionen von Atomkernen mit Neutronen}
Kerne, deren Halbwertszeit zwischen Sekunden und Stunden liegt müssen vor Experimenten hergestellt werden, indem sie mit Teilchen aktiviert werden. Neutronen eigenen sich besonders gut, da sie keine Ladung besitzen und somit die Coulombbarriere des Kerns nicht überwinden müssen.
Wird ein Kern von einem Neutron getroffen nimmt dieser das Neutron auf. Dabei wird die Energie des entstandenen Zwischenkerns -- auch Compoundkern genannt -- um die kinetische Energie und Bindungsenergie des Neutrons angehonen. Die dazukommende Energie verteilt sich instantan im Kern, sodass kein Nuklid abgestoßen werden kann. Die überschüssige Energie muss den angeregten Kern in Form eines emittierten $\gamma$--Quants verlassen. Dieser instabile Zwischenkern erlangt durch den $\beta⁻$--Zerfall eine stabile Konfiguration.
\begin{equation}
\ce{^{m}_zA + n -> ^{m+1}_zA^{\text{*}} -> ^{m+1}_zA + \gamma}
\end{equation}
\begin{equation}
\ce{^{m+1}_zA -> ^{m+1}_{z+1}C + e⁻ + \bar{{\nu}}_e + E_{kin}}
\end{equation}
Die Wahrscheinlichkeit des Neutroneneinfangs wird bestimmt durch den Wirkungsquerschnitt
\begin{equation}
\sigma=\frac{u}{nKd}
\label{eq:wirkungsquerschnitt}
\end{equation}
mit $[\sigma]= 10^{-24}\si{\centi\meter}²= 1\text{barn}$. Damit ist eine fiktive Fläche gemeint, welche den Kern umgibt. Dabei ist $\sigma$ so gewählt, dass ein Neutron, welches auf diese Fläche trifft, absorbiert wird. $d$ ist dabei die Dicke des Absorbers, $K$ die Anzahl der Atome pro Kubikzentimeter und $n$ die Anzahl der auftreffenden Atome pro Sekunde.
$\sigma$ ist stark von der Geschwindigkeit $v$ der Neutronen abhängig. $v$ ist über die de-Broglie-Wellenlänge
\begin{equation}
\lambda=\frac{h}{mv}
\label{eq:wellenlänge}
\end{equation}
charaktirisiert. Wird die Wellenlänge mit dem Kernradius $R$ verglichen gilt für schnelle Neutronen $\lambda<<R$, sodass die Wechselwirkungsprozesse durch einfache geometrische Überlegungen in Analogie zur geometrischen Optik beschrieben werden können. Für langsame Neutronen ist $\lambda>R$.
Da $\sigma$ abhängig von der Energie der Neutronen ist, gilt
\begin{equation}
\sigma(E)=\sigma_0\sqrt{\frac{E_\mathup{r}}{E}}\frac{\tilde{c}}{(E-E_\mathup{r})²+\tilde{c})}.
\label{eq:wirkungsquerschnitt_energie}
\end{equation}
Dabei sind $\tilde{c}$ und $\sigma_0$ konstant; $E_\mathup{r}$ repräsentiert die Energieniveaus des Zwischenkerns.
 Entspricht die Energie $E_\mathup{n}$ eines Neutrons der Energiedifferenz $E_\mathup{1}-E_2=\mathup{\Delta{E}}$ zweier Energieniveaus des Kerns findet Resonanzabsorption statt, da $\sigma(E)$ maximal wird. Für $E_\mathup{n}<<E_\mathup{r}$ kann $(E-E_\mathup{r})²$ als konstant angesehen werden. Daraus folgt, dass $\sigma\propto \frac{1}{\sqrt{E}}\propto\frac{1}{v}$ ist. Langsame Neutronen befinden sich also für längere Zeit in der Einwirkungssphäre des Kerns als schnelle Neutronen. Der Wirkungsquerschnitt ist für langsame Neutronen also sehr viel größer als für schnelle. Die Wahrscheinlichkeit des Neutroneneinfangs und damit der Anregung der Nuklide ist für langsame Neutronen eher gegeben.
\subsection{Thermische Neutronen}
In diesem Falle werden die Neutronen die nachfolgend dargestellte Reaktion gewonnen.  Das $\alpha$-Teilchen stammt hier aus dem Zerfall von $226$-Radon.
\begin{equation}
\ce{^9_4 Be + ^4_2 He -> ^12_9 C + ^1_0 n}
\end{equation}
Die Neutronen besitzen ein kontinuierliches Energiespektrum, welches bis $13,7\,\si\mega \mathup{e}\si\volt$ reicht. Um eine Geschwindigkeit von $v=2,2\,\frac{\si{\kilo\meter}}{\si\second}$ zu erreichen, was einer Energie von $E_\mathup{n}\approx 0,025\,\text{e}\si\volt$ bei $T=290\,\si\kelvin$ entspricht, werden die Neutronen durch Materialschichten abgebremst. 
Durch eine Vielzahl elastischer Stöße wird die Energie der Neutronen abgegeben. Ähneln sich die Stoßpartner, ist die Bremsung am effektivsten. Sie wird hier mit Paraffin erreicht. 
\subsubsection{Zerfall instabiler Isotope}
\label{sec:zerfall}
Es werden die Zerfallsprozesse von Indium und Rhodium untersucht.
Für den Zerfall bei Indium gilt
\begin{equation}
\ce{^{115}_{59}In + ^{1}_0n -> ^{116}_{50}Sn + \beta^- + \bar{\nu_\mathup{e}}}.
\end{equation}
 Wird \ce{^103 Rh} von einem Neutron getroffen besteht die Möglichkeit, dass mit unterschiedlich großer Wahrscheinlichkeit verschiedene Reaktionen ablaufen:
\begin{equation}
\ce{^{103}_{45}Rh + ^{1}_0n -> ^{104i}_{45}Rh -> ^{104}_{45}Rh + \gamma -> ^{104}_{46}Pd + \beta^- + \bar{\nu_\mathup{e}}}  \quad ,10\%
\end{equation}
\begin{equation}
\ce{^{103}_{45} Rh + ^1_0 n -> ^{104}_{45} Rh-> ^{104}_{46} Pd + \beta^- + \bar{\nu_\mathup{e}}} \quad ,90\%.
\end{equation}
Dabei meint \ce{^{104i}_{45}Rh} den isomeren Zwischenkern. Eine andere Anordnung der Nuklide ruft eine andere Halbwerrtszeit und Energie hervor.
Das Zerfallsgesetz ist 
\begin{equation}
N(t)=N_0e^{-\lambda t}.
\label{eq:zerfallsgesetz}
\end{equation}
Aus 
\begin{equation}
\frac{1}{2}N(t)=N_0e^{-\lambda T}
\label{eq:zerfallsgesetz_0.5}
\end{equation}
kann die Halbwertszeit $T$ zu
\begin{equation}
T=\frac{\ln{2}}{\lambda}
\label{eq:halbwertszeit}
\end{equation}
bestimmt werden.
Einfacher ist es jedoch, $T$ über das Zeitintervall $\Delta{t}$ zu bestimmen. Aus
\begin{equation}
N_\mathup{\Delta{t}}=N(t)-N(t+\Delta{t})
\end{equation}
folgt mit Gleichung \eqref{eq:zerfallsgesetz} für die Anzahl der zerfallenen Kerne
\begin{equation}
N_\mathup{\Delta{t}}=N_0\mathup{e}^{-\lambda t}-N_0\mathup{e}^{-\lambda(t+\Delta{t})}.
\end{equation}
Umformungen ergeben
\begin{equation}
\ln{(N_\mathup{\Delta{t}})}=\underbrace{\ln{(N_0(1-\mathup{e}^{-\lambda \Delta{t}}))}}_{\text{a}}-\lambda t
\label{eq:startwert}
\end{equation}
mit dem konstanten Term $a$.
