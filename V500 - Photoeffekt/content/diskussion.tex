\section{Diskussion}
\label{sec:Diskussion}
\subsection{Fehleranalyse} % (fold)
\label{sub:Diskussion1}
Als optisches Experiment wird vorzugsweise dunkle Umgebung gefordert.
Weiter sind wegen der geringen Stromstärken  empfindliche Messgeräte erforderlich, deren Kabel von Störfeldern befreit sein müssen.\\
Da ein Experimentieren unter Ausschluss sämtlicher Fremdlichtquellen nicht möglich ist, konnte der dadurch entstandene Fehler gering gehalten werden, indem Fremdlicht konstant gehalten wird.
Der Einfluss von Störfelder wurden durch die Wahl von Koaxialkabeln verringert.
% subsection fehleranalyse (end)
\subsection{Bestimmung des Planckschen Wirkungsquantums} % (fold)
\label{sub:Diskussion2}
Die zu bestimmende Größe, $\sfrac{h}{e}$, wurde in Abschnitt \ref{sec:Auswerung1} zu 
\begin{equation}
	\frac{h}{e}=\SI{3.38(64)e-15}{\weber}
\end{equation}
bestimmt.
Der Mittelwert weicht von der Literaturangabe dieses Wertes, $\sfrac{h}{e}=\SI{4.13e-15}{\weber}$ \cite{texas_instruments1},\cite{texas_instruments2}, um $18.3\,\%$ 
ab. Der Literaturwert liegt in der Nähe der Standardabweichung des Messwertes.
Das Ergebnis zeigt die Präzission des Experimentes an.
% subsection diskussion_d (end)
\subsection{Erklärung des Stromverlaufes bei Abbremsung} % (fold)
\label{sub:Diskussion3}
In Abschnitt \ref{sec:Auswertung2} wird das Verhalten der Photodiode bei hohen Spannungen aufgezeigt.
Werden stark-bremsende Spannungen angelegt, so stagniert der Photostrom bei einem festen, geringen Wert.
Dies wird durch durch positive Spannungen bis etwa $\SI{20}{\volt}$ realisiert.
Der Stromfluss ist der Richtung des unbeeinflussten Photostroms entgegengesetzt.

Zur Erklärung wird die Photokathode herangenommen. 
Sie besteht aus einer dünnen, metallischen Schicht, welche bei Betriebstemperatur zum Teil verdampft.
Im evakuiertem Glaskörper der Photodiode befindet sich daher ein Aerosol aus Kathodenmaterial, das imstande ist, 
geringe Ströme zu übertragen. 
Diese Ströme sind auf die hohe Spannung zwischen Kathode und Anode zurückzuführen, deren Ausgleich über das Gas geschieht.
Dass der negative Strom bereits für kleine Spannungen erreicht wird, zeigt, dass die Anode nur ein schwaches Gegenfeld aufbaut.
Die Austrittsarbeit der Anode scheint gering.\\
Für Spannungen größer als $\SI{20}{\volt}$ wird die Zerstörung der Photodiode durch Funkenschlag erwartet.

\subsection{Erklärung des Stromverlaufes bei Beschleunigung} % (fold)
\label{sub:Diskussion4}
Wird die Polung der Spannung gedreht, sodass die Photokathode negativ und die Anode positiv geladen ist, 
werden Photoelektronen beschleunigt.
In Abschnitt \ref{sec:Auswertung2} wird beschrieben, dass der Photostrom mit steigender Spannung wächst und einen Grenzwert erreicht.
Zur Erklärung des Grenzwertes wird der Einfluss von Intensität auf den Photostrom beschrieben.\\
Der Betrag des Photostromes ist abhängig von der Intensität des Lichtes; je höher die Lichtintensität ist, desto größer ist der Photostrom.
In Abwesenheit von Photonen werden nur geringe Mengen von Elektronen aus der Kathode gelöst (vgl. \textsc{Edison}--\textsc{Richardson}-Effekt), der Photostrom wird trotz der beschleunigenden Spannung im Wesentlichen von den Photonen ausgelöst.
Da die Intensität des Lichtes ein Maß für die Anzahl der Photonen ist, ist mit fester Intensität des einstrahlenden Lichtes ein Grenzwert für den Photostrom gegeben.
Die Lichtfrequenz hat keinen Einfluss auf den Sättigungswert.

Der Sättigungswert wird asymptotisch erreicht, da die von Photonen ausgelösten Elektronen -- anders als bei stimulierter Emission -- keine vorgegebene Richtung haben.
Die gesamte kinetische Energie eines ausgelösten Elektrons besteht aus der Energie $E_\text{kin}$, die nach Abzug der Austrittsarbeit von der Photoenergie $h\nu$ den Elektronen zur Verfügung steht, 
und der Energie, die das Elektron nach der \textsc{Fermi}--\textsc{Dirac}-Statistik besitzt.
Die kinetische Energie der Elektronen ist unregelmäßig verteilt.
Dieser Sachverhalt erklärt ebenfalls das nicht-abrupte Abbrechen des Photostromes in der Nähe von $U_\text{G}$.

Um den Sättigungswert bei geringen Spannungen zu erreichen, kann der Aufbau der Photodiode angepasst werden.
Hierzu muss die bestrahlte Fläche der Kathode optimal sein und 
die Absorption des Lichtes von dem Glaskörper klein gehalten werden.
Der Sättigungswert der in diesem Experiment verwendeten Photodiode zeigt, dass die Intensität des Lichtes hoch und die Photokathode optimal beleuchtet wurde.
