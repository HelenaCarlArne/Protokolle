\section{Diskussion}
\label{sec:Diskussion}
\subsection{Fehleranalyse} % (fold)
\label{sub:Diskussion1}
Für Experimente in der Optik werden dunkle Umgebungen bevorzugt.
Weiter sind wegen der geringen auftretenden Stromstärken $I$ empfindliche Messgeräte erforderlich, deren Kabel von Störfeldern befreit sein müssen.\\
Da ein Experimentieren unter Ausschluss sämtlicher Fremdlichtquellen nicht möglich ist, konnte der dadurch entstandene Fehler gering gehalten werden, indem das Fremdlicht konstant gehalten wird.
Der Einfluss von Störfelder wurden durch die Wahl von Koaxialkabeln verringert.
% subsection fehleranalyse (end)
\subsection{Bestimmung des Planckschen Wirkungsquantums} % (fold)
\label{sub:Diskussion2}
Die zu bestimmende Größe $\sfrac{h}{e}$ wurde in Abschnitt \ref{sec:Auswertung1} zu 
\begin{equation}
	\frac{h}{e}=\SI{3.4(6)e-15}{\weber}
\end{equation}
bestimmt.
Der Mittelwert weicht von der Literaturangabe dieses Wertes \\
$\sfrac{h}{e}=\SI{4.13e-15}{\weber}$ \cite{texas_instruments1},\cite{texas_instruments2} um $17.7\,\%$ 
ab. Der Literaturwert liegt in der Nähe der Standardabweichung des Messwertes.
Das Ergebnis zeigt die verhältnismäßig hohe Genauigkeit des Experimentes an.
% subsection diskussion_d (end)
\subsection{Erklärung des Stromverlaufes bei Abbremsung} % (fold)
\label{sub:Diskussion3}
In Abschnitt \ref{sec:Auswertung2} wird das Verhalten der Photodiode bei hohen Spannungen $U$ aufgezeigt.
Werden stark bremsende Spannungen $U$ angelegt, so stagniert der Photostrom $I$ bei einem festen, geringen Wert.
Dies wird durch Spannungen $U\approx\SI{20}{\volt}$ realisiert.
Der Stromfluss ist der Richtung des unbeeinflussten Photostroms entgegengesetzt.

Zur Erklärung wird die Photokathode betrachtet. 
Sie besteht aus einer dünnen, metallischen Schicht, welche bei Betriebstemperatur zum Teil verdampft.
Im evakuiertem Glaskörper der Photodiode befindet sich daher ein Aerosol aus Kathodenmaterial, das imstande ist, 
geringe Ströme zu übertragen. 
Diese Ströme sind auf die hohe Spannung $U$ zwischen Kathode und Anode zurückzuführen, deren Ausgleich über das Gas geschieht.
Dass der negative Strom bereits für kleine Spannungen erreicht wird, zeigt, dass die Anode nur ein schwaches Gegenfeld aufbaut.
Daher scheint die Austrittsarbeit der Anode gering.\\
Für sehr hohe Spannungen $U$ wird die Zerstörung der Photodiode durch Funkenschlag erwartet.

\subsection{Erklärung des Stromverlaufes bei Beschleunigung} % (fold)
\label{sub:Diskussion4}
Wird die Polung der Spannung $U$ gedreht, sodass die Photokathode negativ und die Anode positiv geladen ist, 
werden Photoelektronen beschleunigt.
In Abschnitt \ref{sec:Auswertung2} wird beschrieben, dass der Photostrom $I$ mit steigender Spannung wächst und einen Grenzwert erreicht.
Zur Erklärung des Grenzwertes wird der Einfluss der Intensität auf den Photostrom beschrieben.\\
Der Betrag des Photostromes $I$ ist abhängig von der Intensität des Lichtes; je höher die Lichtintensität ist, desto größer ist der Photostrom $I$.
In Abwesenheit von Photonen werden nur geringe Mengen von Elektronen aus der Kathode gelöst \footnote{vgl. Edison--Richardson-Effekt}, der Photostrom $I$ wird trotz der beschleunigenden Spannung $U$ im Wesentlichen von den Photonen ausgelöst.
Da die Intensität des Lichtes ein Maß für die Anzahl der Photonen ist, ist mit fester Intensität des einstrahlenden Lichtes ein Grenzwert für den Photostrom $I$ gegeben.
Die Lichtfrequenz $\nu$ hat keinen Einfluss auf den Sättigungswert.

Der Sättigungswert wird asymptotisch erreicht, da die von Photonen ausgelösten Elektronen -- anders als bei stimulierter Emission -- keine vorgegebene Richtung haben.
Die gesamte kinetische Energie eines ausgelösten Elektrons besteht aus der Energie $E_\text{kin}$, die nach Abzug der Austrittsarbeit $W_\text{K}$ von der Photoenergie $h\nu$ den Elektronen zur Verfügung steht, 
und der Energie $\zeta$, die das Elektron nach der \textsc{Fermi}--\textsc{Dirac}-Statistik besitzt.
Die kinetische Energie $E_\text{kin}$ der Elektronen ist unregelmäßig verteilt.
Dieser Sachverhalt erklärt ebenfalls das nicht-abrupte Abbrechen des Photostromes in der Nähe von $U$.

Um den Sättigungswert bei geringen Spannungen zu erreichen, kann der Aufbau der Photodiode angepasst werden.
Hierzu muss die bestrahlte Fläche der Kathode größtmöglich sein und 
die Absorption des Lichtes von dem Glaskörper klein gehalten werden.
Das Erreichen des Sättigungswertes der in diesem Experiment verwendeten Photodiode zeigt, dass die Intensität des Lichtes hoch und die Photokathode optimal beleuchtet wurde.
