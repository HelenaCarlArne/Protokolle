\section{Diskussion}
\label{sec:Diskussion}
\subsection{Eichung}
Um mit höherer Fehlersicherheit die Amplituden und Phasendifferenzen bestimmen zu können, wird die Temperaturwelle als Überlagerung zweier Funktionen angenommen. 
\begin{equation}
	\label{annahme}
	f_\text{Schwingung}= f_\text{Grundschwingung}+f_\text{Amplitude}
\end{equation}
Die in Abschnitt \ref{sec:Auswertung} als Amplitudenfunktion bezeichnete Funktion der Klasse \eqref{Amplitudenklasse} beschreibt das Ansteigen der Schwingung durch Erwärmung des Stabes im Ganzen; die Grundschwingung ist im Idealfall eine reine trigonometrische Funktion (vgl. Gleichung \ref{temperaturwelle} bei festem $x$). 
Dass die Erwärmung des Stabes als eine solche $e$-Funktion angesehen werden kann, wird in Diagramm \ref{fig:entftemp} ersichtlich.
Da die Amplitudenfunktionen sehr ähnlich verlaufen und sich im Wesentlichen durch den y-Achsenabschnitt voneinander unterscheiden, kann in guter Näherung die Schwingung wie angegeben aufgespalten werden, wodurch die wahre Amplitude $T_\text{max}$ der Grundschwingung gut angenährt wird.
Dies rechtfertigt die Annahme, die Temperaturwelle \ref{temperaturwelle} durch \eqref{annahme} anzunähern.

%Würden sich die Amplitudenfunktionen stark unterscheiden, würde dies zu großen Abweichungen führen, da die echte Temperaturwelle \ref{temperaturwelle} keine Superposition von Funktionen ist.

\subsection{Wärmekapazität}
Vergleicht man die Literaturwerte der Wärmekapazitäten für Aluminium, Messing und Edelstahl mit den experimentell gewonnenen Werten wie in Tabelle \ref{tab:waermeleitfaehigkeitwerte}, ergeben sich Abweichungen die im Rahmen einer brauchbaren Messung.
Die geringen relativen Abweichungen von den Literaturwerten der Wärmeleitfähigkeiten von Messing und Aluminium zeugen von guter Messung, 
die absolute Abweichung von Edelstahl, die im Bereich der absoluten Abweichungen der anderen Metalle liegt, zeigt eine höhere relative Abweichung.
Daraus folgt, dass eine gute Aussage über die Wärmeleitfähigkeiten getroffen werden kann.

Die Bestimmung der Wärmeleitfähigkeit $\kappa$ in Abschnitt \ref{sub:waermekapazitaet} verifiziert somit die Behauptung in Abschnitt \ref{sub:statik}, dass Aluminium in der Auswahl der Metalle die größte und Edelstahl die geringste Wärmeleitfähigkeit aufweist.

\subsection{Von den Messfehlern und Fehlerrechnung}
Wesentliche Fehlerquellen liegen im Ablesen des Abstands $\Delta{x}$ der Thermoelemente voneinander und im Einfluss der Raumtemperatur.
Das Ablesen geschieht mit einem handelsüblichen Lineal und aufgrund der Bauart der Platine ist exaktes Abmessen erschwert. 
Gegen den Einfluss der Raumtemperatur wird eine Abschirmung benutzt, welche ein übermäßiges, unbeabsichtigtes Abkühlen auf ein vertretbares Maß reduziert. Des Weiteren ist die Präzison der Thermoelemente unbekannt, weshalb eine Fehlerrechnung seitens der Temperatur nicht möglich ist.
Eine exakte Fehlerrechnung ist dadurch nicht möglich.
Liegen Verunreinigungen in den Metallen vor, die nicht nachgewiesen werden können, so kann die Vergleich der gemessenen Wärmeleitfähigkeiten mit der Literatur kritisch sein, 
da die Verunreingungen Einfluss auf themodynamische Parameter haben könnten. Ebenso lässt sich für Messing und Edelstahl nur ein breit gefächerter Bereich der Wärmeleitfähigkeit, je nach Zusammensetzung, in der Literatur finden, sodass ohne genaue Kenntnis der Probenzusammensetzung kein exakter Literaturwert angegeben werden kann.

\subsection{Zusammenfassung}
Bei sorgfältiger Messung eignet sich der Versuch gut zur Bestimmung der Wärmeleitfähigkeiten, sofern nur der ungefähre Wert interessiert oder sofern ein Wert für die Wärmeleitfähigkeit verifiziert werden soll.\\
Eine präzise Messung der Wärmeleitfähigkeiten von Metallen ist nur dann durch diesen Versuch möglich, wenn hochreine Metalle untersucht werden und wenn die Isolierung die Metalle effektiv vor dem Einfluss der Raumtemperatur bewahrt. 
Alternativ kann, sofern die Platine diese Umgebung zulässt, im Vakuum gearbeitet werden.
Weiter müssen zur präzisen Messung die Eichung der Thermoelemente sowie deren Abstand und Unsicherheit bekannt sein.

