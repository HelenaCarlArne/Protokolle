\section{Ziel}
\label{sec:ziel}

Ziel des Versuches ist, die Wärmeleitfähigkeit von den Metallen Messing, Edelstahl und Aluminium zu bestimmen. Dafür werden diese über ein Peltier-Element erhitzt und die Temperatur der Proben an zwei verschiedenen Orten gemessen.

\section{Theorie}
\label{sec:theorie}

Ist in einem abgeschlossenen System ein Temperaturunterschied vorhanden, findet Wärmetransport statt, um ein Temperaturgleichgewicht zu erreichen. Dies kann durch Konvektion, Wärmestrahlung oder Wärmeleitung geschehen. 
Bei Konvektion vermischen sich bei Fluiden unterschiedliche warme Temperaturbereiche; 
durch Wärmestrahlung gibt ein Körper oder ein Fluid seine Wärme an die Umgebung ab.

Die in diesem Versuch betrachtete Wärmeleitung in festen Körpern geschieht über freie Elektronen und Phononen, quantisierte Schwingungen innerhalb der Gitterstruktur des Metalls.
Dabei fließt eine Wärmemenge
\begin{equation}
	\label{waermemenge}
	\mathup{d}Q=-\kappa A\frac{\partial{T}}{\partial{x}}\mathup{d}t
\end{equation}
durch den Festkörper mit Querschnittsfläche $A$ von hoher zu niedriger Temperatur.
Die Wärmeleitfähigkeit $\kappa$ ist eine Materialkonstante.
Mit \eqref{waermemenge} und der Wärmestromdichte $j_\mathup{w}$, welche den Wärmestrom mit 
\begin{equation}
	\label{waermestromdichte}
	j_\mathup{w}= -\kappa \frac{\partial{T}}{\partial{x}}
\end{equation}
bezüglich der Querschnittsfläche $A$ darstellt,
kann mit Hilfe der Kontinuitätsgleichung die eindimensionale Wärmeleitungsgleichung
\begin{equation}
	\label{waermeleitungsgleichung}
	\frac{\partial{T}}{\partial{t}} =  \frac{\kappa}{c\rho}\frac{\partial^2{T}}{\partial{x^2}}
\end{equation}
hergeleitet werden.
Diese beschreibt die räumliche und zeitliche Entwicklung der Temperaturverteilung. 
Dabei ist
\begin{equation}
	\label{temperaturleitfaehigkeit}
	\sigma_\mathup{T}=\frac{\kappa}{{\rho}c}
\end{equation}
die Temperaturleitfähigkeit, welche die Schnelligkeit angibt, mit welcher der Temperaturausgleich stattfindet.
Sie wird bestimmt durch die Dichte $\rho$ und die spezifische Wärmekapazität $c$. 
Dabei beschreibt $c$ die Energiemenge, die benötigt wird, um $\SI{1}{\kilo\gram}$ des untersuchten Materials um $\SI{1}{\kelvin}$ zu erwärmen

Werden Körper durch periodischen Temperaturwechsel geheizt oder gekühlt, breiten sich in seinem Innern Temperaturwellen der Form
\begin{equation}
	\label{temperaturwelle}
	T(x,t)= T_\mathup{max} \exp{\left(-\sqrt{\frac{\omega \rho c}{2\kappa}}x\right)} \cos{\left(\omega t - \sqrt{\frac{\omega \rho c}{2\kappa}}x\right)}
\end{equation}
aus, was eine Lösung der Wärmeleitungsgleichung \eqref{waermeleitungsgleichung} für lange Probenstäbe darstellt. 
Diese setzt sich zusammen aus der Amplitude $T_\text{max}$, der Wellenzahl
\begin{equation}
	\label{wellenzahl}
	k=\frac{\omega \rho c}{2\kappa},
\end{equation}
einem Schwingterm und einer Exponentialfunktion mit negativem Exponenten, der für die Dämpfung verantwortlich ist.
Über die Dispersionsbeziehung, die Phasengeschwindigkeit $v$ und die Frequenz $\omega$ miteinander zu
\begin{equation}
	\label{dispersion}
	v=\frac{\omega}{k}=\frac{\omega}{\sqrt{\frac{\omega\rho c}{2\kappa}}}=\sqrt{\frac{2\kappa\omega}{c\rho}}
\end{equation}
verknüpft, ergibt sich mit \eqref{wellenzahl}, den Beziehungen die Winkelgeschwindigkeit $\omega=\frac{2\pi}{\tilde{T}}$ und die Phase $\phi=2\pi\Delta{t}\frac{1}{\tilde{T}}$ mit der Periodendauer $\tilde{T}$ eine Gleichung für die Wärmeleitfähigkeit.
Es gilt
\begin{equation}
	\label{waermeleitfaehigkeit}
	\kappa=\frac{c\rho{\Delta{x}}^2}{2\Delta{t}\ln\left({\frac{A_\text{nah}}{A_\text{fern}}}\right)}.
\end{equation}
$\frac{A_\text{nah}}{A_\text{fern}}$ ist das Verhältnis der Wellenamplituden, welche an zwei Orten $x_1$ und $x_2$ im Abstand
$\Delta{x}=x_2-x_1$ gemessen werden und den Gangunterschied $\Delta{t}$ aufweisen. 
Hiermit kann $\kappa$ über den zeitlichen Verlauf der Temperatur bestimmt werden.
% section theorie (end)