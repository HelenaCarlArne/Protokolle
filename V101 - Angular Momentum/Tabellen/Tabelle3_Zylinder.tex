\begin{table}
	\centering
	\sisetup{table-format=2.3}
	\begin{tabular}{S[table-format=1.2] S[table-format=1.3] S[table-format=2.3] S[table-format=1.3] S[table-format=3.2]}
	\toprule
	\multicolumn{2}{c}{Schwingungsdauer} & \multicolumn{2}{c}{Abmessungen}&{Masse}\\

{$5T/\:\si{\second}$} & {$T/\:\si{\second}$} & {$\text{Höhe}\:H/\:\si{\centi\meter}$} & {$\text{Durchmesser}\:D/\:\si{\centi\meter}$} & {$M/\:\si{\gram}$}\\
	\midrule
4.41 &  0.882 &	10.120	& 9.848	& 368.57 \\
4.35 &	0.870 & 10.150	& 9.850	& 368.57 \\
4.44 &	0.888 & 10.120	& 9.844	& 368.57 \\
4.44 &	0.888 & 10.102	& 9.840	& 368.57 \\
4.36 &	0.872 & 10.112	& 9.844	& 368.57 \\
4.41 &  0.882 &	10.112	& 9.850	& 368.58 \\
4.32 &	0.864 & 10.058	& 9.850	& 368.59 \\
4.40 &	0.880 & 10.070	& 9.844	& 368.57 \\
4.33 &	0.886 & 10.068	& 9.844	& 368.58 \\
4.39 &  0.878 &	10.072	& 9.850	& 368.57 \\
	\bottomrule
	\end{tabular}
	\caption{Messung zur Bestimmung des Eigenträgheitsmomentes eines Zylinders.}
	\label{tab:M3 I_Z}
\end{table}
