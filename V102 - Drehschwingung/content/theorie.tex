\section{Zielsetzung}
Elastische Konstanten charakterisieren das Verhalten eines Stoffes bei Krafteinwirkung.
Sie fungieren als Proportionalitätsfaktoren zwischen der pro Flächeneinheit angreifenden Kraft und der daraus resultierenden relativen Deformation -- einer Gestalts- oder Volumenänderung. 
Ziel ist es, die elastischen Konstanten einer Metalllegierung zu bestimmen. Außerdem soll das magnetische Moment eines Permanentmagneten, sowie das Erdmagnetfeld gemessen werden.
\section{Theorie}
\label{sec:Theorie}
\subsection{Wirkung von Kräften auf Körper}
Wirken Kräfte auf einen Körper ein, kann dies auf zwei Arten geschehen.
Die Kraft kann einerseits an jedem Volumenelement angreifen, dadurch den Bewegungszustand des Körpers ändern und ihn in eine Translations- oder Rotationsbewegung versetzen.
Andererseits ist es möglich, dass die angreifende Kraft nur auf die Oberfläche des Körpers wirkt und dazu führt, dass sich Gestalt bzw. Volumen ändern. 

Diese Kraft pro Flächeneinheit wird als Spannung definiert. 
Diese teilt sich in zwei Komponenten auf: die Normalkomponente $\sigma$ bewirkt eine Längenänderung senkrecht, eine Tangentialspannung $\tau$ eine Längenänderung parallel zur Kraftrichtung auf eine Probe. 
Die Kräfte wirken nachweislich an der Oberfläche und jeder beliebigen Querschnittsfläche des Körpers. 
In Festkörpern sind die Atome in einem Kristallgitter regelmäßig angeordnet und befinden sich mit ihren direkten Nachbarn in einem Gleichgewichtszustand, in dem sich die abstoßenden und anziehenden Kräfte zu Null addieren. 
Durch Krafteinwirkung wird ein neuer Zustand hergestellt, indem der Abstand $r_0$ der Teilchen zueinander variiert wird und sich ein neuer Gleichgewichtszustand mit dem Abstand $r'_0$ einstellt.
Das \textsc{Hooke}sche Gesetz
\begin{alignat}{4}
\sigma&=E\frac{\mathup{\Delta}{L}}{L}\qquad &&\text{oder} \quad P&&&=Q\frac{\mathup{\Delta}{V}}{V}
\label{eq:hooke}
\end{alignat}
beschreibt für hinreichend kleine Kräfte an der Oberfläche bzw. an jedem Volumenelement einen linearen Zusammenhang zwischen der Spannung und der durch diese hervorgerufenen relative Deformation.
Liegt eine elastische Deformation vor, so kehrt der Körper in seine Ausgangslage zurück, sobald die Krafteinwirkung unterbrochen ist.
Der Vorgang ist reversibel. 
\subsection{Die elastischen Konstanten eines Materials}
Werden unsymmetrische Kristalle mit richtungsabhängigen elektrostatischen Kräften betrachtet, müssen  viele Komponenten gemessen und errechnet werden. 
Isotrope Körper, beispielsweise polykristalline Kristalle, zeichnen sich stattdessen durch richtungsunabhängige elastische Konstanten aus und können theoretisch durch zwei Konstanten beschrieben werden. 
Es erweist sich jedoch als zweckmäßig, insgesamt vier Konstanten einzuführen.
Der Schubmodul $G$ beschreibt die Gestalts-, der Kompressionsmodul $Q$ die Volumenelastizität.
Der Elastizitätsmodul $E$ ist der Proportionalitätsfaktor aus Gleichung \eqref{eq:hooke}. 
Die \textsc{Poisson}sche Querkontraktionszahl
\begin{equation}
\mu=-\frac{\mathup{\Delta}{B}}{B}\frac{L}{\mathup{\Delta}{L}}
\label{eq:mu}
\end{equation}
beschreibt die relative Längenänderung bei angreifender Normalspannung in Spannungsrichtung.
Die genannten Moduln sind nicht unabhängig voneinander und stehen über
\begin{alignat}{3}
	 E=2G(\mu+1) &\quad\text{und} &&\quad E=3(1-2\mu)Q
\label{eq:modulbeziehungen}
\end{alignat}
in Beziehung.

\subsection{Der Schubmodul \texorpdfstring{$G$}{G} eines Drahtes}
Erfährt ein Probekörper ausschließlich Tangentialspannungen, verformt er sich so, dass eine Scherung um den Winkel $\alpha$ auftritt. Nach \textsc{Hooke} ist mit
\begin{equation}
\tau=G\alpha
\label{eq:hooke_G}
\end{equation}
 die Spannung proportional zum Scherungswinkel. 
Experimentell wird ein einseitig fest eingespannter zylinderförmiger Draht um den Winkel $\phi$ verdreht, wobei ein Drehmoment $M$ wirkt. 
Die Schichten des Zylindermantels werden dabei um den Winkel $\alpha$ ausgelenkt. 
Da das Drehmoment abhängig vom Hebelarm, also vom Probendurchmesser ist, werden im weiteren Verlauf infinitisemale Drehmomente d$M$ betrachtet, die eine Kraft d$K$ auf das Massenelement im Radius $r$ bewirken.
Aus
\begin{equation}
\mathup{d}M=r\mathup{d}K
\end{equation}
folgt mit \eqref{eq:hooke_G} und $\tau=\frac{\mathup{d}K}{\mathup{d}F}$
\begin{equation}
\mathup{d}M=rG\alpha\mathup{d}F.
\end{equation}
Aus dem Zusammenhang $\alpha=\frac{r\phi}{L}$, dem Flächeninhalt des Kreisrings und anschließender Integration über den Radius der Probe folgt eine Formel für das Gesamtdrehmoment
\begin{equation}
M=\frac{\pi R^4 G}{2L}\phi=D\phi.
\label{eq:Richtgrosse}
\end{equation}
Wird an das freie Drahtende ein Körper mit Trägheitsmoment $\theta$ angehängt, kann der Draht ungedämpfte harmonische Schwingungen der Dauer 
\begin{equation}
T_{G}=2\pi\sqrt{\frac{\theta}{D}}
\label{eq:t}
\end{equation}
 ausführen. %Es wirken zwei entgegengesetzte Drehmomente aufeinander. Je nach Form des Körpers ist $M_\mathup{T}$ unterschiedlich, hier wird 
 Das Trägheitsmoment einer Kugel ist
\begin{equation}
\theta=\frac{2}{5}m_\mathup{K}{r_\mathup{K}}²,
\label{eq:theta_k}
\end{equation}
 womit sich
\begin{equation}
G=\frac{16\pi m_\mathup{K} {r_\mathup{K}}^2 L}{5T²R⁴}
\label{eq:G}
\end{equation}
ergibt.
Der Vorteil der dynamischen Methode über die Messung der Schwingungsdauer ist die geringere Fehleranfälligkeit. Die Ergebnisse bei statischer Messung können durch eventuelle elastische Nachwirkungen verfälscht werden.

\subsection{Das magnetische Moment \texorpdfstring{$m$}{m}}

Befinden sich Draht und angehängter Körper mit einem magnetischen Moment $m$ in einem homogenen Magnetfeld 
\begin{equation}
B=\frac{8\mu_0}{\sqrt{125}} \frac{I}{R_\mathup{HS}}n,
\label{eq:B_HS}
\end{equation}
erzeugt von \textsc{Helmholtz}-Spulen mit Windungszahl $n$, Radius $R_\mathup{HS}$ und  Spulenstrom $I$, führt er Schwingungen mit 
\begin{equation}
T_m=2\pi\sqrt{\frac{\theta}{mB+D}}
\label{eq:t_hs}
\end{equation}
aus. Die Periodendauer wird geringer, da ein zusätzliches Drehmoment wirkt.

%\subsection{Bestimmung des Erdmagnetfeldes $B_\mathup{E}$}

