\section{Diskussion}
\label{sec:Diskussion}

\subsection{Ergebnis des Versuches}
%Der Lock-In-Verstärker ist awesome.
Der Lock-In-Verstärker eignet sich hervorragend zur Messung von schwachen und gestörten Signalen.
Das Signal kann dabei Störungen aufweisen, die die Signalstärke um Größenordnungen übertreffen.
Ist die Frequenz des zu messenden Signales bekannt, so kann es präzise von den Störungen gefiltert und verstärkt werden.
Dies führt zu einem wesentlichen Unterschied zu reinen Bandpass-Filtern:
im Vergleich zu diesen ist der Lock-In-Filter in der Lage, Signale mit variabler Frequenz zu verstärken. 
Dies folgt daraus, dass der Vorteil des Lock-In-Verstärkers darin besteht, über das Referenzsignal zu gegebener Zeit die gewünschte Frequenz zu filtern.

Im Versuch konnte das \SI{300}{\hertz}-Signal der LED über eine Strecke von etwa $\SI{1}{\meter}$ nachgewiesen werden, wobei keine speziellen Maßnahmen getroffen wurden, um Einflüsse durch Störlicht zu vermeiden.
Die Intensität der LED war dabei nicht wesentlich größer als das Umgebnungslicht des Raumes.

\subsection{Anwendung des Lock-In-Verstärkers}
Anwendung findet der Lock-In-Verstärker beispielsweise in der Elektro- und Informationstechnik zur Übermittlung von Daten über langen Strecken.
Bei der Messung von Signalen sollte im Idealfall die Frequenz des Signales bekannt sein, etwa bei amplitudenmodulierten Wellen.

Ein anderer Verwendungszweck ist die Bestimmung des Phasenversatzes von zwei Signalen. 
Nach Gleichung \eqref{cosinus_ausgangsspannung} ist die ausgegebene Gleichspannung maximal, wenn die Eingangssignales keinen Phasenversatz aufweisen. 
In Abschnitt \ref{sec:Auswertung1} wird gezeigt, wie sich ein Phasenunterschied auf die resultierende Spannung auswirkt. 