\section{Diskussion}
\label{sec:Diskussion}
\section{Diskussion}
\label{sec:Diskussion}

Vergleicht man die Literaturwerte der Wärmekapazitäten für Aluminium, Messing und Edelstahl mit den experimentell gewonnenen Werten wie in Tabelle \ref{tab:waermeleitfaehigkeitwerte}, ergeben sich Abweichungen von $\approx60\%$ bei jeder Probe. Trotz der ernormen Abweichung vom Literaturwert kann eine gute Aussage über das Verhältnis der Wärmeleitfähigkeiten getroffen werden, wenn die Stäbe untereinander verglichen werden und der absolute Wert keine Rolle spielt.

Die größte Fehlerquelle liegt im Ablesen des Abstands $\Delta{x}$ der Thermoelemente voneinander. Dies geschieht mit einem handelsüblichen Lineal und exaktes Abmessen ist aufgrund der Bauart der Platine schwierig. Desweiteren ist nicht bekannt, wie genau die Thermoelemente selber die Temperatur messen -- eventuelle Unsicherheiten werden in der Auswertung nicht betrachtet.

Für die Stoffgemische kann kein exakter Literautwert angegeben werden, so lange die genaue Stoffzusammensetzung unbekannt ist. Ebenso können im Reinstoff Aluminium Verunreinigungen für Fehler sorgen. 
