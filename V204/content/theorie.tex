\section{Ziel}
\label{sec:ziel}
Ziel des Versuches ist, die Wärmeleitung von den Metallen Messing, Edelstahl und Aluminium, zu bestimmen.

\section{Theorie}
\label{sec:theorie}

In einem abgeschlossenen System findet Wärmetransport statt, um ein Temperaturgleichgewicht zu erreichen. Dies kann durch Konvektion, Wärmestrahlung oder Wärmeleitung geschehen. Bei Konvektion vermischen sich Gase oder Flüssigkeiten mit unterschiedlichen warmen Temperaturbereichen; durch Wärmestrahlung gibt ein Körper mit einer von der Umgebung unterschiedlichen Temperatur die Wärme an diese ab.

Die in diesem Versuch betrachtete Wärmeleitung in festen Körpern geschieht über freie Elektronen und Phononen - Quasiteilchen, welche auf die Energieübertragung durch Gitterschwingungen zurückzuführen sind. 
Dabei fließt eine Wärmemenge 
\begin{equation}
	\label{waermemenge}
	\mathup{d}Q=-\kappa\mathup{A}\frac{\partial{T}}{\partial{x}}\mathup{d}t
\end{equation}

durch den Probestab mit Querschnittsfläche A von hoher zu niedriger Temperatur. Die Wärmeleitfähigkeit $\kappa$ ist eine Materialkonstante, die Wärmeleitfähigkeit.
% section theorie (end)
