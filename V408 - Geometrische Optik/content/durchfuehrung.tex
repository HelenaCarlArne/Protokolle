\section{Durchführung}
\label{sec:Durchfuehrung}
\subsection{Verifikation der Linsengleichung}
Die Brennweite $f$ einer dünnen Linse wird mithilfe der Linsengleichung bestimmt und mit der Herstellerangabe verglichen.
Dazu wird auf einer optischen Bank eine Halogenlampe, ein Gegenstand "\emph{Pearl L}",  eine Linse mit der Brennweite $f_1=\SI{100}{\milli\meter}$ und ein Schirm gestellt. \\
Indem die Gegenstandsweite $g$ festgelegt und die Bildweite $b$ so variiert wird, dass der Gegenstand auf dem Schirm scharf abgebildet wird, wird die Bildweite $b_\text{i}$ für jeweils zehn verschiendene Gegenstandsweiten $g_\text{i}$ gemessen.
Um die Messgenauigkeit graphisch darzustellen, werden in einem $b$-$g$-Diagramm die ermittelten Werte eingezeichnet und linear verbunden.
Dies wird für eine Linse mit einer Brennweite von $f_2=\SI{50}{\milli\meter}$ wiederholt.


\subsection{Brennweitenbestimmung nach \textsc{Bessel}}
Die Messvorrichtung wird analog zum ersten Teil aufgebaut.
Der Abstand \\$e=b+g$ zwischen Gegenstand und Schirm wird festgehalten und zur Abbildung eine Linse bekannter Brennweite $f$ benutzt, wobei 
\begin{equation}
	e\ge4f
\end{equation}
gelten muss.
Für zehn verschiedene Abstände $e_i$ werden jeweils die beiden Linsenpositionen gesucht, die ein scharfes Bild erzeugen. 
Die beiden Wertepaare $(b_i,g_i)$ pro Abstand $e_i$ werden aufgenommen.

\subsection{Bestimmung der Lage von Hauptachsen nach \textsc{Abbe}}
Die Messvorrichtung wird analog zum ersten Teil aufgebaut, zusätzlich wird eine Zerstreuungslinse unmittelbar vor der Sammellinse aufgebaut.
Die Zerstreuungslinse wird zum Gegenstand gerichtet, das Linsenpaar wird als ein festes Linsensystem betrachtet und der Mittelpunkt der Sammellinse als Messpunkt $A$ betrachtet.

Ausgehend vom Messpunkt $A$ werden die Gegenstands- und Bildweitenpaar ($b'$,$g'$) sowie die Vergrößerung $V$ des Gegenstandes bei scharfer Abbildung gemessen. Mithilfe der Gleichungen \eqref{eq:abbe} kann die Lage der Hauptachsen $h$ und $h'$ ausgehend vom Messpunkt $A$ bestimmt werden.
Hierzu werden in einem Diagramm
$g′$ gegen $(1+\frac{1}{V})$ und $b′$ gegen $(1+V)$ aufgetragen und mittels Regression die Brennweite $f$ und die relativen Lagen $h$ und $h'$ bestimmt.

\newpage