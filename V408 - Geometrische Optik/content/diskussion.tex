\section{Diskussion}
\label{sec:Diskussion}
\subsection{Fehlerdiskussion}
Über die einzelnen Teile des Experimentes hinweg wird eine scharfe Abbildung des Gegenstandes "Perl L" gefordert.
Die in Abschnitt \ref{sec:theorie} genannten Abbildungsfehler, 
insbesondere die in Abschnitt \ref{sec:auswertung2} bestätigte chromatische Abberation, 
erschweren das Finden der richtigen Projektionsweiten. 
Das exakte Bestimmen der Projektionsweiten ist ohne weitere Maßnahmen oder geräte-unterstützte Messung, etwa durch einen CCD-Chip, nicht möglich.
Durch Bisektion kann die Größenordnung und die Umgebung von $b$ und $g$ bestimmt werden; dies weist sich als eine brauchbare Näherung.
Im Abschnitt \ref{sec:auswertung1} wird die verhältnismäßig hohe Sicherheit in $b$ und $g$ besonders durch das Diagramm \ref{fig:bgdiagramm} erkennbar.

\subsection{Linsengleichung}
Mit einer Abweichung von wenigen Prozent von der Herstellerangabe, konnte die Brennweite einer Linse mithilfe der Linsengleichung \eqref{eq:linsengleichung} berechnet werden.
Die Messung bestätigt die damit Gültigkeit der Linsengleichung.

\subsection{Methode nach \texorpdfstring{\textsc{Bessel}}{Bessel}}
Die Methode von \texorpdfstring{\textsc{Bessel}}{Bessel} kann mit der konventionellen Methode über die Linse 1 verglichen werden.\\
Die Abweichung des Mittelwertes von der Herstellerangabe ist ein direktes Maß für die Fehleranfälligkeit der Methoden. 
Es ist erkennbar, dass die Methode nach \texorpdfstring{\textsc{Bessel}}{Bessel} für die in diesem Experiment durchgeführte Bestimmung der konventionelle Methode unterlegen ist.

Für die Methode nach \texorpdfstring{\textsc{Bessel}}{Bessel} werden der Abstand von Gegenstand und Schirm sowie die Differenz der Projektionslängen benötigt. 
Da diese in abgewandelter Form ebenfalls für die Referenzmethode gilt, kann bei der höheren Unsicherheit von statistischen Fehlern ausgegangen werden.

\subsection{Methode nach \texorpdfstring{\textsc{Abbe}}{Abbe}}
Die Standardabweichung der relativen Lage $h$ und $h'$ von den Hauptachsen zeigen mit 10\% und 25\% starke Unsicherheit an.
Dass die Summe eines Linsensystems über die Summe der Brechkräfte $D$ beschrieben werden kann, wird mit diesem Ergebnis nicht bestätigt.