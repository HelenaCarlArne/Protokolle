\section{Zielsetzung}
Ziel des Experimentes ist, anhand einer elektrischen Schaltung das Verhalten eines schwingfähigen Systems zu erkunden. 
Betrachtet werden eine reine Schwingung und eine erzwungene Schwingung jeweils mit Dämpfung.

\section{Theorie}
\label{sec:Theorie}
\subsection{Reine Schwingung mit Dämpfung}
Das System wird in \ref{fig:schaltkreis_rein} gezeigt, die jeweiligen Spannungsabfälle und -quellen sind eingezeichnet.
Nach der Zweiten Kirchhoffschen Regel ist in einer Masche die Summe aller Spannungen gleich Null. 
Daraus folgt für das System
\begin{equation}
	U_\text{R}(t)+U_\text{C}(t)-U_\text{L}(t)=0
\end{equation}
unter Beachtung der Stromrichtung der Komponenten.
Mit den allgemeinen Formeln für die Spannung der jeweiligen Schaltkreiskomponente,
\begin{align}
	U_\text{R}(t) &= \text{R} I(t)\\
	U_\text{C}(t) &= \frac{Q(t)}{\text{C}} \\
	U_\text{L}(t) &= -\text{L}\frac{\mathup{d}I(t)}{\mathup{d}t},\\
\end{align}
und der Substitution
\begin{equation}
	I=\frac{\mathup{d}Q(t)}{\mathup{d}t},
\end{equation}
wird das System mit der Differentialgleichung für den Strom $I(t)$ beschrieben.
Es gilt
\begin{equation}
	\frac{\mathup{d^2}I(t)}{\mathup{d}t^2}+\frac{\mathup{R}}{\mathup{L}}\frac{\mathup{d}I(t)}{\mathup{d}t}+\frac{1}{\mathup{LC}}.
\end{equation}
Die allgemeine Lösung der Differentialgleichung ist
\begin{equation}
	I(t) = A_1\cdot e^{\mathup{\omega_1}t} +A_2\cdot e^{\mathup{\omega_2}t}
	\label{eq:allgloesung}
\end{equation}
mit der Abkürzung
\begin{align}
	\mathup{\omega_{1,2}}= \frac{\mathup{R}}{2\mathup{L}}\pm\sqrt{\frac{1}{2\mathup{LC}}-\frac{\mathup{R^2}}{4\mathup{L^2}}}.
\end{align}
Für die präzise Lösung wird zwischen den drei Fällen unterschieden, die durch den Wurzelterm gegeben sind.
%\subsubsection{Schwingfall}
Für $\frac{1}{2\mathup{LC}}-\frac{\mathup{R^2}}{4\mathup{L^2}}<0$ ist $\omega$ in \eqref{eq:allgloesung} komplex.\\
%\subsubsection{aperiodischer Grenzfall}
Für $\frac{1}{2\mathup{LC}}=\frac{\mathup{R^2}}{4\mathup{L^2}}$ verschwindet der Wurzelterm von $\omega$ in \eqref{eq:allgloesung}.
Die so vereinfachte Lösung heißt aperiodischer Grenzfall und beschreibt den ausgezeichneten Stromverlauf, der nach Auslenkung die Ruhelage - hier $I(t)=0 fur_alle_t$ - am Schnellsten wieder erreicht.\\
%\subsubsection{Kriechfall}
Für $\frac{1}{2\mathup{LC}}-\frac{\mathup{R^2}}{4\mathup{L^2}}>0$ ist der Wurzelterm von $\omega$ in \eqref{eq:allgloesung} rein reell.
Die so vereinfachte Lösung ist eine exponentielle Abnahme.
Es gilt
\begin{equation}
	I(t)=e{−(\frac{\mathup{R}}{\mathup{2L}} − \sqrt{\frac{\mathup{R}}{\mathup{4L}} −\frac{1}{\mathup{LC}}})t}
\end{equation}
- Zwei Zeiten
-Thomson-Schwingdauer
\subsection{Erzwungene Schwingung mit Dämpfung}
Betrachtet wird wie oben DGL mit Inhomogenität. 
Diese setzt voraus, dass die von außen angelegte Spannung eine Schwingung der Form REAL(e-Funkt).

Die Lösung ist
% Geld ist ja bekanntlich im Portmonee. Da steckt man es rein und es kommt nicht da raus. Außer man kauft etwas davon, zum Beispiel Suppe. (In der Tüte oder nicht, dass ist in diesem Fall egal). Aber wie teuer ist Suppe?

% Jeder kennt das berühmte Bild von einer Suppe. Es war aber nicht in einer Tüte, nein, nein, es war in einer (bunten bzw. roten) Dose! Und hätte man die Dose geöffnet? Hätte es jemand gemacht, man könnte es schreiben. Hat aber Keiner. Deswegen kann man es schwer sagen. Und Suppe: Man kann sie auch mit dem Geld bezahlen.

% Das durchschnittliche Einkommen eines deutschen Haushaltes beträgt etwa 2700 €, nach Abzug aller relevanten und laufenden Kosten (Steuern, Miete et cetera) noch etwa 1350 €.

% Man kann aber nicht alles das Geld für die Suppe (oder z.B. Trockenobst) ausgeben. Kann man natürlich, aber da gibt es ja noch andere Sachen.

% Geld wächst ja bekanntlich auch nicht am Baum (genau wie Suppe), schon wieder etwas das auffällt. Aber Suppe kann man zumindest herstellen. Eine durchschnittliche Suppenküche stellt viel Suppe her, zum Beispiel um sie zu verkaufen oder zu vermieten. Aber wie wird die Suppe überhaupt zu Trockensuppe gemacht?

% Die Suppe besteht bestimmt so aus 20\% Wasser, vielleicht mehr. In der Tüte, da kommt noch Wasser zur Suppe rein, dann ist es Suppe. Also Wasser zum Suppenmehl. Typisch deutsch? Weit gefehlt, auch in China gibt es mittlerweile mehrere Prozent Suppe aus dem Kochbeutel. Auch in Norwegen oder Albanien.

% Aber wieso eigentlich ausgerechnet Suppe? Warum nicht zum Beispiel Gänsebraten oder so was? Ganz einfach: Suppe enthält viele Vitamine. So wie Saft. Und die erstaunliche Parallele: Saft und Suppe kann man beides kalt essen. Es geht bei Suppe also vielmehr um die Temperatur (in °C oder Kelvin, aber nicht in Fahrenheit (oder so)). Probieren Sie es selber aus: Suppe ist ein Gericht, dass am besten kalt serviert wird.

% Genauso wie Geld. Denn das gibt es (es ist also) meistens im Portmonee.
