\section{Diskussion}
\label{sec:Diskussion}

Im Fall der gedämpften Schwingung fällt die Spannung exponentiell ab. Die einhüllende Exponentialfunktion in Abbildung XY stimmt gut mit den Messwerten überein. Der Dämpfungswiderstand $R_\mathup{eff}=(134\pm1)\si{\Omega}$ weist eine Abweichung von $99.40\%$ vom Gerätewiderstand $R_1=(67.2\pm0.2)\si{\Omega}$ auf. Der relativ große Fehler wird durch in der Rechnung nicht betrachtete Leitungs-, Bauteil- und Generatorinnenwiderstände hervorgerufen. Unter Berücksichtigung dieser Widerstände würde sich für den Gesamtwiderstand ein größerer Wert ergeben.
Vergleicht man die gemessene und theoretische Abklingdauer miteinander ergibt sich eine Abweichung von $49.70\%$ bei $T_\mathup{ex}=(0.251\pm0.002)\si\second$ und $T_\mathup{ex,theo}=(0.499\pm0.003)\si\second$. 

Der aperiodische Grenzfall wird bei einem gemessenen Widerstand von $R_\mathup{ap}=4500\si\Omega$ realisiert. Der theoretisch ermittelte Widerstand $R_\mathup{ap,theo}=5700\si\Omega$ weicht um $26.67\%$ ab.

Zwischen den experimentell bestimmten und theoretisch errechneten Resonanzüberhöhungen liegt nur eine geringe Abweichung vor, obwohl viele fehlerbehaftete Größen in deren Berechnung einfließen. $q=5.556$ und $q_\mathup{theo}=(5\pm1)$ unterscheiden sich um $11.12\%$.

Wegen der schwankenden Amplitude $U_0$ lässt sich der Maximalwert der Kondensatorspannung eher ungenau bestimmen. Der ermittelte Wert gibt Informationen über Güte und daher die maximal erreichbare Spannung wider.

Die Schärfe der Resonanz wird über die Resonanzbreite dargestellt. Die Abweichung von $7.69\%$ zwischen $\Delta{f}=7\si\kilo{\hertz}$ und $\Delta{f_\mathup{t}}=6.5\si\kilo{\hertz}$ ist gering trotz der Tatsache, dass $f_+$ und $f_-$ sich durch Auswählen der Messwerte ergeben, zugehörig zu den Bruchteilen der Maximalspannung. Das heißt, dass die Genauigkeit durch die Auflösung der Messwertbestimmung und der Genauigkeit der Messwerte gegeben ist.

Bei geringen Frequenzen -- gegenüber der Resonanzfrequenz -- tritt ein geringer Phasenunterschied auf, der mit steigender Frequenz zunimmz. Auch hier stimmt die experimentelle Tatsache mit dem theoretischen arctan-Zusammenhang überein. 
Eine lineare Darstellung im Resonanzbereich gelingt gut; an den Rändern ergeben sich geringe Abweichungen durch einen etwas zu groß gewählten Resonanzbereich.
