\begin{table}
\centering
\begin{tabular}{S S S[table-format=3.1] S[table-format=3.1] }
\toprule
\multicolumn{2}{c}{Spektrallinie} & \multicolumn{1}{c}{Wellenlänge}&\multicolumn{1}{c}{Winkel}\\
{Intensität} & {Farbe} & {$\lambda\:/\si{\nano\meter}$} & {$\varphi'\:/\si{\degree}$}\\
\midrule
\text{stark}   & \text{gelb}    & 579.1 & 279.6\\
\text{stark}   & \text{gelb}    & 577.0 & 279.9\\
\text{stark}   & \text{grün}    & 546.1 & 282.0\\
\text{schwach} & \text{blaugrün}& 491.6 & 285.7\\
\text{stark}   & \text{violett} & 435.8 & 289.5\\
\text{schwach} & \text{violett} & 434.7 & 289.7\\
\text{stark}   & \text{violett} & 407.8 & 291.5\\
\text{stark}   & \text{violett} & 404.7 & 291.7\\
\bottomrule
\end{tabular}
\caption{Ergebnisse der Berechnung zur Bestimmung der Ladung eines Öltröpfchens.}
\label{tab:hgspektrum}
\end{table}

