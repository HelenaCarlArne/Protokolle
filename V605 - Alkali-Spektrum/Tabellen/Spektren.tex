\begin{table}
\centering
\begin{tabular}{S  S[table-format=3.1] S[table-format=1.4] S[table-format=3.1] S[table-format=1.0(1)] S[table-format=1.0] S[table-format=1.3(1)] S[table-format=1.5(1)] }
\toprule
%\multicolumn{2}{c}{Spektrallinie} & \multicolumn{1}{c}{Wellenlänge}&\multicolumn{1}{c}{Winkel}\\
{Element} & {$\varphi\:/\si\degree$} & {$\varphi\:/\si\degree$} & {$\Delta{s}\:/Skt.$}& {$\lambda\:/\si{\nano\meter}$} & {$\Delta{\lambda}\:/\si{\nano\meter}$}&{$\Delta{E}\pm\Delta{E_{err}}\:/\si{\milli}e\si\volt$}&{$\sigma\pm\sigma_{err}$}\\
\midrule
\text{Na}  & 277.1 &  0.0262 &  50.0 & 615(3) & 0.73 &  2.41(2)    &   7.34066(6)  \\
           & 278.9 & -0.0052 &   4.8 & 589(3) & 0.70 &  0.252(2)   &   8.91784(4)  \\
           & 280.3 & -0.0297 &   3.7 & 568(3) & 0.05 &  0.209(2)   &   9.01436(3)  \\
\text{Ka}  & 279.8 & -0.0209 & 128.0 & 576(3) & 1.88 &  7.03(7)    &  13.0625(1)    \\
           & 279.9 & -0.0227 & 158.0 & 575(3) & 2.32 &  8.73(8)    &  12.7337(1)  \\
           & 283.0 & -0.0768 & 113.0 & 529(2) & 1.65 &  7.34(7)    &  12.9996(1)  \\
           & 283.3 & -0.0820 & 100.0 & 525(2) & 1.46 &  6.60(6)    &  13.1565(1)    \\
\text{Rb}  & 276.4 &  0.0384 & 656.0 & 625(3) & 9.63 & 30.5(3)     &  26.8692 (2)   \\
\bottomrule
\end{tabular}
\caption{Ergebnisse der Berechnung zur Bestimmung der Ladung eines Öltröpfchens.}
\end{table}


