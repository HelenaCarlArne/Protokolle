\section{Diskussion}
\label{sec:Diskussion}
\subsection{Absorptionskoeffizienten}
Mit $0,84\%$ hat der Absorptionskoeffizient von Blei einen sehr geringen relativen Fehler. Auch der Fehler von Eisen mit $10,00\%$ liegt noch in einem annehmbaren Rahmen. Die Werte weichen um $11,91\%$ bzw. $54,11\%$ vom durch die Theorie berechneten Wert ab. Daraus kann auf die auftretenden Wechselwirkungsprozesse geschlossen werden. Es wird bestätigt, dass bei kleinen Kernladungszahlen der Photoeffekt vernachlässigt werden kann, wohingegen bei Blei auch der Photoeffekt als Wechselwirkung auftritt.
\subsection{Reichweite und Maximalenergie}
Die Reichweite der Elektronen hat einen relativen Fehler von $6,89\%$; der Fehler der Energie ist $23,33\%$. Vom Literaturwert $E_\mathup{lit}=0,294\,\si{\mega}\mathup{e}\si\volt$ weicht der experimentell bestimmte Wert um $2,04\%$ ab. Die Methode erweist sich als sehr genau; größte Fehlerquelle ist der Nulleffekt. Dieser kann nie komplett ausgeschlossen werden, jedoch aber durch eine genaue Messung für ein genügen großes $t$ bestimmt und bei der Auswertung berücksichtigt werden.
