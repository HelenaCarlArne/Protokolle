\section{Durchf\"uhrung}
\label{sec:Durchfuehrung}

Man nehme γ-Absorptionskurven von einem der Materialien Cu, Zn oder Fe und dazu Pb auf und bestimme daraus mittels Ausgleichsrechnung die Absorptionskoeffizienten und die Größe N(0). 
Als Strahlungsquelle verwende man das Nuklid 137Cs, eventuell auch 60Co. 

Man vergleiche den gemessenen Absorptionskoeffizienten aus mit für \texorpdfstring{$\gamma$}{Gamma}-Strahlung gerechneten Werten für Absorptionskoeffizienten und ziehe daraus Schlüsse über die vorliegenden Absorptionsmechanismen. 

Man nehme eine β-Absorptionskurve bei Aluminium auf und verwende sie zur Bestimmung der Maximalenergie des verwendeten β-Strahlers (99Tc)