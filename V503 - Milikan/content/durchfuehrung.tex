\section{Durchf\"uhrung}
\label{sec:Durchfuehrung}

\subsection{BEstimmung der Elementarladung mitdem Millikan-Versuchsaufbau}
In einem ungeladenen Plattenkondensator wird Öl fein zerstäubt.
Zur Beobachtung der Tröpfchen dient eine Kamera, die mit aureichender Vergrößerung die Tröpfchen innerhalb des Kondensator sichtbar macht.
Es wird die Zeit $t$ gemessen, die ein ausgewähltes Tröpfchen zum Überstreichen einer abgemessenen Strecke $s$ benötigt, und die dadurch die Geschwindigkeit $v$ bestimmt.\\
Ein Teilchen ist als messbar zu betrachten, wenn die Horizontalbewegung minimal ist, 
sich die natürliche Fallgeschwindigkeit $v_0$ ohne Einfluss des elektrischen Feldes in der Größenordnung von $v_0=\SI{0.01}{\centi\meter}$ bewegt und 
das Tröpfchen bei der Zerstäubung eine Ladung aufgenommen hat, sodass es auf das elektrische Feld des Plattenkondensators reagiert.

Ist ein messbares Teilchen erkannt worden, so wird die natürliche Fallgeschwindigkeit $v_0$ gemessen. 
Anschließend wird bei festgelegter Plattenspannung $U$ die Steiggeschwindigkeit $v_\text{auf}$ und die Sinkgeschwindigkeit $v_\text{ab}$ unter Einfluss des elektrischen Feldes gemessen.
Die Messung wird an demselben Tröpfchen mit gleicher Kondensatorspannung solange wiederholt, sodass für jedes Teilchen die   Geschwindigkeiten $v_\text{ab}$ und $v_\text{auf}$ jeweils dreifach gemessen werden.
Nach der Messung eines Tröpfchens wird mithilfe der Gleichung
\begin{equation}
	2\cdot v_0=v_\text{ab}+v_\text{auf}
	\label{eq:plaus_test}
\end{equation}
überprüft, ob das Tröpfchen im Verlauf der Messung die Ladung abgegeben hat und somit für die Auswertung zurückzustellen ist.
Dieses Verfahren wird für vier weitere Tröpfchen unter gleichen Bedingungen wiederholt, sodass für eine angelegte Kondensatorspannung $U$ fünf Messwert-Datensätze von je einem Tröpfchen zur Auswertung bereitstehen.

Die Kondesatorspannung $U$ ist zwischen $\SI{200}{\volt}$ und $\SI{300}{\volt}$ zu wählen.
Das Verfahren wird für insgesamt 5 Spannungen durchgeführt.
%Für jedes Teilchen werden insgesamt sieben Zeiten, entsprechend sieben Geschwindigkeiten, gemessen.

