\newpage
\section{Diskussion}
\label{sec:Diskussion}

\subsection{Fehleranalyse}
Mithilfe der Kontrolle, gegeben durch Gleichung \eqref{eq:plaus_test}, werden 17 Werte ausgewählt.
Die restlichen 8 Tröpfchen sind wegen starker Abweichung von der Erwartung nicht berücksichtigt worden.
Im Verlauf des Experiments sind Öltröpfchen beobachtet worden, die trotz sorgfältiger Zerstäubung keine Reaktion auf elektrische Felder zeigten. 
Daraus kann geschlossen werden, dass einige Tröpfchen ihre Ladung verloren haben.
Für weitere Versuchsdurchführung nach diesem Aufbau ist darauf zu achten, dass die Ladung nicht abgegeben werden kann.
Mögliche Ursache für die Ladungsabgabe sind eine zu hohe Tröpfchendichte,
im Weiteren kann in einem separaten Versuch der Einfluss untersucht werden, 
den der Ladungszustand des Plattenkondensators im Moment der Einspritzung auf die Ladungsabgabe hat.\\
Außerdem wurden vorwiegend Tröpfchen einer Größe für die Auswertung benützt.
Mit großer Variation in der Tröpfchen-Größe kann der Fehler gering gehalten werden, da eventuelle systematische Fehler, die von der Größe abhängig sind, erkannt und ausgeschlossen werden könnten.
Auffällig sind die geringen berechneten Ladungen in Tabelle \ref{tab:T6}, welche zum großen Teil weit unterhalb des Literaturwertes der Elementarladung liegen. Dies deutet auf einen systematischen Fehler hin, der sich durch die gesamte Messung zieht. Größte Fehleranfälligkeit besitzt die Zeitmessung. Durch Reaktionszeit oder menschliches Versagen werden falsche Zeiten aufgenommen. Andere Gründe sind eine fehlerhafte Kondensatorspannung oder korrodierte Kontakte des Kondensators. Bei einer erneuten Messung wäre ein Ausschluss der systematischen Fehler vor Messbeginn sinnvoll.

Wegen der hohen Ausscheiderate die geringere Disparität der Teilchen ist mit hoher statischer Unsicherheit der Ergebnissen zu rechnen.

\subsection{Vergleich mit der Literatur}
Die berechnete Elementarladung $e$ weicht um $32,3\%$ vom Literaturwert ab, die \textsc{Avogadro}-Konstante um $47,9\%$. 
%Nach Gleichung \eqref{eq:plaus_test} werden alle Tropfen, die die Gleichung nicht erfüllen, von der Messung ausgeschlossen.
Es werden Tropfen aussortiert, deren relativer Fehler von $q_\mathup{C}$ größer als $20\%$ ist. Für die tatsächliche Berechnung von $e$ können nur wenige Daten verwendet werden, deren Fehler sich auf $e$ und $N_\mathup{A}$ auswirken. Um genauere Ergebnnisse zu erhalten müsste ein großer Satz an Daten aufgenommen werden. Des Weiteren wäre ein besseres Ergbnis zu erzielen, wenn vor der Auswertung alle Tröpfchen, deren berechnete Ladung kleiner als die wirkliche Elementarladung ist, ebenfalls aussortiert werden.
