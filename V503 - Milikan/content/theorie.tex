\section*{Ziel}

Es wird die Elementarladung $e$ bestimmt. 
Weiter wird mithilfe der Erkenntnisse die Avogadro-Konstante $N_0$ ermittelt.
\section{Theorie}
\label{sec:Theorie}

Bewegt sich ein Öltröpfen im luftgefüllten Raum, so erfährt es Auftrieb neben der Gravitation.
Die Bewegungsgleichung für ein solchen Tröpfen ist mit
\begin{equation}
	trol
	\label{eq:bewgl_1}
\end{equation}
gegeben.
Befindet sich auf dem Tröpfen, etwa ausgehend von triboelektrischen Effekten bei der Zerstäubung,
eine Ladung, so reagiert es auf anwesende elektrische Felder.
Im Folgenden wird ein homogenes elektrisches Feld $E$ angenommen, das in Fallrichtung des unbeeinflussten Tröpfchens ausgerichtet ist.%; es ist also kollinear zum Lot.
Bei bekanntem elektrischen Feld $E$, so muss in der Gleichung \ref{eq:bewgl_1} die zusätzlich auftretende Kraft berücksichtigt werden.
Es gilt hierfür im Allgemeinen
\begin{equation}
	Foo baa baz,
	\label{eq:bewgl_2a}
\end{equation}
je nach Polung des von außen angelegten elektrischen Feldes $E$ ist die Bewegung des Tröpfchen in oder gegen ihre Fallrichtung.
Hieraus folgen für beide Fälle die Bewegungsgleichungen
\begin{align}
	Foo baa baz.
	\label{eq:bewgl_2}
\end{align}
Die Bewegungsgleichungen berücksichtigen Luftreibung.
Aufgrund dieser Reibung wird bei Erreichen der Grenzgeschwinidkeit $v_0$ ein Kräftegleichgewicht eingestellt,
die Geschwindigkeit $v_0$ ist daher konstant.
Ist das homogene elektrische Feld $E$ bekannt, so kann mithilfe der Grenzgeschwindigkeit $v_0$ Aussage getroffen werden,
welcher Radius $R$ das gemessene Tröpfchen aufweist und welche Ladung $q$ es trägt.
\begin{equation}
	trol
	\label{eq:radius}
\end{equation}
\begin{equation}
	trol
	\label{eq:ladung}
\end{equation}
Da die laminare Luftreibung $F_\text{\eta}$ nach Stokes für die hierauftretenden Tröpfchen nicht gilt, werden die Gleichungen \eqref{eq:radius} und \eqref{eq:ladung} über die Viskosität angepasst. 
Es gilt 
\begin{equation}
	trol
	\label{eq:cunningham}
\end{equation}
mit dem Cunningham-Korrekturterm $B=Foo$.