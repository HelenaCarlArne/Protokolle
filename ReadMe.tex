\documentclass[
  parskip=half,
  bibliography=totoc,     % Literatur im Inhaltsverzeichnis
  captions=tableheading,  % Tabellenüberschriften
  titlepage=firstiscover, % Titelseite ist Deckblatt
]{scrartcl}
%
% LaTeX2e korrigieren.
\usepackage{fixltx2e}
% Warnung, falls nochmal kompiliert werden muss
\usepackage[aux]{rerunfilecheck}
%
\usepackage{blindtext}
% deutsche Spracheinstellungen
\usepackage{polyglossia}
\setmainlanguage{german}
%
% unverzichtbare Mathe-Befehle
\usepackage{amsmath}
% viele Mathe-Symbole
\usepackage{amssymb}
% Erweiterungen für amsmath
\usepackage{mathtools}
%
% Fonteinstellungen
\usepackage{fontspec}
\defaultfontfeatures{Ligatures=TeX}
%
\usepackage[
  math-style=ISO,    % \
  bold-style=ISO,    % |
  sans-style=italic, % | ISO-Standard folgen
  nabla=upright,     % |
  partial=upright,   % /
]{unicode-math}
%
% Warnung! Bei Aktivierung der alternativen mathfonts (nächsten drei Befehle) 
% könnte die math-Umgebung nicht mehr funktionieren. -Arne
\setmathfont{Latin Modern Math}
\setmathfont[range={\mathscr, \mathbfscr}]{XITS Math}
\setmathfont[range=\coloneq]{XITS Math}
\setmathfont[range=\propto]{XITS Math}
% make bar horizontal, use \hslash for slashed h
\let\hbar\relax
\DeclareMathSymbol{\hbar}{\mathord}{AMSb}{"7E}
\DeclareMathSymbol{ℏ}{\mathord}{AMSb}{"7E}
%
% richtige Anführungszeichen
\usepackage[autostyle]{csquotes}
%
% Zahlen und Einheiten
\usepackage[
  locale=DE,                   % deutsche Einstellungen
  separate-uncertainty=true,   % Immer Fehler mit \pm
  per-mode=symbol-or-fraction, % m/s im Text, sonst Brüche
]{siunitx}
%für schöne Mengenangaben
\usepackage{dsfont}
% chemische Formeln
\usepackage[version=3]{mhchem}
%
% schöne Brüche im Text
\usepackage{xfrac}
%
% Floats innerhalb einer Section halten
\usepackage[section, below]{placeins}
% Captions schöner machen.
\usepackage[
  labelfont=bf,        % Tabelle x: Abbildung y: ist jetzt fett
  font=small,          % Schrift etwas kleiner als Dokument
  width=0.9\textwidth, % maximale Breite einer Caption schmaler
]{caption}
% subfigure, subtable, subref
\usepackage{subcaption}
%
% Grafiken können eingebunden werden
\usepackage{graphicx}
% größere Variation von Dateinamen möglich
\usepackage{grffile}
%
% Standardplatzierung für Floats einstellen
\usepackage{float}
\floatplacement{figure}{h}
\floatplacement{table}{h}
%
% schöne Tabellen
\usepackage{booktabs}
%
% Seite drehen für breite Tabellen
\usepackage{pdflscape}
%
% Literaturverzeichnis
\usepackage[style=numeric,backend=biber]{biblatex}
% Quellendatenbank
\addbibresource{lit.bib}
\addbibresource{programme.bib}
%
% Hyperlinks im Dokument
\usepackage[
  unicode,
  pdfusetitle,    % Titel, Autoren und Datum als PDF-Attribute
  pdfcreator={},  % PDF-Attribute säubern
  pdfproducer={}, % "
]{hyperref}
% erweiterte Bookmarks im PDF
\usepackage{bookmark}
%
% Trennung von Wörtern mit Strichen
\usepackage[shortcuts]{extdash}
%
\author{
  Helena Nawrath
  \texorpdfstring{
    \\
    \href{mailto:helena.nawrath@tu-dortmund.de}{helena.nawrath@tu-dortmund.de}
  }{}%
  \texorpdfstring{\and}{, }
  Carl Arne Thomann
  \texorpdfstring{
    \\
    \href{mailto:arnethomann@me.com}{arnethomann@me.com}
  }{}
}
\publishers{TU Dortmund – Fakultät Physik}
\usepackage{enumitem}
\usepackage{wasysym}
\usepackage{blindtext}
\usepackage{framed}
\usepackage{float}
\floatstyle{boxed}
\restylefloat{figure}


\title{My \emph{git}-Guide}
\begin{document}
\maketitle
\tableofcontents
\newpage
\section{Einleitung}
An dieser Stelle möchte noch eine töfte Anleitung stehen \smiley
\newpage

\section{Grundlagen}									%Die erste Section
\begin{figure}
\centering
\caption{Die vier Instanzen}
\begin{minipage}{0.24\textwidth}
	\begin{framed} 		
	\centering 
	(a)\\
	\vspace{0,3cm}
	Verzeichnis \\
	\phantom{Repository}
	\vspace{1cm}
	\end{framed} 
\end{minipage}
\begin{minipage}{0.24\textwidth}
	\begin{framed} 		
	\centering 
	(b)\\
	\vspace{0,3cm}
	Index \\
	\phantom{Repository}
	\vspace{1cm}
	\end{framed} 
\end{minipage}
\begin{minipage}{0.24\textwidth}
	\begin{framed} 		
	\centering 
	(c)\\
	\vspace{0,3cm}
	Lokales Repository
	\vspace{1cm} 
	\end{framed} 
\end{minipage}
\begin{minipage}{0.24\textwidth}
	\begin{framed} 		
	\centering 
	(d)\\
	\vspace{0,3cm}
	Entferntes Repository 
	\vspace{1cm}
	\end{framed} 
\end{minipage}
\end{figure}

Das

\newpage

\section{Erstellen von entfernten Repositorys}				%Die zweite Section
\begin{enumerate}
	\item{Erstelle einen Ordner mit dem Namen Deines (zukünftigen) Repositorys.}
	\item{Öffne den Ordner in der Kommandozeile}
	\item{Melde bei GitHub das Repos an}
	\item{Verwende in der Kommandozeile die Befehle, die auf GitHub angezeigt werden:}
	\begin{enumerate}
		\item{\emph{touch} README.md}
		\item{\emph{git init}}
		\item{\emph{git add} README.md}
		\item{\emph{git add} \{andere Dateien\}}
		\item{\emph{git commit -m} "first commit"}
		\item{\emph{git remote add origin} \{Adresse von GitHub.com\}}
		\item{\emph{git push -u origin master}}
	\end{enumerate}
\end{enumerate}
\hrule 
\vspace{3pt}
Dabei stehen die Befehle  für folgende Aktivitäten:
\begin{description}[leftmargin=\parindent,labelsep=8pt]
	\item[a] Erstellt eine leere Datei, die als erste Datei ins entfernte Repository geladen wird.
	\item[b] Initialisiert das \emph{git}-System
	\item[c] Fügt die Dateien zu dem Staging hinzu, ab jetzt kann git darauf Einfluss nehmen. \\ \phantom{} Vor dem "Adden" war das nicht möglich
	\item[d] Veranlasst das Übernehmen der Änderungen zu einem Datencommit,\\ \phantom{} welcher bereit ist, auf den Server gelegt zu werden.
	\item[e]  Hiermit wird das Ur-Repository vom Rechner auf den Server geladen.\\ \phantom{} Damit ist git in gewohnter Weise betriebsbereit.
	\item[f] Das ist der erste Push-Befehl. Alle folgenden sind \emph{git push origin master} oder \emph{git push}
\end{description}
\hrule 
\vspace{3pt}
\end{document}