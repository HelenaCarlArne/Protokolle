
\begin{table}
	\centering
	\sisetup{table-format=2.3}
	\begin{tabular}{S[table-format=1.1] S[table-format=3.0] S[table-format=1.2] S[table-format=1.2] }
	\toprule
	\multicolumn{2}{c}{Rechteckspannung} &\multicolumn{2}{c}{Sinusspannung} \\
	{$I_\mathup{eff}/\:\si{\milli{\ampere}}$} & {$U_\mathup{k}/\:\si\volt$} &{$I_\mathup{eff}/\:\si{\milli{\ampere}}$} & {$U_\mathup{k}/\:\si{\volt}$}\\
	\midrule
 1.0 & 260 & 0.72 & 0.20\\
 1.1 & 255 & 0.38 & 0.25\\
 1.2 & 245 & 0.36 & 0.27\\
 1.4 & 235 & 0.32 & 0.30\\
 1.6 & 230 & 0.27 & 0.33\\
 1.9 & 215 & 0.22 & 0.35\\
 2.2 & 190 & 0.14 & 0.40\\
 2.7 & 160 & 0.11 & 0.42\\
 3.5 & 115 & 0.10 & 0.44\\
 4.1 &  80 & 0.09 & 0.45\\
\\
	\bottomrule
	\end{tabular}
	\caption{Messdaten mit Rechteck- und Sinusspannung vom RC-Generator.}
	\label{tab:Recht_Sin}
\end{table}
