\section{Auswertung}
\label{sec:Auswertung}

\subsection{Umrechnung der Messdaten} % (fold)
\label{sec:umrechnung}
Die im Appendix vorliegenden Diagramme lassen zu, Messdaten mit hoher Genauigkeit in ihren Koordinaten zu bestimmen.
Die x-Koordinaten der auszuwertenden Punkte werden in Bezug auf den eingezeichneten Koordinatenursprung bestimmt und für die Auswertung anhand der Gleichung 
\begin{equation}
	U=a_\text{lin}\cdot x+b_\text{lin}
	\label{eq:umrechnung}
\end{equation}
in eine Spannung $U$ umgerechnet.
Die Einheiten von $a_\text{lin}$ und $b_\text{lin}$ sind gemäß der Größenumrechnung passend gewählt, 
es gelten $[a_\text{lin}]=\si{\volt\per\centi\meter}$ und $[b_\text{lin}]=\si{\volt}$.
Diese fehlerbehaftete Umrechnungsparameter stammen von einer linearen Regression mehrer fester Skalenpunkte, für welche die Spannung bekannt ist.

\begin{table}
	\centering
	\sisetup{table-format = -1.3(1)}
		\begin{tabular}{l S S}
		\toprule
		{Plot Nr.}&\multicolumn{2}{c}{Umrechnungsparameter}\\
		&{$a_\text{lin}$/$\:(\si{\volt\per\centi\meter})$} & {$b_\text{lin}$/$\:\si{\volt}$}\\
		\midrule
		1&		0.403(4)& 	-0.12(6)\\
		2&		0.405(4)& 	0.015(6)\\
		3&		2.89(2)& 	-3.9(2)	\\
		4& 		2.40(3)& 	-1.1(5)	\\
		\end{tabular}
	\caption{Parameter der linearen Regression der Diagramm-Skalen für die Umrechnung der Messdaten; 
	Umrechnung Diagrammlänge zu Spannung.}
	\label{tab:umrechnung}
\end{table}

%Ekalt
%0.4036+/-0.0044 -0.1246+/-0.0579
%Ewarm
%0.4047+/-0.0040 0.0154+/-0.0594
%Ion
%2.8874+/-0.0175 -3.8736+/-0.2034
%FH
%2.4000+/-0.0277 -1.1260+/-0.4554

Analoge Umrechnung ist für die y-Koordinaten in Stromstärken möglich.
\begin{equation}
	I=a'_\text{lin}\cdot x+b'_\text{lin}
	\label{eq:umrechnung}
\end{equation}
Es gelten $[a'_\text{lin}]=\si{\nano\ampere\per\centi\meter}$ und $[b'_\text{lin}]=\si{\nano\ampere}$.


\begin{table}
	\centering
	\sisetup{table-format=1.3}
		\begin{tabular}{l S[table-format=1.3] S[table-format=1.3]}
		\toprule
		{Plot Nr.}&\multicolumn{2}{c}{Umrechnungsparameter}\\
		&{$a_\text{lin}$/$\:(\si{\nano\ampere\per\centi\meter})$} & {$b_\text{lin}$/$\:\si{\nano\ampere}$}\\
		\midrule
		1&	0.41& 1.00\\
		2&	1.77& 2.20\\
		\end{tabular}
	\caption{Parameter der linearen Regression der Diagramm-Skalen für die Umrechnung der Messdaten; 
	Umrechnung Diagrammlänge zu Stromstärke.}
	\label{tab:umrechnung}
\end{table}
%Ekalt0.41051.0000
%Ewarm1.77422.2000

% section umrechnung (end)
\subsection{Energieverteilung der Elektronen} % (fold)
\label{sec:energiespektren}

% section aufnahme_der_energiespektren (end)

\subsection{Ionisierungskurven von Quecksilber} % (fold)
\label{sec:ion}

% section ion (end)

\subsection{Franck--Hertz-Kurven} % (fold)
\label{sec:fhk}

% section franck_hertz_kurven (end)