\documentclass[
    parskip=half,
    bibliography=totoc,     % Literatur im Inhaltsverzeichnis
    captions=tableheading,  % Tabellen�berschriften
    titlepage=firstiscover, % Titelseite ist Deckblatt
    ]{scrartcl}
    
\usepackage[top=2cm, bottom=4cm, left=2cm, right=2cm]{geometry}
\usepackage{color}
\usepackage[usenames,dvipsnames]{xcolor}
\definecolor{light-red}{HTML}{FFBABA}

% LaTeX2e korrigieren.
\usepackage{fixltx2e}

% Warnung, falls nochmal kompiliert werden muss
\usepackage[aux]{rerunfilecheck}

% Deutsche Spracheinstellungen
\usepackage{polyglossia}
\setmainlanguage{german}

% Unverzichtbare Mathe-Befehle
\usepackage{amsmath}

% Viele Mathe-Symbole
\usepackage{amssymb}

% Erweiterungen f�r amsmath
\usepackage{mathtools}

% Fonteinstellungen
\usepackage{fontspec}
\defaultfontfeatures{Ligatures=TeX}

\usepackage[
    math-style=ISO,    % \
    bold-style=ISO,    % |
    sans-style=italic, % | ISO-Standard folgen
    nabla=upright,     % |
    partial=upright,   % /
    ]{unicode-math}

\setmathfont{Latin Modern Math}
\setmathfont[range={\mathscr, \mathbfscr}]{XITS Math}
\setmathfont[range=\coloneq]{XITS Math}
\setmathfont[range=\propto]{XITS Math}

% Das hquer-Symbol versch�nern
\let\hbar\relax
\DeclareMathSymbol{\hbar}{\mathord}{AMSb}{"7E}
\DeclareMathSymbol{?}{\mathord}{AMSb}{"7E}

% Richtige Anf�hrungszeichen
\usepackage[autostyle]{csquotes}

% Zahlen und Einheiten
\usepackage[
  locale=DE,                   % Deutsche Einstellungen
  separate-uncertainty=true,   % Immer Fehler mit \pm
  per-mode=symbol-or-fraction, % m/s im Text, sonst Br�che
]{siunitx}

% Chemische Formeln
\usepackage[version=3]{mhchem}

% Sch�ne Br�che im Text
\usepackage{xfrac}

% Floats innerhalb einer Section halten
\usepackage[section, below]{placeins}

% Captions sch�ner machen.
\usepackage[
    labelfont=bf,        % Tabelle x: Abbildung y: ist jetzt fett
    font=small,          % Schrift etwas kleiner als Dokument
    width=0.9\textwidth, % Maximale Breite einer Caption schmaler
    ]{caption}

% Subfigure, subtable, subref
\usepackage{subcaption}

% Grafiken einbinden
\usepackage{graphicx}

% Gr��ere Variation von Dateinamen m�glich
\usepackage{grffile}

% Standardplatzierung f�r Floats einstellen
\usepackage{float}
\floatplacement{figure}{htbp}
\floatplacement{table}{htbp}

% Sch�ne Tabellen
\usepackage{booktabs}

% Seite drehen f�r breite Tabellen
\usepackage{pdflscape}

% Literaturverzeichnis
\usepackage{biblatex}

% Quellendatenbank
\addbibresource{lit.bib}
\addbibresource{programme.bib}

% Hyperlinks im Dokument
\usepackage[
    unicode,
    pdfusetitle,    % Titel, Autoren und Datum als PDF-Attribute
    pdfcreator={},  % PDF-Attribute s�ubern
    pdfproducer={}, % "
    ]{hyperref}
    
% Erweiterte Bookmarks im PDF
\usepackage{bookmark}

% Trennung von W�rtern mit Strichen
\usepackage[shortcuts]{extdash}

% Blindtext erzeugen
\usepackage{blindtext}

% Support f�r mdframed
\usepackage{kvoptions}
\usepackage{xparse}
\usepackage{etoolbox}
\usepackage{tikz}

% Sch�ne mehrseitige Rahmen um Text erzeugen
\usepackage{mdframed}
\mdfsetup{skipabove=\topskip,skipbelow=\topskip}
% Neue Mathematikbefehle
\DeclareMathOperator{\rank}{rang}
\DeclareMathOperator{\cond}{cond}
\newcommand{\up}{\mathup}
\newcommand{\R}{\mathbb{R}}
\newcommand{\N}{\mathbb{N}}
\newcommand{\upD}{\mathup{\Delta}}

% Wichtiger Befehl zur Erstellung einer blauen Box um einen Text
\newcommand\mybox[2][]{\tikz[overlay]\node[fill=blue!20,inner sep=4pt, anchor=text, rectangle, rounded corners=1mm,#1] {#2};\phantom{#2}}
\newenvironment{Versuch}[1]{
    \mdfsetup{
        innertopmargin=8pt,
        linecolor=blue!20,
        linewidth=2pt,
        topline=true,
        backgroundcolor=blue!20
        }
    \begin{mdframed}
    \Large{\textbf{#1}}
    \end{mdframed}}
    {}

\newenvironment{Stichworte}{
    \mdfsetup{
        frametitle={\mybox[fill=blue!20]{\Large{Stichworte}}},
        frametitleaboveskip=0pt,
        innertopmargin=10pt,
        linecolor=blue!20,
        linewidth=2pt,
        topline=true,
        backgroundcolor=white
        }
    \begin{mdframed}}
    {\end{mdframed}}    
    
\newenvironment{Zielsetzung}{
    \mdfsetup{
        frametitle={\mybox[fill=blue!20]{\Large{Zielsetzung}}},
        frametitleaboveskip=0pt,
        innertopmargin=10pt,
        linecolor=blue!20,
        linewidth=2pt,
        topline=true,
        backgroundcolor=white
        }
    \begin{mdframed}}
    {\end{mdframed}}

\newenvironment{Theorie}{
    \mdfsetup{
        frametitle={\mybox[fill=blue!20]{\Large{Theorie}}},
        frametitleaboveskip=0pt,
        innertopmargin=10pt,
        linecolor=blue!20,
        linewidth=2pt,
        topline=true,
        backgroundcolor=white
        }
    \begin{mdframed}}
    {\end{mdframed}}

\newenvironment{Durchführung}{
    \mdfsetup{
        frametitle={\mybox[fill=blue!20]{\Large{Durchführung}}},
        frametitleaboveskip=0pt,
        innertopmargin=10pt,
        linecolor=blue!20,
        linewidth=2pt,
        topline=true,
        backgroundcolor=white
        }
    \begin{mdframed}}
    {\end{mdframed}}
    
\newenvironment{Auswertung}{
    \mdfsetup{
        frametitle={\mybox[fill=blue!20]{\Large{Auswertung}}},
        frametitleaboveskip=0pt,
        innertopmargin=10pt,
        linecolor=blue!20,
        linewidth=2pt,
        topline=true,
        backgroundcolor=white
        }
    \begin{mdframed}}
    {\end{mdframed}}
    
\newenvironment{Diskussion}{
    \mdfsetup{
        frametitle={\mybox[fill=blue!20]{\Large{Diskussion}}},
        frametitleaboveskip=0pt,
        innertopmargin=10pt,
        linecolor=blue!20,
        linewidth=2pt,
        topline=true,
        backgroundcolor=white
        }
    \begin{mdframed}}
    {\end{mdframed}}

\newenvironment{Merke}{
    \mdfsetup{
        frametitle={\mybox[fill=red]{\Large{Merke}}},
        frametitleaboveskip=0pt,
        innertopmargin=5pt,
        linecolor=red,
        linewidth=1pt,
        topline=true,
        backgroundcolor=light-red
        }
    \begin{mdframed}}
    {\end{mdframed}}

\newenvironment{Appendix}{
    \mdfsetup{
        frametitle={\mybox[fill=blue!20]{\Large{Appendix}}},
        frametitleaboveskip=0pt,
        innertopmargin=10pt,
        linecolor=blue!20,
        linewidth=0pt,
        topline=false,
        backgroundcolor=white
        }
    \begin{mdframed}}
    {\end{mdframed}}
    
\begin{document}

    \begin{Versuch}{V704: Absorption von Gamma- und Betastrahlung}
    
	\begin{Zielsetzung}
		Nachdem eine Vielzahl von Absorptionsmechanismen von Strahlung in der Theorie vorgestellt wurde, wird versucht, anhand der Durchdringtiefe von Gammastrahlung auf den Mechanismus zu schließen.
		Für Betastrahlung wird untersucht, welche die maximale kinetische Energie ist und wie tief Betastrahlung eindringt.
	\end{Zielsetzung}

        \begin{Theorie}
            \emph{Wirkungsquerschnitt} $\sigma$:\\
            $\sigma$ ist eine Fläche bestimmter Größe, die dem Absorber spezifisch zugeordnet wird.
			Dabei wird angenommen, dass der Absorber punktuell aus diesen Flächen $\sigma$ besteht
			und nur an diesen Stellen zum Aufhalten der Strahlung in der Lage ist. 
			Trifft in diesem Modell ein Photon oder ein Elektron auf eine dieser Flächen, wird es absorbiert; 
			andernfalls durchdringt die Strahlung die Materie unbeeinflusst.

			\emph{Absorptionsgesetz}:\\
			Mithilfe des Wirkungsquerschnittes $\sigma$ wird eine Wahrscheinlichkeit formuliert, mit welcher ein Teilchen/Strahlung absorbiert wird.
			$n$ ist die Anzahl der Absorper in einem kleinen Volumen.
			Die Näherung dabei lautet, dass es sich um Materie geringer Dicke d$x$ handelt, 
			\begin{equation}
				\mathup{d}N=-N(x)  \underbrace{n \sigma \mathup{d}x}_{\text{W'keit}},
				\label{eq:Absorptionsgesetz_Vorstufe}
			\end{equation}
			sodass auf die Gesamtdicke integriert werden kann.
			Es ergibt sich
			\begin{equation}
				N(D)=N_0 \exp(-n \sigma D)=N_0 \exp(-\mu D)
				\label{eq:Absorptionsgesetz}
			\end{equation}
			mit dem Absorptionskoeffizienten $\mu = n\sigma$

			\emph{Spezifische Größen}:\\
			Nach der spezifischen Dicke
			\begin{equation}
				D_{\sfrac{1}{2}}=\frac{\ln(2)}{\mu},
			\end{equation}
			halbiert sich im Schnitt die Intensität der Strahlung.
			Die Anzahl $n$ der Absorber pro Volumeneinheit wird mit
			\begin{equation}
				n=\frac{zN_\text{L}}{V_\text{Mol}}=\frac{zN_\text{L}\rho}{M}
			\end{equation}
			abgeschätzt, wodurch 
			\begin{equation}
				\sigma=\frac{M\mu}{zN_\text{L}\rho}
				\label{eq:Wirkungsquerschnitt}
			\end{equation}
			gilt.
			Darin: $z$ die Ordnungszahl des Absorberatoms, $N_\text{L}$ die Loschmidtsche Zahl, $V_\text{Mol}$ das Molvolumen, $M$ das Molekulargewicht und $\rho$ Dichte.

			\emph{Wechselwirkung von Gammastrahlung}:
			\begin{itemize}
				\item{innerer Photoeffekt an Elektronen}

				Das gestoßene Elektron wird dem Atom herausgelöst und besitzt die kinetische Energie $E_e=h\nu-E_\text{B}$ mit der zu überwindenden Elektronen-Bindungsenergie $E_\text{B}$. 
				Dies tritt bei typischen $\gamma$-Energien nur bei inneren Elektronen auf. 
				Die entstehenden Lücken werden durch äußere Elektronen unter Emission von Röntgenstrahlung und Auger-Elektronen aufgefüllt.
				\item{\emph{Compton-Effekt}}

				Das Photon wird im Zuge des Compton-Effekts an einem freien Elektron oder Elektron in der äußeren Hülle gestreut und gibt seine Energie teilweise ab. 
				Dies ruft eine Richtungs- und Impulsänderung hervor.
				Mit einer komplizierten Formel für den Wirkungsquerschnitt $\sigma$ ergibt sich
				\begin{equation}
					\mu_\text{C}=n\sigma_\text{C}(\epsilon)=\frac{Z N_L \rho}{M}\sigma_\text{C}(\epsilon).
				\label{eq:mu_c}
				\end{equation}
				\item{\emph{Paarerzeugung}}

				Ist die Energie des $\gamma$-Quants sehr groß, kann es unter Erzeugung von Elektron und Positron zur Paarbildung kommen.
				Forderungen: $E_\text{Gamma}\ge 2m_\text{Elektron,0}\cdot c^2$ und ein Teil des Impulses muss von einem Stoßpartner übernommen.
				Hierzu dienen die Atomkerne des Absorbermaterials, in  deren Coulomb-Feldern  sich die  Teilchenpaare bilden.
			\end{itemize}
			In Reihenfolge treten bei ansteigender Gamma-Energie Photoeffekt, Compton-Effekt und Paarerzeugung auf.

		\emph{Wechselwirkung von Betastrahlung}:
		Kein geschlossenes Absorptionsgesetz,\eqref{eq:Absorptionsgesetz} gilt näherungsweise

		\begin{itemize}
			\item{\emph{Elastische Streuung am Atomkern:}}
				Wesentliche Ablenkung, aber keine Abbremsung:
				Durch mehrfache Ablenkung ist die Länge der Bahn deutlich größer als der ungehinderte, direkte Weg.
				Das magnetische Moment der Elektronen wird nicht betrachtet.

			\item{\emph{Inelastische Streuung an dem Atomkern}}

				Bewegen sich geladene Teilchen im Coulomb-Feld von entgegengesetzt geladenen Teilchen, so werden sie beschleunigt, 
				wobei Energie in Form von Strahlung abgegeben werden muss.

			\item{\emph{Inelastische Streuung an den Elektronen des Absorbermaterials}}

				Die inelastische Streuung an Elektronen führt (mehrfach) zur Anregung oder zur Ionisierung von Absorberatomen.
				Bei geringer Energie, beispielsweise $E\approx\SI{150}{\kilo\electronvolt}$ bei Aluminium, ist aber damit zu rechnen, 
				dass die $\beta^-$-Strahlung bereits ab einer Dicke von $\SI{0.15}{\milli\meter}$ vollständig abgebremst wird.
		\end{itemize}
		Die maximale Energie der Betastrahlung ist empirisch über die maximale Reichweite mit
		\begin{equation}
			E(D)=1.92\sqrt{\rho^2D^2+0.22\rho D} \quad\si{\mega\electronvolt}
		\end{equation}
        \end{Theorie}
        
        \begin{Durchführung}
            Die Hintergrundstrahlung wird bestimmt.

            \emph{Gamma:} Anschließend wird bei verschiedenen Blendenmaterialien die Anzahl der ankommenden Strahlungsteilchen pro Zeiteinheit ermittelt und davon Offset abgezogen.
            Das Ergebnis wird durch halblogarithmischen Auftrag linearisiert und damit gemäß \eqref{eq:Absorptionsgesetz} $N_0$ und $\mu$ bestimmt.
            Durch Vergleich mit den Theoriewerten kann auf den Gamma-Absorptionsmechanismus geschlossen werden.

            \emph{Beta:} Die maximale Energie und Reichweite der Teilchen wird bestimmt, indem Aluminiumblenden verschiedener Dicke vor den Detektor gesetzt werden.

        \end{Durchführung}        
        
        \begin{Auswertung}
        	\emph{Gamma:} Wegen der geringen Energien ist Paarerzeugung grundsätzlich ausgeschlossen.\\
        	Eisenblende: Da die Abweichung des Ergebnis von der Compton-Theoriewert hoch ist,
        	kann der Compton-Effekt ausgeschlossen und Photo-Effekt als maßgeblicher Effekt erkannt werden.\\
        	Bleiblende: Da die Abweichung des Ergebnis für Blei von der Compton-Theoriewert gering ist,
        	kann der Photo-Effekt vernachlässigt werden.

        	\emph{Beta:} Die Reichweite kann auf $\SI{2,9}{\centi\meter}$ abgeschätzt werden. 
        \end{Auswertung}

        \begin{Diskussion}
        	Die Messungen werden präziser, je länger gemessen wird.
        	Es kann bereits bei Messintervallen von $\SI{600}{\second}$ verlässliche Aussage über den Gamma-Absorptionsmechanismus sowie über die Beta-Reichweite und -Energie getroffen werden.
        	Die verwendete Methode ist experimentell gut geeignet.
        \end{Diskussion}
    \end{Versuch}
    
\end{document}