\documentclass[
    parskip=half,
    bibliography=totoc,     % Literatur im Inhaltsverzeichnis
    captions=tableheading,  % Tabellen�berschriften
    titlepage=firstiscover, % Titelseite ist Deckblatt
    ]{scrartcl}
    
\usepackage[top=2cm, bottom=4cm, left=2cm, right=2cm]{geometry}
\usepackage{color}
\usepackage[usenames,dvipsnames]{xcolor}
\definecolor{light-red}{HTML}{FFBABA}

% LaTeX2e korrigieren.
\usepackage{fixltx2e}

% Warnung, falls nochmal kompiliert werden muss
\usepackage[aux]{rerunfilecheck}

% Deutsche Spracheinstellungen
\usepackage{polyglossia}
\setmainlanguage{german}

% Unverzichtbare Mathe-Befehle
\usepackage{amsmath}

% Viele Mathe-Symbole
\usepackage{amssymb}

% Erweiterungen f�r amsmath
\usepackage{mathtools}

% Fonteinstellungen
\usepackage{fontspec}
\defaultfontfeatures{Ligatures=TeX}

\usepackage[
    math-style=ISO,    % \
    bold-style=ISO,    % |
    sans-style=italic, % | ISO-Standard folgen
    nabla=upright,     % |
    partial=upright,   % /
    ]{unicode-math}

\setmathfont{Latin Modern Math}
\setmathfont[range={\mathscr, \mathbfscr}]{XITS Math}
\setmathfont[range=\coloneq]{XITS Math}
\setmathfont[range=\propto]{XITS Math}

% Das hquer-Symbol versch�nern
\let\hbar\relax
\DeclareMathSymbol{\hbar}{\mathord}{AMSb}{"7E}
\DeclareMathSymbol{?}{\mathord}{AMSb}{"7E}

% Richtige Anf�hrungszeichen
\usepackage[autostyle]{csquotes}

% Zahlen und Einheiten
\usepackage[
  locale=DE,                   % Deutsche Einstellungen
  separate-uncertainty=true,   % Immer Fehler mit \pm
  per-mode=symbol-or-fraction, % m/s im Text, sonst Br�che
]{siunitx}

% Chemische Formeln
\usepackage[version=3]{mhchem}

% Sch�ne Br�che im Text
\usepackage{xfrac}

% Floats innerhalb einer Section halten
\usepackage[section, below]{placeins}

% Captions sch�ner machen.
\usepackage[
    labelfont=bf,        % Tabelle x: Abbildung y: ist jetzt fett
    font=small,          % Schrift etwas kleiner als Dokument
    width=0.9\textwidth, % Maximale Breite einer Caption schmaler
    ]{caption}

% Subfigure, subtable, subref
\usepackage{subcaption}

% Grafiken einbinden
\usepackage{graphicx}

% Gr��ere Variation von Dateinamen m�glich
\usepackage{grffile}

% Standardplatzierung f�r Floats einstellen
\usepackage{float}
\floatplacement{figure}{htbp}
\floatplacement{table}{htbp}

% Sch�ne Tabellen
\usepackage{booktabs}

% Seite drehen f�r breite Tabellen
\usepackage{pdflscape}

% Literaturverzeichnis
\usepackage{biblatex}

% Quellendatenbank
\addbibresource{lit.bib}
\addbibresource{programme.bib}

% Hyperlinks im Dokument
\usepackage[
    unicode,
    pdfusetitle,    % Titel, Autoren und Datum als PDF-Attribute
    pdfcreator={},  % PDF-Attribute s�ubern
    pdfproducer={}, % "
    ]{hyperref}
    
% Erweiterte Bookmarks im PDF
\usepackage{bookmark}

% Trennung von W�rtern mit Strichen
\usepackage[shortcuts]{extdash}

% Blindtext erzeugen
\usepackage{blindtext}

% Support f�r mdframed
\usepackage{kvoptions}
\usepackage{xparse}
\usepackage{etoolbox}
\usepackage{tikz}

% Sch�ne mehrseitige Rahmen um Text erzeugen
\usepackage{mdframed}
\mdfsetup{skipabove=\topskip,skipbelow=\topskip}
% Neue Mathematikbefehle
\DeclareMathOperator{\rank}{rang}
\DeclareMathOperator{\cond}{cond}
\newcommand{\up}{\mathup}
\newcommand{\R}{\mathbb{R}}
\newcommand{\N}{\mathbb{N}}
\newcommand{\upD}{\mathup{\Delta}}

% Wichtiger Befehl zur Erstellung einer blauen Box um einen Text
\newcommand\mybox[2][]{\tikz[overlay]\node[fill=blue!20,inner sep=4pt, anchor=text, rectangle, rounded corners=1mm,#1] {#2};\phantom{#2}}
\newenvironment{Versuch}[1]{
    \mdfsetup{
        innertopmargin=8pt,
        linecolor=blue!20,
        linewidth=2pt,
        topline=true,
        backgroundcolor=blue!20
        }
    \begin{mdframed}
    \Large{\textbf{#1}}
    \end{mdframed}}
    {}

\newenvironment{Stichworte}{
    \mdfsetup{
        frametitle={\mybox[fill=blue!20]{\Large{Stichworte}}},
        frametitleaboveskip=0pt,
        innertopmargin=10pt,
        linecolor=blue!20,
        linewidth=2pt,
        topline=true,
        backgroundcolor=white
        }
    \begin{mdframed}}
    {\end{mdframed}}    
    
\newenvironment{Zielsetzung}{
    \mdfsetup{
        frametitle={\mybox[fill=blue!20]{\Large{Zielsetzung}}},
        frametitleaboveskip=0pt,
        innertopmargin=10pt,
        linecolor=blue!20,
        linewidth=2pt,
        topline=true,
        backgroundcolor=white
        }
    \begin{mdframed}}
    {\end{mdframed}}

\newenvironment{Theorie}{
    \mdfsetup{
        frametitle={\mybox[fill=blue!20]{\Large{Theorie}}},
        frametitleaboveskip=0pt,
        innertopmargin=10pt,
        linecolor=blue!20,
        linewidth=2pt,
        topline=true,
        backgroundcolor=white
        }
    \begin{mdframed}}
    {\end{mdframed}}

\newenvironment{Durchführung}{
    \mdfsetup{
        frametitle={\mybox[fill=blue!20]{\Large{Durchführung}}},
        frametitleaboveskip=0pt,
        innertopmargin=10pt,
        linecolor=blue!20,
        linewidth=2pt,
        topline=true,
        backgroundcolor=white
        }
    \begin{mdframed}}
    {\end{mdframed}}
    
\newenvironment{Auswertung}{
    \mdfsetup{
        frametitle={\mybox[fill=blue!20]{\Large{Auswertung}}},
        frametitleaboveskip=0pt,
        innertopmargin=10pt,
        linecolor=blue!20,
        linewidth=2pt,
        topline=true,
        backgroundcolor=white
        }
    \begin{mdframed}}
    {\end{mdframed}}
    
\newenvironment{Diskussion}{
    \mdfsetup{
        frametitle={\mybox[fill=blue!20]{\Large{Diskussion}}},
        frametitleaboveskip=0pt,
        innertopmargin=10pt,
        linecolor=blue!20,
        linewidth=2pt,
        topline=true,
        backgroundcolor=white
        }
    \begin{mdframed}}
    {\end{mdframed}}

\newenvironment{Merke}{
    \mdfsetup{
        frametitle={\mybox[fill=red]{\Large{Merke}}},
        frametitleaboveskip=0pt,
        innertopmargin=5pt,
        linecolor=red,
        linewidth=1pt,
        topline=true,
        backgroundcolor=light-red
        }
    \begin{mdframed}}
    {\end{mdframed}}

\newenvironment{Appendix}{
    \mdfsetup{
        frametitle={\mybox[fill=blue!20]{\Large{Appendix}}},
        frametitleaboveskip=0pt,
        innertopmargin=10pt,
        linecolor=blue!20,
        linewidth=0pt,
        topline=false,
        backgroundcolor=white
        }
    \begin{mdframed}}
    {\end{mdframed}}
    
\begin{document}

    \begin{Versuch}{V204: Wärmeleitung in Metallen}
    	\begin{Stichworte}
    		Wärmeleitungs- und Diffusionsgleichung,
    		Temperaturausgleich,
    		Wärmekapazität,
    		statische und dynamische Messung.

    		Wärmeleitfähigkeit,
    		Temperaturleitfähigkeit,
    		der Unteschied der beiden Leitfähigkeiten.
    	\end{Stichworte}

        \begin{Zielsetzung}
            Eine der Wärmetransportmöglichkeiten, die Wärmeleitung, wird untersucht.
            Hierzu werden verschiedene Metalle verschiedener Abmessung erhitzt und die Temperaturentwicklung gemessen.
        \end{Zielsetzung}

        \begin{Theorie}
            Ist in einem abgeschlossenen System ein Temperaturunterschied vorhanden, 
            findet Wärmetransport statt, 
            um ein Temperaturgleichgewicht zu erreichen.

            \begin{center}
            \begin{description}
            	\item [Konvektion] Fluide verschiedener Temperaturen vermischen mitunter chaotisch, Videostichwort: Konvektionsströme.
            	\item [Wärmestrahlung] Abgabe von Wärme via em-Wellen, nicht an Materie gebunden.
            	\item [Wärmeleitung]
            \end{description}
            \end{center}
            Beim Übergang der Wärmeenergie werden quantisierte Schwingungen innerhalb der Gitterstrukturen des Metalls
            \begin{equation}
				\label{waermemenge}
				\mathup{d}Q=-\kappa A\frac{\partial{T}}{\partial{x}}\mathup{d}t
			\end{equation}
			übertragen, die Materialkonstante $\kappa$ ist die Wärmeleitfähigkeit.
			Mithilfe der (geforderten) Kontinuität wird die Wärmeleitungsgleichung 
			\begin{equation}
				\label{waermeleitungsgleichung}
				\frac{\partial{T}}{\partial{t}} =  \underbrace{\frac{\kappa}{c\rho}}_{\sigma_\text{T}}\frac{\partial^2{T}}{\partial{x^2}}
			\end{equation}
			formuliert, in welcher die Konstanten
			als Temperaturleitfähigkeit $\sigma_\mathup{T}$ zusammengefasst werden.
			$c$ beschreibt dabei allerdings die spezifische Wärme(-kapazität), 
			die benötigte Energiemenge, 
			um $\SI{1}{\kilo\gram}$ um $\SI{1}{\kelvin}$ zu erwärmen.
			Werden Körper durch periodischen Temperaturwechsel geheizt und gekühlt, breiten sich in Innern Temperaturwellen der Form
			\begin{equation}
				\label{temperaturwelle}
				T(x,t)= \underbrace{\vphantom{\underbrace{\biggl(\biggr)}_{k}}T_\mathup{max}}_{\text{Amplitude}} \underbrace{\exp{\Biggl(-\underbrace{\sqrt{\frac{\omega \rho c}{2\kappa}}}_{k}x\Biggr)}}_{\text{Dämpfung}} \underbrace{\cos{\Biggl(\omega t - \underbrace{\sqrt{\frac{\omega \rho c}{2\kappa}}}_{k}x\Biggr)}}_{\text{Schwingung}}
			\end{equation}
			aus.
			Dies ist eine spezielle Lösung der Wärmeleitungsgleichung \eqref{waermeleitungsgleichung}.

			Die Dispersionsgleichung lautet
			\begin{equation}
				\label{dispersion}
				v=\frac{\omega}{k}=\frac{\omega}{\sqrt{\frac{\omega\rho c}{2\kappa}}}=\sqrt{\frac{2\kappa\omega}{c\rho}}.
			\end{equation}
			Über weitere Formeln wird die Gleichung
			\begin{equation}
				\label{waermeleitfaehigkeit}
				\kappa=\frac{c\rho(\upD{x})^2}{2\upD{t}\ln\left({\frac{A_\text{nah}}{A_\text{fern}}}\right)}
			\end{equation}
			gefunden, die die Wärmeleitfähigkeit mit Messgrößen knüpft.
			$\frac{A_\text{nah}}{A_\text{fern}}$ ist das Verhältnis der Wellenamplituden, welche an zwei Orten $x_1$ und $x_2$ im Abstand
			$\upD{x}=x_2-x_1$ gemessen werden und den zeitlichen Gangunterschied $\upD{t}$ aufweisen. 
        \end{Theorie}
        
        \begin{Durchführung}
            Thermoelemente und Peltier-Elemente sind Gegenspieler.
            Mithilfe eines Peltier-Elementes 
            (zwei Halbleiter vers. Energieniveaus)
            kann geheizt und gekühlt werden, 
            indem Wärme per Peltier-Effekt transportiert wird.
            Die Heizwirkung erfolgt \emph{nicht} durch die thermische Wirkung von Strom.
            Thermoelemente geben eine Spannungsdifferenz aus, 
            wenn zwischen den Temperaturfühlern eine Temperaturdifferenz besteht.
            Die Spannungsdifferenz kann durch Eichung/Kalibration in eine Temperatur umgerechnet werden.
            Alle Messungen startet bei abgekühlten Stäben.

            Bei statischer Messung werden die benutzten Stäbe aus vers. Metallen durchgehend geheizt und die Temperaturentwicklung entlang des Stabes untersucht.
            Bei dynamischer Messung werden im 
            $\SI{40}{\second}$-Wechsel (Kurzintervall-Messung) oder im
            $\SI{200}{\second}$-Wechsel (Langintervall-Messung) 
            die Stäbe gekühlt und geheizt.
        \end{Durchführung}        
       
        \begin{Auswertung}
        	Statische Messung:
        	\begin{itemize}
        		\item Der allgemeine Verlauf der Temperatur ist unabhängig von dem Material: begrenztes exp. Wachstum.
        		\item Durch schmale Stäbe wird weniger Wärme transporiert als durch breite.
        		\item Je höher die Leitfähigkeit, desto schneller erfolgt der Wärmeaustausch
        	\end{itemize}
        	Dynamische Messung:\\
        	Der Temperaturverlauf besteht aus Schwingterm und exponentiellem Ansteigen: der Stab wärmt sich im Mittel während der Messung langsam auf.
       		Nach Subtraktion der exponentiellen Steigung kann die globale Staberwärmung bestimmt und herausgerechnet werden.
        	Dies ermöglicht eine präzise Bestimmung der Amplitude
        	\begin{itemize}
        		\item Die Temperaturamplitude wird innerhalb des Stabes gedämpft, die Frequenz bleibt gleich, es ist eine Phasendifferenz erkennbar.
        		\item Die Amplitudendämpfung ist bei Edelstahl sehr stark, bei Alu sehr gering.
        	\end{itemize}
        \end{Auswertung}

        \begin{Diskussion}
        	\begin{itemize}
        		\item Es ergeben sich Abweichungen im Rahmen einer brauchbaren Messung, Abweichung im Bereich von $10\%$. 
        		\item Aluminium in der Auswahl der Metalle die größte und Edelstahl die geringste Wärmeleitfähigkeit.
        		\item Literaturwerte\\
        		Messing:\SI{120}{\watt\per\meter\per\kelvin} \quad Aluminium:\SI{236}{\watt\per\meter\per\kelvin}\quad Edelstahl:\SI{15}{\watt\per\meter\per\kelvin}
        	\end{itemize}
            \end{Diskussion}

        \begin{Merke}
        	Datenlogger sind doof.
        \end{Merke}


    \end{Versuch}
    
\end{document}