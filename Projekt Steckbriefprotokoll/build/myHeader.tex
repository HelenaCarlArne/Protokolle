\documentclass[
    parskip=half,
    bibliography=totoc,     % Literatur im Inhaltsverzeichnis
    captions=tableheading,  % Tabellen�berschriften
    titlepage=firstiscover, % Titelseite ist Deckblatt
    ]{scrartcl}
    
\usepackage[top=2cm, bottom=4cm, left=2cm, right=2cm]{geometry}
\usepackage{color}
\usepackage[usenames,dvipsnames]{xcolor}
\definecolor{light-red}{HTML}{FFBABA}

% LaTeX2e korrigieren.
\usepackage{fixltx2e}

% Warnung, falls nochmal kompiliert werden muss
\usepackage[aux]{rerunfilecheck}

% Deutsche Spracheinstellungen
\usepackage{polyglossia}
\setmainlanguage{german}

% Unverzichtbare Mathe-Befehle
\usepackage{amsmath}

% Viele Mathe-Symbole
\usepackage{amssymb}

% Erweiterungen f�r amsmath
\usepackage{mathtools}

% Fonteinstellungen
\usepackage{fontspec}
\defaultfontfeatures{Ligatures=TeX}

\usepackage[
    math-style=ISO,    % \
    bold-style=ISO,    % |
    sans-style=italic, % | ISO-Standard folgen
    nabla=upright,     % |
    partial=upright,   % /
    ]{unicode-math}

\setmathfont{Latin Modern Math}
\setmathfont[range={\mathscr, \mathbfscr}]{XITS Math}
\setmathfont[range=\coloneq]{XITS Math}
\setmathfont[range=\propto]{XITS Math}

% Das hquer-Symbol versch�nern
\let\hbar\relax
\DeclareMathSymbol{\hbar}{\mathord}{AMSb}{"7E}
\DeclareMathSymbol{?}{\mathord}{AMSb}{"7E}

% Richtige Anf�hrungszeichen
\usepackage[autostyle]{csquotes}

% Zahlen und Einheiten
\usepackage[
  locale=DE,                   % Deutsche Einstellungen
  separate-uncertainty=true,   % Immer Fehler mit \pm
  per-mode=symbol-or-fraction, % m/s im Text, sonst Br�che
]{siunitx}

% Chemische Formeln
\usepackage[version=3]{mhchem}

% Sch�ne Br�che im Text
\usepackage{xfrac}

% Floats innerhalb einer Section halten
\usepackage[section, below]{placeins}

% Captions sch�ner machen.
\usepackage[
    labelfont=bf,        % Tabelle x: Abbildung y: ist jetzt fett
    font=small,          % Schrift etwas kleiner als Dokument
    width=0.9\textwidth, % Maximale Breite einer Caption schmaler
    ]{caption}

% Subfigure, subtable, subref
\usepackage{subcaption}

% Grafiken einbinden
\usepackage{graphicx}

% Gr��ere Variation von Dateinamen m�glich
\usepackage{grffile}

% Standardplatzierung f�r Floats einstellen
\usepackage{float}
\floatplacement{figure}{htbp}
\floatplacement{table}{htbp}

% Sch�ne Tabellen
\usepackage{booktabs}

% Seite drehen f�r breite Tabellen
\usepackage{pdflscape}

% Literaturverzeichnis
\usepackage{biblatex}

% Quellendatenbank
\addbibresource{lit.bib}
\addbibresource{programme.bib}

% Hyperlinks im Dokument
\usepackage[
    unicode,
    pdfusetitle,    % Titel, Autoren und Datum als PDF-Attribute
    pdfcreator={},  % PDF-Attribute s�ubern
    pdfproducer={}, % "
    ]{hyperref}
    
% Erweiterte Bookmarks im PDF
\usepackage{bookmark}

% Trennung von W�rtern mit Strichen
\usepackage[shortcuts]{extdash}

% Blindtext erzeugen
\usepackage{blindtext}

% Support f�r mdframed
\usepackage{kvoptions}
\usepackage{xparse}
\usepackage{etoolbox}
\usepackage{tikz}

% Sch�ne mehrseitige Rahmen um Text erzeugen
\usepackage{mdframed}
\mdfsetup{skipabove=\topskip,skipbelow=\topskip}