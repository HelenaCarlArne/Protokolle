\section{Durchführung}
\label{sec:Durchfuehrung}
\subsection{Wärmekapazität des Dewar-Gefäßes}
Um die Wärmekapazität des Dewar-Gefäßes berechnen zu können, müssen die in Gleichung \eqref{c_Dewar} vorkommenden Größen bestimmt werden.
Zuerst werden ein Becherglas und das Dewar-Gefäß zu gleichen Teilen mit etwa $\SI{0.3}{\liter}$ Wasser gefüllt. 
Die Wassermassen $m_\mathup{y}$ (Becherglas) und $m_\mathup{x}$ (Dewar-Gefäß) werden mit Hilfe einer Schnellwaage bestimmt, indem die Differenz zwischen Leermassen und dem Gewicht der gefüllten Gefäße gebildet wird.
Die Wassertemperatur $T_\mathup{x}$ des Wassers im Dewar-Gefäß wird mit einem digitalen Thermometer bestimmt; das Wasser im Becherglas wird mit einer Heizplatte erhitzt, bevor ebenfalls die Temperatur $T_\mathup{y}$ gemessen wird. 
Anschließend wird das warme Wasser in das Dewar-Gefäß gegossen und die Mischtemperatur $T_\mathup{m}$ gemessen, welche sich nach kurzer Zeit in Form eines stationären und nur geringen Schwankungen unterlegenen Wertes einstellt.  
Um diesen Prozess zu beschleunigen wird das Wasser mit einem Rührfisch gemischt.

\subsection{Spezifische Wärmekapazität der Probekörper}
Es werden Blei, Graphit und Kupfer als Probekörper gewählt.
Die Probe wird in ein Wasserbad gehängt und auf eine Temperatur $T_\mathup{k}\approx80\si\celsius$ erhitzt. 
Währenddessen wird das Dewar-Gefäß mit Wasser der Temperatur $T_\mathup{W}$ und Masse $m_\mathup{W}$ gefüllt. 
Nach Messen von $T_\mathup{k}$ wird die Probe in das Wasser des Dewar-Gefäßes eingetaucht und die Wassertemperatur $T_\mathup{m}$ so lange gemessen, bis keine maßgeblichen Schwankungen mehr auftreten und die Probe mit dem Wasser im thermischen Gleichgewicht steht.
Die Probenmassen werden ebenfalls mit Hilfe der Schnellwaage bestimmt.
Der gesamte Messvorgang wird für Blei drei Mal durchgeführt, die restlichen Proben werden einmal untersucht.
