\section{Auswertung}
\label{sec:Auswertung}
\subsection{Wärmekapazität des Dewar-Gefäßes}
\begin{table}[ht]
	\centering
	\begin{tabular}{ccccc}%c}
		\toprule
		{Masse $m_\text{x}$}& {Masse $m_\text{y}$}  &{Temperatur  $T_\text{x}$}&{Temperatur $T_\text{y}$}  &{Temperatur $T_\text{m}$}\\
		{[$\si{\gram}$]}& {[$\si{\gram}$]}  &{[$\si{\kelvin}$]}& {[$\si{\kelvin}$]}  &{[$\si{\kelvin}$}]\\
		\midrule
		%566 & 
		296 & 270 & 294.75 & 364.35 & 325.15\\
		\bottomrule
	\end{tabular}
	\caption{Messwerte der Kalorimetrie des Dewargefäßes.}
	\label{tab:messung1}
\end{table}
Zur Bestimmung der Wärmekapazität des verwendeten Dewar-Gefäßes, deren Wert für die folgenden Berechnungen notwendig ist,
wird die Temperatur $T_\text{x}$ und die Masse $m_\text{x}$ der Wassermenge im Gefäß, die Temperatur $T_\text{y}$ und die Masse $m_\text{y}$ der hinzugefügten Wassermenge und die Mischungstemperatur $T_m$ bestimmt.
Es ergeben sich dadurch die Werte für die Bestimmung der Wärmekapazität in Tabelle \ref{tab:messung1}.

Gemäß der Formel \eqref{c_Dewar} und mit dem Literaturwert der spezifischen Wärmekapazität von Wasser
$c_\text{w}=\SI{4.18}{\joule\per\gram\per\kelvin}$\cite{V201} ergibt sich die Wärmekapazität des Dewargefäßes,
\begin{equation}
	c_\mathup{g}m_\mathup{g}=\SI{218.02}{\joule\per\kelvin}.
	\label{wert:waerme_dewar}
\end{equation}

\subsection{Wärmekapazität und Molwärme der Proben}
\begin{table}[htp]
	\centering
	\begin{tabular}{lccccc}
		\toprule
		& {$m_\text{w}$} & {$m_\text{k}$} & {$T_\text{w}$} & {$T_\text{k}$} & {$T_\text{m}$}\\
		&[{$\si{\gram}$}]& {[$\si{\gram}$]}  &{[$\si{\kelvin}$]}& {[$\si{\kelvin}$]}  &{[$\si{\kelvin}$]}\\
		\midrule
		Blei		&574&	385.55	& 296.65 &	356.05&	299.05\\
					&537&	385.55	& 294.95 &	364.15&	298.35\\
					&583&	385.55	& 295.45 &	348.85&	297.65\\	
		Graphit		&544&	107.23	& 295.65 &	356.45&	298.75\\
		Kupfer		&562& 	237.95	& 295.25 &	355.45&	299.55\\
		\bottomrule
	\end{tabular}
	\caption{Messwerte der Kalorimetrie der Proben.}
	\label{tab:messung2}
\end{table}
$m_\text{k}$ ist die Masse der Probe, $T_\text{k}$ ist die Temperatur der Probe, nachdem sie erhitzt und bevor sie in das Dewargefäß gelassen wird.
$m_\text{w}$ ist die Masse des Wassers im Dewar-Gefäß, $T_\text{w}$ ist seine Temperatur.
Es wird die Formel \eqref{c_Probe}
benutzt, um die spezifische Wärmekapazität der Proben zu ermitteln.
Die Messwerte sind in der Tabelle \ref{tab:messung2} aufgetragen.

Es wird der Zusammenhang zwischen den beiden Festlegungen für Molwärmen $C_\text{p}$ und $C_\text{V}$ nach Gleichung \eqref{alphakappalpha} benutzt, um
die experimentell schwierig zu bestimmende Molwärme $C_\text{V}$ mit den errechneten Werten von $C_\text{p}$ für jede Probe zu bestimmen.
\begin{table}[htbp]
	\centering
	\begin{tabular}{lccc}
		\toprule
		&{spez. Wärmekapazität $c_\text{k}$}& {Molwärme $C_\text{p}$}  &{Molwärme $C_\text{V}$}\\
		&{[$\si{\joule\per\kelvin\per\gram}$]}  &{[$\si{\joule\per\kelvin\per\mol}$]}& {[$\si{\joule\per\kelvin\per\mol}$]}\\
		\midrule
		{Blei}	&0.286	&59.225	& 57.490\\
				&0.330	&68.386	& 66.655\\
				&0.296	&61.308	& 59.581\\
		{Blei, gemittelt}	&$0.3\pm0.1$	&$63\pm3$	&$61\pm3$\\
			%{Blei, gemittelt}	&$0.31\pm0.08$	&$63.7\pm16.5$	&$61.242\pm2.773$\\
		{Graphit}&1.249 &14.983& 14.952\\
		{Kupfer}&0.83	&89.717	& 88.983\\
		\bottomrule
	\end{tabular}
	\caption{Ermittelte Eigenschaften der Proben.}
	\label{wert:proben}
\end{table}
Durch mehrfaches Messen bei der Blei-Probe ist eine Fehlerrechnung möglich.
Via Mittelwertbildung und der Bestimmung der Standardabweichung des Mittelwertes
\begin{subequations}
	\begin{equation}
		\sigma_x = \sqrt{\sigma_x^2} := \sqrt{\frac{1}{n^2-n} \sum_{i=1}^n{(x_i-\bar{x})^2}}
	\end{equation}
	\begin{equation}
		\bar{x} = \frac{1}{n} \sum_{i=1}^n{x_i}.
	\end{equation}
\end{subequations}
mit $x\in\{c_\mathup{k},C_\mathup{p},C_\mathup{V}\}$ ergeben sich für Blei die gemittelten Werte in Tabelle \ref{wert:proben}
\subsection{Vergleich mit dem Dulong--Petit-Gesetz und der Literatur}
\begin{table}[htbp]
	\centering
	\begin{tabular}{lccc}
		\toprule
		&\multicolumn{3}{c}{Spezifische Wärmekapazität $c_\text{k}$ der Proben}\\
		&{Gemessen [$\si{\joule\per\kelvin\per\gram}$]}  &{Literatur [$\si{\joule\per\kelvin\per\gram}$]}& {Abweichung [$\%$]}\\
		\midrule
		{Blei}	&0.286	&0.129	&121.71\\
				&0.330	&0.129	&155.81\\
		 		&0.296	&0.129	&129.46\\
		{Blei, gemittelt}	&$0.3\pm0.1$	&0.129	&140.31\\
		{Graphit}&1.249 &0.715& 74.69\\
		{Kupfer}&0.83	&0.381	& 117,82\\
		\bottomrule
	\end{tabular}
	\caption{Abweichung der spezifischen Wärme von der Literatur. \cite{waermekapazitaetkupfer} \cite{waermekapazitaetblei} }
	\label{tab:compare}
\end{table}
\begin{table}[htbp]
	\centering
	\begin{tabular}{ccc}
		\toprule
		\multicolumn{3}{c}{Abweichung der Molwärme $C_\text{V}$ [$\%$]}\\
		{Blei, gemittelt}	&{Graphit}&{Kupfer}\\
		\midrule
		146	&40.54	&108.33\\
		\bottomrule
	\end{tabular}
	\caption{Abweichungen vom erwarteten Wert nach Dulong--Petit-Gesetz.}
	\label{tab:failz}
\end{table}
Die Abweichungen der spezifischen Wärmekapazitäten der Proben von den Literaturwerten sind in Tabelle \ref{tab:compare} aufgetragen.
Nach Kapitel \ref{sec:Theorie} werden die ermittelten Molwärmen $C_\mathup{V}$ bei konstantem Volumen mit dem nach dem Dulong--Petit-Gesetz in Gleichung \eqref{eq:dulong-petit} erwarteten Wert verglichen.
Die Abweichungen sind in Tabelle \ref{tab:failz} aufgetragen. 
