\section{Ziel}

Versuchsziel ist es, herauszufinden, ob die Methoden der klassischen Physik zur Beschreibung von Teilchenschwingungen in Festkörpern ausreichen oder ob diese quatenmechanisch beschrieben werden müssen. Dazu wird mit einem Mischungskalorimeter die material- und temperaturunabhängige Molwärme bestimmt, die klassisch nach dem Dulong-Petit-Gesetz konstant $C_\mathup{V}=3R$ beträgt.
\section{Theorie}
\label{sec:Theorie}

Wird ein Körper  um eine Temperatur $\Delta{T}$ erhitzt, wird dazu die Wärmemenge 
\begin{equation}
\Delta{Q}=c m \Delta{T}
\label{eq:waermekapazitaet}
\end{equation}
aufgewandt. Die materialabhängige Konstante $c$ bezeichnet die Wärmekapazität. Bezogen auf die erwärmte Masse $m$ des Körpers, wird von der spezifischen Wärmekapazität gesprochen.
Wenn keine Arbeit zum Erwärmen des Körpers aufgebracht wird kann nach dem 1. Hauptsatz der Thermodynamik
\begin{equation}
	\Delta{U}=\Delta{Q}+\Delta{A}
	\label{eq:hs_1}
\end{equation}
mit $\Delta{A}=0$ die Wärmemenge auch als Energieform aufgefasst werden. Damit ist die Einheit der Wärmekapazität $[c]=\si{\joule\per{\kilo\gram\kelvin}}$.

Erhitzt man $\SI{1}{\mol}$ Atome mit  d$Q$ um d$T$ kann das bei konstantem Druck oder bei konstantem Volumen geschehen.
 Dann ist die aufgewandte Molwärme $C_\mathup{p}$ bzw. $C_\mathup{V}$. 
Verwendet man erneut Gleichung \eqref{eq:hs_1} 
ergibt sich für ein konstantes Volumen 
\begin{equation}
	C_\mathup{V}=\frac{\mathup{d}{Q}}{\mathup{d}{T}}=\frac{\mathup{d}{U}}{\mathup{d}{T}}.
	\label{eq:molwaerme}
\end{equation}

Diese Tatsache erklärt die kinetische Theorie der Wärme; sie beschreibt makroskopische Vorgänge durch das mikroskopische Verhalten von Atomen, indem die einzelnen Energien der Atome summiert und anschließend gemittelt werden. 
Daraus ergibt sich die gemittelte innere Energie
\begin{equation}
	\bar{U}=\bar{E}_\mathup{kin.}+\bar{E}_\mathup{pot.}.
	\label{eq:innere_Energie}
\end{equation}
Wird die Schwingung eines einzelnen Atomes betrachtet, schwingt es aufgrund zweier Kräfte -- Trägheits- und rücktreibende Kraft -- um seine Ruhelage. 
\begin{equation}
	F_\mathup{T}+F_\mathup{R}=m\ddot{x}+Dx=0.
\end{equation}
Diese Bewegungsgleichung beschreibt einen harmonischen ungedämpften Oszillator. 
Durch Integration ergibt sich sowohl für die potentielle, als auch für die kinetische Energie $\bar{E}_\mathup{pot.}=\frac{1}{4}DA²=\bar{E}_\mathup{kin.}$ mit der Amplitude $A$ der Lösung der Differentialgleichung.
Eingesetzt in Gleichung \eqref{eq:innere_Energie}
 folgt daraus mit dem Äquipatitionstheorem $\bar{U}=2\bar{E}_\mathup{kin.}=kT$. 
Jenes besagt, dass ein Atom im thermischen Gleichgewicht mit seiner Umgebung eine kinetische Energie $\bar{E}_\mathup{kin.}=kT$ pro Bewegungsfreiheitsgrad besitzt. 
$k=(1.38\cdot{10⁻²³}\pm9.1\cdot{10⁻⁷})\si{\joule\per\kelvin}$ ist die Boltzmann-Konstante.
Wird nun erneut $1\si\mol$ Atome betrachtet, die jeweils drei Freiheitsgrade besitzen ergibt sich für die gemittelte innere Energie 
\begin{equation}
\tilde{\bar{U}}=3N_\mathup{L}\bar{U}=3N_\mathup{L}kT=3R.
\end{equation}
$N_\mathup{L}=(2.687\cdot 10²⁵\pm0.0000024)\si{\meter}⁻³$ ist die Lohschmidtsche Zahl und gibt die Anzahl der Atome innerhalb einer Volumeneinheit an. 
Wird der Ausdruck für die gemittelte innere Energie der Atome in Gleichung \eqref{eq:molwaerme} eingesetzt ist
\begin{equation}
C_\mathup{V}=\frac{3RT}{\mathup{d}T}=3R.
\end{equation}
Dies ist die Aussage des Dulong-Petit-Gesetzes der klassischen Physik : Bei konstantem Volumen beträgt die Molwärme $C_\mathup{V}=3R$; $R=(8.314\pm9.1\cdot{10⁻⁷})\si{\joule\per{\kilo\gram\kelvin}}$ ist dabei die allgemeine Gaskonstante.
Die Molwärme ist für die meisten Elemente schon bei $T=20\si{\celsius}$ konstant -- nur für Atome mit gerigem Atomgewicht wie Blei oder Bohr gilt dies erst ab $T\approx{1000\si{\celsius}}$. Für sehr geringe Temperaturen $\lim_{\mathclap{T \to{0}}}C_\mathup{V}=0$. 


Die klassische Physik nimmt an, dass die Energieaufnahme und -abgabe kontinuierlich geschieht. Die Quantentheorie revidiert diese Annahme: Schwingt ein Oszillator mit der Frequenz $\omega$ werden nur Energiepakete der Form $\Delta{U}=n \hbar \omega$ abgegeben, wobei $n$ Element der natürlichen Zahlen ist. 
Die mittlere innere Energie aller Atome ist nicht mehr proportional zur Temperatur $T$ und es ergibt sich die komplizierte Abhängigkeit
\begin{equation}
{\bar{U}=\hbar\omega(\exp(\frac{\hbar\omega}{kT-1})})⁻¹,
\end{equation}

die Energie dargestellt als geometrische Reihe.
\begin{equation}
\tilde{{\bar{U}}}=3N_\mathup{L}\bar{U}.
\end{equation}

 Analog zur klassischen Mechanik eingesetzt in \eqref{eq:molwaerme} ergibt sich 
\begin{equation}
C_\mathup{V}=\frac{3N_\mathup{L}\bar{U}}{\mathup{d}T}.
\end{equation}

Für den Fall geringer Temperaturen gilt dasselbe wie in der klassischen Physik $\lim_{\mathclap{T \to{0}}}C_\mathup{V}=0$, im Falle hoher Temperaturen gilt $\lim_{\mathclap{T \to{infinity}}}C_\mathup{V}=3R$, wenn die e-Funktion mit der Taylor-Entwicklung genähert wird.

Dies zeigt, dass das Dulong-Petit-Gesetz für extrem hohe Temperaturen einer geeigneten Näherung entspricht und der quantisierte Energieaustausch durch einen kontinuierlichen dargestellt werden kann. Dies ist dann der Fall, wenn $\hbar\omega<<kT$ ist. 
Für geringe Atomgewichte gilt die Näherung erst für entsprechend große Temepraturen, weil die Frequenz $\omega\sim \frac{1}{\sqrt{m}}$ ist.

Im Versuch soll die Molwärme bei konstantem Druck $C_\mathup{p}$ gemessen werden. Diese steht über
\begin{equation}
C_\mathup{p}-C_\mathup{V}=9{\alpha}²\kappa V_0 T
\end{equation}
mit der Molwärme bei konstantem Volumen $C_\mathup{V}$ in Beziehung. $\alpha$ ist der Ausdehnungskoeffizient, $\kappa$ das Kompressionsmodul und $V_0$ das Molvolumen.
Außerdem wird vom Idealfall ausgegangen, dass die aufgenommene Wärmemenge $Q_1$ gleich der abgegebenen Wärmemenge $Q_2$ ist und kein weiterer Wärmeaustausch mit der Umgebung stattfindet.
Mit GLeichung \eqref{eq:waermekapazitaet} und einigen Umformungen ergibt sich
\begin{equation}
c_\mathup{k}=\frac{(c_\mathup{w}m_\mathup{w}+c_\mathup{g}m_\mathup{g})(T_\mathup{m}-T_\mathup{w})}{m_\mathup{k}(T_\mathup{k}-T_\mathup{m})}
\end{equation}
für die Wärmekapazität der Probe. 
In einer seperaten Messung muss zuvor die spezifische Wärmekapazität 
\begin{equation}
c_\mathup{g}m_\mathup{g}=\frac{c_\mathup{w}m_\mathup{y}(T_\mathup{y}-T'_\mathup{m})-c_\mathup{w}m_\mathup{x}(T'_\mathup{m}-T_\mathup{x})}{(T'_\mathup{m}-T_\mathup{x})}
\end{equation}
des Dewar-Gefäßes bestimmt werden. 
Die Molwärme kann über $C=cM$ mit der molaren Masse $M$ bestimmt werden.


