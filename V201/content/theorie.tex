\section{Ziel}

Ziel des Versuches ist es, herauszufinden, ob die Methoden der klassischen Physik zur Darstellung von Teilchenschwingungen in Festkörpern ausreichen oder ob diese quantenmechanisch beschrieben werden müssen. Dazu wird mit einem Mischungskalorimeter die material- und temperaturunabhängige Molwärme verschiedener Proben bestimmt und anschließend die Gültigkeit beider Theorien bewertet.
\section{Theorie}
\label{sec:Theorie}

Wird ein Körper  um eine Temperatur $\Delta{T}$ erhitzt, wird dazu die Wärmemenge 
\begin{equation}
\Delta{Q}=c m \Delta{T}
\label{eq:waermekapazitaet}
\end{equation}
aufgewandt. Die vom Material abhängigen Faktoren $c m$ werden Wärmekapazität genannt. 
Unabhängig von der erwärmten Masse $m$ des Körpers wird von der spezifischen Wärmekapazität $c$ gesprochen.
Wenn keine Arbeit zum Erwärmen des Körpers aufgebracht wird, kann nach dem 1. Hauptsatz der Thermodynamik
\begin{equation}
	\Delta{U}=\Delta{Q}+\Delta{A}
	\label{eq:hs_1}
\end{equation}
mit $\Delta{A}=0$ die Wärmemenge auch als Energieform aufgefasst werden. 
Damit ist die Einheit der Wärmekapazität $[c]=\si{\joule\per\kilo\gram\per\kelvin}$.
Erhitzt man $\SI{1}{\mol}$ Atome mit d$Q$ um d$T$ kann das bei konstantem Druck oder Volumen geschehen.
Die hierzu aufgewandten Wärmen werden als Molwärme $C_\mathup{p}$ bzw. $C_\mathup{V}$ bezeichnet. 
Verwendet man erneut Gleichung \eqref{eq:hs_1} 
ergibt sich für ein konstantes Volumen 
\begin{equation}
	C_\mathup{V}=\frac{\mathup{d}{Q}}{\mathup{d}{T}}=\frac{\mathup{d}{U}}{\mathup{d}{T}}.
	\label{eq:molwaerme}
\end{equation}
Die kinetische Theorie der Wärme beschreibt makroskopische Vorgänge durch das mikroskopische Verhalten von Atomen, indem die einzelnen Energien der Atome summiert und anschließend gemittelt werden. 
Die gemittelte Gesamtenergie setzt sich aus der potentiellen und kinetischen Energie der Atome zusammen:
\begin{equation}
	\bar{U}=\bar{E}_\mathup{kin.}+\bar{E}_\mathup{pot.}.
	\label{eq:innere_Energie}
\end{equation}
Wird die Schwingung eines einzelnen Atomes betrachtet, oszilliert es aufgrund zweier Kräfte -- Trägheits- und rücktreibende Kraft -- um seine Ruhelage. 
\begin{equation}
	F_\mathup{T}+F_\mathup{R}=m\ddot{x}+Dx=0.
\end{equation}
Diese Bewegungsgleichung beschreibt einen harmonischen ungedämpften Oszillator. 
Durch Integration der Lösung ergibt sich sowohl für die potentielle, als auch für die kinetische Energie $\bar{E}_\mathup{pot.}=\frac{1}{4}DA²=\bar{E}_\mathup{kin.}$ mit der Schwingungsamplitude $A$.
Eingesetzt in Gleichung \eqref{eq:innere_Energie}
folgt daraus mit dem Äquipatitionstheorem, welches einem Atom im thermischen Gleichgewicht mit seiner Umgebung die kinetische Energie $\bar{E}_\mathup{kin.}=\frac{1}{2}kT$ pro Bewegungsfreiheitsgrad zuordnet, $\bar{U}=2\bar{E}_\mathup{kin.}=kT$. 
Werden nun erneut $\SI{1}{\mol}$ Atome mit je drei Freiheitsgraden betrachtet, ergibt sich für die gemittelte innere Energie 
\begin{equation}
	\tilde{\bar{U}}=3N_\mathup{L}\bar{U}=3N_\mathup{L}kT=3\text{R}T.
\end{equation}
Die allgemeine Gaskonstante $\text{R}$\cite{Gaskonstante} setzt sich aus der Boltzmann-Konstante \\$k=(1.38\cdot{10⁻²³}\pm9.1\cdot{10⁻⁷})\,\si{\joule\per\kelvin}$\cite{Boltzmannkonstante} und 
der Loschmidtschen Zahl \\$N_\mathup{L}=(2.687\cdot 10²⁵\pm0.0000024)\,\si{\meter}⁻³$\cite{Loschmidtzahl} zusammen,
$N_\mathup{L}$ gibt die Anzahl der Atome innerhalb einer Volumeneinheit an. 
Ihr Wert beträgt $\text{R}=(8.314\pm9.1\cdot{10⁻⁷})\,\si{\joule\per\kilo\gram\per\kelvin}$.

Wird der Ausdruck für die gemittelte innere Energie der Atome in Gleichung \eqref{eq:molwaerme} eingesetzt, ist
\begin{equation}
	C_\mathup{V}=\frac{3\text{R}T}{\mathup{d}T}=3\text{R}.
\label{eq:dulong-petit}
\end{equation}
Dies ist die Aussage des Dulong--Petit-Gesetzes der klassischen Physik.\\

Im Versuch soll die Molwärme bei konstantem Druck $C_\mathup{p}$ gemessen werden. 
Diese steht über
\begin{equation}
	C_\mathup{p}-C_\mathup{V}=9{\alpha}²\kappa V_0 T
	\label{alphakappalpha}
\end{equation}
mit der Molwärme bei konstantem Volumen $C_\mathup{V}$ in Beziehung. 
$\alpha$ ist dabei der Ausdehnungskoeffizient, $\kappa$ das Kompressionsmodul und $V_0$ das Molvolumen.
Außerdem wird vom Idealfall ausgegangen, die aufgenommene Wärmemenge $Q_1$ ist gleich der abgegebenen Wärmemenge $Q_2$, d.h. 
es findet kein weiterer Wärmeaustausch mit der Umgebung statt.
Die Molwärme hängt über $C=cM$ mit der molaren Masse $M$ mit der Wärmekapazität zusammen.
Mit Gleichung \eqref{eq:waermekapazitaet} und weiteren Umformungen ergibt sich
\begin{equation}
	c_\mathup{k}=\frac{(c_\mathup{w}m_\mathup{w}+c_\mathup{g}m_\mathup{g})(T_\mathup{m}-T_\mathup{w})}{m_\mathup{k}(T_\mathup{k}-T_\mathup{m})}
	\label{c_Probe}
\end{equation}
für die Wärmekapazität der Probe. 
In einer seperaten Messung wird zuvor die Wärmekapazität 
\begin{equation}
	c_\mathup{g}m_\mathup{g}=\frac{c_\mathup{w}m_\mathup{y}(T_\mathup{y}-T'_\mathup{m})-c_\mathup{w}m_\mathup{x}(T'_\mathup{m}-T_\mathup{x})}{(T'_\mathup{m}-T_\mathup{x})}
	\label{c_Dewar}
\end{equation}
des Dewar-Gefäßes bestimmt werden. 

