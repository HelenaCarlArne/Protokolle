\section{Diskussion}
\label{sec:Diskussion}
\subsection{Messfehler und Unsicherheiten}
Der Versuch ist sehr anfällig gegenüber ungewollter Abkühlung und Beeinflussung durch die Versuchssituation.
Dies ist in der Abweichung der spezifischen Wärmekapazitäten in Tabelle \ref{tab:compare} von den Literaturwerten sichtbar.

Zur Optimierung des Versuches muss um den Einfluss der Raumtemperatur auf die Messwerte Sorge getragen werden.

\subsection{Abweichungen von dem Dulong--Petit-Gesetz}
Die in Tabelle \ref{tab:failz} aufgetragenen Abweichungen der ermittelten Molwärmen von den Vorhersagen des Dulong--Petit-Gesetzes nach Gleichung \eqref{eq:dulong-petit} zeigen, dass die reale Molwärme nicht durch $C_\text{V}=3\text{R}$ beschrieben werden kann. 
Insbesondere das "quantenmechanisch interessante" Material Graphit zeigt eine starke Abweichung von der Vorhersage.
Es zeigt die Grenzen des Dulong--Petit-Gesetzes experimentell auf.

\subsection{Grenzen des Dulong--Petit-Gesetzes}
% Dies habe ich nicht recht verstanden.
% Bitte kontrollieren! :)
Die Molwärmen der meisten Elemente bei Raumtemperatur $T=20\si{\celsius}$ sind konstant.
Im Kontrast ist für Atome mit geringem Atomgewicht wie Beryllium oder Bor die Konstanz der Molwärme erst bei höheren Temperaturen $T\approx{1000\si{\celsius}}$ sichtbar. 
Weiter verschwindet für sehr geringe Temperaturen die Molwärme.

Die klassische Physik nimmt an, dass die Energieaufnahme und -abgabe kontinuierlich geschieht.
Die Quantentheorie revidiert diese Annahme: 
schwingt ein Oszillator mit der Frequenz $\omega$, so werden nur Energiepakete, sogenannte Quanten, der Größe $\Delta{U}=n \hbar \omega$ mit $n\in\mathbb{N} $ abgegeben.
Die mittlere innere Energie aller Atome ist nicht mehr proportional zur Temperatur $T$ und es ergibt sich eine komplizierte Abhängigkeit indem das Integral über die Auftrittswahrscheinlichkeit der verschiedenen Energiepakete, die Boltzmannverteilung, multipliziert mit deren Wert gebildet wird:
\begin{equation}
	{\bar{U}=\hbar\omega(\exp(\frac{\hbar\omega}{kT-1})})⁻¹,
	\label{energie}
\end{equation}
die Energie dargestellt als geometrische Reihe.
% Dies habe ich nicht recht verstanden.
% Bitte kontrollieren! :)
Damit ist
\begin{equation}
	\tilde{{\bar{U}}}=3N_\mathup{L}\bar{U}
\end{equation}
und in Gleichung \eqref{eq:molwaerme} eingesetzt ergibt sich, analog zur klassischen Mechanik,    
\begin{equation}
	C_\mathup{V}=\frac{3N_\mathup{L}\bar{U}}{\mathup{d}T}.
	\label{quanten}
\end{equation}
Für den Fall geringer Temperaturen gilt für quantenmechanische Überlegungen in Gleichung \eqref{quanten} $C_\mathup{V}=0$. 
Wird die Exponential-Funktion \eqref{energie} mit einer Taylorreihe entwickelt, nähert sich ihr Wert für große Temperaturen entsprechend dem Dulong--Petit-Gesetz der Konstanten $C_\mathup{V}=3R$.

Dies zeigt, dass das Dulong--Petit-Gesetz für extrem hohe Temperaturen einer geeigneten Näherung entspricht und der quantisierte Energieaustausch durch einen kontinuierlichen Austausch dargestellt werden kann. 
Dies ist dann der Fall, wenn $\hbar\omega<<kT$ ist. 
Für geringe Atomgewichte gilt die Näherung erst für entsprechend große Temperaturen, weil die Frequenz $\omega\sim\frac{1}{\sqrt{m}}$ ist.


