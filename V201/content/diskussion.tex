\section{Diskussion}
\label{sec:Diskussion}
\subsection{Messfehler und Unsicherheiten}
Der Versuch ist sehr anfällig gegenüber ungewollter Abkühlung und Beeinflussung durch die Versuchssituation.
Dies ist in der Abweichung der spezifischen Wärmekapazitäten in Tabelle \ref{tab:compare} von den Literaturwerten sichtbar.\
Zur Optimierung des Versuches muss um den Einfluss der Raumtemperatur auf die Messwerte Sorge getragen werden.

\subsection{Abweichungen von dem Dulong--Petit-Gesetz}
Die in Tabelle \ref{tab:failz} aufgetragenen Abweichungen der ermittelten Molwärmen von den Vorhersagen des Dulong--Petit-Gesetzes nach Gleichung \eqref{eq:dulong-petit} zeigen, dass die reale Molwärme nicht durch $C_\text{V}=3\text{R}$ beschrieben werden kann. 
Es zeigt die Grenzen des Dulong--Petit-Gesetzes experimentell auf.

Nach Abschnitt \ref{sec:TheorieGrenze} beschreibt das Dulong--Petit-Gesetz bei Elementen mit großer Molmasse die Molwärme $C_\text{V}$ bereits bei Raumtemperatur und für Elemente mit kleiner Molmasse erst für hohe Temperaturen, etwa $\SI{1000}{\degreeCelsius}$, gut. 
Dies kann mit diesem Versuch nicht bestätigt werden;
die Temperaturen, die für diesen Effekt erforderlich sind, sind wesentlich größer als die im Versuch auftretenden Temperaturen.