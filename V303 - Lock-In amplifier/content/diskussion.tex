\section{Diskussion}
\label{sec:Diskussion}

\subsection{Ergebnis des Versuches}
%Der Lock-In-Verstärker ist awesome.
Der Lock-In-Verstärker eignet sich hervorragend zur Messung von schwachen und gestörten Signalen.
Das Signal kann dabei Störungen aufweisen, die die Signalstärke wesentlich übertreffen, im Versuch wurde die Störsicherheit bei geringem Verhältnis von Störung zu Signal nachgewiesen (vgl. Tabelle \ref{tab:spannung}).

Im Versuch konnte das \SI{300}{\hertz}-Signal der LED über eine Strecke von etwa $\SI{1}{\meter}$ nachgewiesen werden, wobei keine speziellen Maßnahmen getroffen wurden, um Einflüsse durch Störlicht zu vermeiden.
Die Intensität der LED war dabei nicht wesentlich größer als das Umgebungslicht des Raumes.

\subsection{Anwendung des Lock-In-Verstärkers}
Anwendung findet der Lock-In-Verstärker beispielsweise in der Elektro- und Informationstechnik zur Übermittlung von Daten über langen Strecken.
Bei der Messung von Signalen sollte im Idealfall die Frequenz des Signales bekannt sein, etwa bei amplitudenmodulierten Wellen.
Ist die Frequenz des zu messenden Signales bekannt und der benutzte Bandpass durchlässig für das Spektrum der Frequenzen, so kann das Signal präzise von den Störungen gefiltert und verstärkt werden.
Dies führt zu einem wesentlichen Unterschied zu reinen Bandpass-Filtern:
der Lock-In-Filter ist in der Lage, Signale mit variabler Frequenz zu verstärken und Störung in Frequenzlage zu filtern (vgl. \ref{sec:Auswertung1}). 

Ein anderer Verwendungszweck ist die Bestimmung des Phasenversatzes von zwei Signalen. 
Nach Gleichung \eqref{cosinus_ausgangsspannung} ist die ausgegebene Gleichspannung maximal, wenn die Eingangssignale keinen Phasenversatz aufweisen. 
In Abschnitt \ref{sec:Auswertung1} wird gezeigt, wie sich ein Phasenunterschied auf die resultierende Spannung auswirkt. 